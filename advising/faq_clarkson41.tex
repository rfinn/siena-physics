\documentclass[12pt]{article}
\usepackage{graphics,graphicx,array,soul}
\usepackage[obeyspaces]{url}
\PassOptionsToPackage{obeyspaces}{url}

\usepackage{hyperref}

\headheight=0in
\headsep=0.2in
\topskip=0in
\setlength{\oddsidemargin}{-0.5in}
\setlength{\evensidemargin}{0cm}
\setlength{\topmargin}{-0.5in}
\setlength{\textwidth}{7.0in}
\setlength{\textheight}{9in}

\begin{document}
\hfill \includegraphics[width=0.35\textwidth]{siena_phys_astro_print_crop.jpg}

\vspace{0.2cm}
\begin{center}
{\LARGE {\bf Frequently Asked Questions}} \\
\medskip
{\Large {\bf 4/1 Program with Clarkson University}}  \\
\medskip
{\em Last Updated: August 2018}
\end{center}
\vspace{0.3cm}

%\noindent This is good.

\begin{itemize}
\item{{\bf {\em What is this document? Where can I get for more information?}} 

This document is an unofficial guide for students at Siena College
\emph{considering} the 4/1 Program; it is \emph{not} a replacement for speaking
at length with your academic advisor and with the Associate Dean of Engineering
at Clarkson University, Hugo Irizarry-Quinones (518-631-9882,
hirizarr@clarkson.edu).  At Siena you may also reach out to Prof. Michele
McColgan (mmccolgan@siena.edu) in the Physics Department if you have specific
questions not answered below.  Finally, you should reach out to your peers at
Siena who have already started taking classes at Clarkson as part of the 4/1
Program---their experiences are an invaluable resource!}

\item{{\bf {\em What is the 4/1 program?}}

The 4/1 Program is an articulation between Siena College and Clarkson University
which allows undergraduate students pursuing their BS in Physics, Environmental
Science, Computer Science, Computational Science, or Mathematics to take
graduate-level courses at the Capital Region Campus (CRC) of Clarkson University
in Schenectady, NY, giving them a jump on obtaining their engineering masters
degree in just one additional year (hence, the ``4/1 Program'').  You may take
up to \underline{three} courses at Clarkson as part of your Siena tuition (i.e.,
without paying any additional fees).  In addition, the courses you take at
Clarkson may also satisfy your upper-level major course requirements, although
that depends on the course and your department---see your academic advisor for
details.}

%In the physics department these courses also count toward your
%required upper-level physics courses (as PHYS400).  See your advisor for
%details.} 

\item{{\bf {\em What engineering degrees are available?}}

The three Master of Science (M.S.) programs currently available to Siena
students are \ul{Electrical Engineering} (MSEE), \ul{Engineering and Management
  Systems} (MSEM), and \ul{Energy Systems} (MSE).  Quoting from the program
catalog, the MSEE program ``explores technologies and industry opportunities in
modern electric machinery, modeling and control of power electronics.''  The
MSEM program integrates ``engineering and/or information systems technologies
with the core components of an MBA, including Business of Energy Program
courses.''  And the MSE degree enables ``students to integrate (1)
mechanical/electrical energy related courses, (2) mechanical and electrical
fundamental discipline courses, and (3) non-technical courses regarding the
impact of environmental, economic, and regulatory issues on energy.''
%\textbf{Any other programs?  Business of Energy?}
}

\item{{\bf {\em What about Mechanical Engineering?}}

The mechanical engineering masters track (\ul{Master of Science in Mechanical
  Engineering}, MSME) is an option that Siena College and Clarkson University
are still discussing, so be sure to ask your academic advisor for the latest
information.  However, it is \emph{possible} that students majoring in applied
physics may be able to pursue a masters in mechanical engineering after they
have completed some supplemental undergraduate mechanical engineering courses at
RPI, Clarkson University, or Union College.}

%\item{{\bf {\em What are the job and internship prospects?}}
%Jobs, internships, etc. 
%}

\item{{\bf {\em When can I start taking classes at Clarkson?}}

Students can start taking classes as soon as they have room in their schedule,
typically in their junior or senior year.  Although these are not formal
prerequisites, for the more technical electrical engineering courses (e.g., {\em
  Linear Control Systems}) it is strongly recommended that students have
completed all three semesters of calculus, differential equations, and either
linear algebra or applied mathematics (ideally, both), as well as some
experience with {\tt Matlab}/{\tt Simulink}.  By contrast, some of the
business-oriented courses (e.g., {\em Fundamentals of the Business of Energy})
may be taken without having completed as much math, although they are still
graduate-level courses and require a significant amount of work.  Finally, some
courses like {\em Wind Energy Engineering} are more project-based, and can
therefore be taken as early as a student's junior year.}

\item{{\bf {\em When are the academic terms at Clarkson?}}

Unlike Siena, the Capital Region Campus of Clarkson University is (currently) on
a 10-week trimester system.\footnote{Please be aware that the CRC campus is
  moving to a semester system most likely in Fall 2019, although the courses
  will still likely be the same duration.}  The fall, winter, and spring
trimesters are roughly from early September to late November, early January to
mid-March, and early April to early June, respectively.  (Please note that only
the fall term is aligned with the housing term at Siena; therefore, if you do
not live locally---and need to take winter and/or spring courses---you will need
to consider where you will live while taking courses at Clarkson.)  In addition,
there is typically a summer session of selective classes from mid-June to the
end of August.  The classes typically take place during the week and in the
evening.  The average class size is around 12 students.}

\item{{\bf {\em What classes are offered?  What are the prerequisites?}}

The courses available varies somewhat from year-to-year, so you should ask your
advisor and/or Dean Irizarry-Quinones for the latest course schedule, and
whether you are ready to start taking courses.  Some of the courses our students
have taken (and when they are typically offered) include {\em Electromechanical
  Energy Conversion} (fall, technical), {\em Electronic Power Conversion}
(winter, technical), {\em Linear Control Systems} (fall, technical), {\em
  Fundamentals of Business of Energy} (online, offered frequently), {\em Solar
  Energy Engineering} (winter, project-based), {\em Wind Energy Engineering}
(spring, project-based).}

\item{{\bf {\em How do I register for classes?  What are the minimum
      requirements?}}

A minimum cumulative 3.0 GPA is generally required before a student can take a
graduate-level class, although exceptions may be made on a case-by-case basis.

In order to register for one or more classes you must complete (each term you
are taking classes) the
\underline{\href{http://www.clarkson.edu/sites/default/files/2017-10/CRC\%20Cross\%20Registration\%20Form.pdf}{Registration
    Form}} specifically designed for Siena students.  The second page of this
registration form has detailed instructions which should be followed carefully,
but basically you will need the signature of your academic advisor, the
Assistant Dean of the School of Science, Angela McKeever (Roger Bacon Hall 212,
mckeever@siena.edu), and the Siena College Registrar, Jim Serbalik (Siena Hall
102, serbalik@siena.edu) before it is sent to Clarkson University.  You will
also need to send an unofficial transcript directly to Dean Irizarry-Quinones
for his review and approval.  Be sure to get the registration form signed and
submitted well in advance of the start of classes!} 

\item{{\bf {\em How do I get credit for these classes on my CAPP report?}}

The classes you take at Clarkson as part of the 4/1 Program are cross-registered
with your Siena schedule.  Therefore, they count toward your full-time status as
a student and appear on your schedule, transcript, and CAPP report, without any
additional fees.}

\item{{\bf {\em Are there research opportunities?}}

Unfortunately there are no research opportunities, although you should speak
with Dean Kozik early and often about possible paid and unpaid internships.}

%\item{{\bf {\em How do I get to Clarkson and where can I park?}}
%Answer}

%\item{{\bf {\em Where can I read the articulation agreement between Siena and
%      Clarkson?}}
%link}

\item{{\bf {\em What happened to Union Graduate College?}}

In January 2016 Union Graduate College became the Clarkson University Capital
Region Campus, a satellite campus of Clarkson University which is located in
Potsdam, NY.}
  
\end{itemize}
  
\end{document}

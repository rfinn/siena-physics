\documentclass{tufte-handout}
\pagenumbering{arabic}
\usepackage{amsmath}
\usepackage[pdftex]{graphicx}
\usepackage{color}
\usepackage{enumitem}
\usepackage{natbib}
\usepackage{comment}
\usepackage{bm}
\usepackage{mdwlist}

% The following package makes prettier tables.  We're all about the bling!
\usepackage{booktabs}

% The units package provides nice, non-stacked fractions and better spacing
% for units.
\usepackage{units}

\newcommand{\angstrom}{\text{\normalfont\AA}}

% The fancyvrb package lets us customize the formatting of verbatim
% environments.  We use a slightly smaller font.
\usepackage{fancyvrb}
\fvset{fontsize=\normalsize}

% Small sections of multiple columns
\usepackage{multicol}

\usepackage{hyperref}
\hypersetup{
    colorlinks=true,       % false: boxed links; true: colored links
    linkcolor=red,       % color of internal links
    citecolor=red,        % color of links to bibliography
    filecolor=magenta,      % color of file links
    urlcolor=blue           % color of external links
}

\newcommand{\cs}{${}^{137}_{\ 55}{\rm Cs}$ }
\newcommand{\ba}{${}^{137}_{\ 56}{\rm Ba }$}
\newcommand{\bam}{${}^{137}_{\ 56}{\rm Ba^* }$}

%\pagestyle{myheadings}
%\markright{Fall~2016\hspace{1.9in}Mechanics Lab}


%\documentclass[12pt]{aastex}
%\usepackage{graphicx,mdwlist,longtable,url}
%\usepackage[final]{pdfpages}
%
%\usepackage[title,titletoc,toc]{appendix}
%\usepackage{longtable}
%
%% left-justify the section headings and make them bigger
%\usepackage{titlesec}
%\newcommand*{\justifyheading}{\raggedright}
%\titleformat*{\section}{\Large\bfseries}
%\titleformat*{\subsection}{\large\bfseries}
%
%\renewcommand{\bottomfraction}{0.9}
%
%\setlength{\parindent}{0.cm}
%\setlength{\parskip}{0.3cm}
%\setlength{\topmargin}{-1.3cm}
%\setlength{\textheight}{9in}
%\setlength{\evensidemargin}{0cm}
%\setlength{\textwidth}{6.46in}
%\setlength{\oddsidemargin}{0in}
%
%\pagestyle{myheadings}
%\markright{Fall~2016\hspace{1.9in}Mechanics Lab}
%
%% these make the longtables span the width of \textwidth
%\setlength\LTleft{0pt}
%\setlength\LTright{0pt}



\begin{document}
{\LARGE {\em 
\noindent Modern Physics---PHYS~220
\vspace{0.5mm}

\noindent Fall 2023
\vspace{3mm}
}}


{\LARGE {\em \noindent Lab: $e/m$ with Teltron Deflection Tube}}

\large{\noindent Josh Diamond,  John Cummings, George Hassel, Mark Rosenberry, \& Matt Bellis}


\vspace{0.5cm}
\noindent{\bf \LARGE Week 1}\\
\vspace{0.5cm}

``Prelab" to be completed in class.

\begin{enumerate}

\item What is the mass of an electron? What is the charge on an electron? (you can look these values up)

\item What is the force on a charge due to an electric field?

\item What is the force on a charge due to a magnetic field?  What other factors does it depend on (other than the charge and the field)?

\item What is the change in potential energy of a charge that moves from one electric potential to another?  What determines if this is an increase or decrease in potential energy?

\item \label{q-velocity} If an electron ``falls'' from rest through a potential to accelerate it, what would its velocity squared be at the end? Derive a relationship, then calculate the velocity an electron would have after
being accelerated through a 1 volt potential. 

\item \label{q-shape} If a moving electron (say the electron from question~\ref{q-velocity}!) enters a place where there is a uniform magnetic field, what would the shape of its path be?

\item \label{q-newton} Using Newton's second law and the acceleration due to circular motion (spoiler for question~\ref{q-shape}!), derive the relationship between the electron charge, the electron mass, the velocity of the electron, the strength of the magnetic field, and the radius of the path of the electron.  
% I used the notation $e$, $m_e$, $v$, $B$, and $r$, respectively, for these quantities, but you are free to use any notation you like provided you use it consistently in this lab, it is understandable, and easily remembered.  

%\item Derive an expression you can use to find the charge-to-mass ratio of the electron: combine the velocity squared from question~\ref{q-velocity} and the relationship from question~\ref{q-newton} to find an expression for $e/{m_e}$ in terms of the accelerating potential, the magnetic field strength, and the radius of the electrons path ($V_a$, $B$, and $r$ in my notation).  

%\item Use the manual for the Teltron electron deflection tube U19155 [1000651] to draw a wiring diagram to use the tube in magnetic deflection mode.  

%\item Experimentally, we will use Helmholtz coils to produce a fairly uniform magnetic field for our electron.  What is special about the geometry of Helmholtz coils?  Find (meaning lookup) the relationship that gives the magnetic field in terms of the current in the coils, properties of the coils such as radius and number of turns of wire, and fundamental constants.

%\item Estimate the specs (voltage and current capabilities) of the power supply(s) you will need.

\end{enumerate}

\section{Design of the experiment}

Look at the equipment available for this experiment. Use the manual for the Teltron electron deflection tube 
U19155 [1000651].

Figure out how you can measure the ratio of the elementary charge e, to the electron mass, m, using the available apparatus.  Write up a couple of paragraphs explaining what data you will take and how you will turn it into a result.

{\it Submit all this in Canvas.}

One you have submitted that, please read the full instructions we have provided on how to make your measurements.  COMMENT in Canvas on the agreement or difference from your own plan.





\end{document}

 

\section{Introduction}

The estimation of uncertainties\sidenote{Note that \textit{uncertainties} are
  not \textit{errors}. Errors are aspects of your experiment that you should
  strive to identify and eliminate, whereas uncertainties (particularly random
  uncertainties) are an intrinsic and unavoidable feature of any and all
  measurements.} is somewhat of an artform rather than a science. 
Many assumptions are often made to give some rigor to this art; unfortunately we
often forget the assumption when applying the procedures derived.  Don't forget
to use your common sense.

Please {\em don't} use percent deviation.  
\begin{equation}
\% {\rm deviation} ~=~ ({{ \rm measured} - {\rm accepted }) \over {\rm accepted }} \times 100
\end{equation}
Although this can be interesting in rare cases, it is almost never used
appropriately.  The much more interesting value is the uncertainty of the
measured value: in other words, if we repeated the measurement how different is
it likely to be between measurements? 

\section{Uncertainties}

\subsection{Device Resolution}

One common source of uncertainty in the lab is due to the resolution of the
measuring device used.\sidenote{For example a scale will have a smallest division,
whether it is a ruler, an analog electrical meter, a clock scale, {\em etc.}}
For a reasonably graduated scale the uncertainty is usually about equal to the
smallest division.  Imagine the possible readings a reasonable person might make
for a given measurement.  Vernier scales require a little more thought; the
resolution might depend on the quality and width of the division marks.

If the device is digital, it will often have a specified resolution.  The trick
can be to locate it.  Sometimes it will be on a plate or sticker on the device
itself.  Check the back or bottom of the instrument if it is not apparent.  You
might have to refer to the manual which may be in the lab.  If you cannot find
the manual in the lab, you may be able to locate it online.  In the absence of
any manufacturer specification, you probably will be close if you use $1/2$ the
last digit (and common sense).

\subsection{Random or Statistical Uncertainties}

Some measurements will be dominated by other random contributions.  Take for
example measuring the period of a pendulum with a stopwatch.  Repeated
measurements will lead to slightly different results due to your reaction time,
{\em etc.}  Here we make some assumptions referred to earlier, such as the
distribution of results is random and normal (or Gaussian).  In other words, if
we repeated the measurement many times and produced a {\em histogram} of the
results, we assume we would get a Gaussian distribution.
%\sidenote{Be sure you understand all the features of the Gaussian curve labeled
%in Figure~\ref{fig:gaussian}.} 

\begin{figure}
\includegraphics[width=4in]{../images/gaussian.jpg}
\caption{A Gaussian distribution. Also known as a normal distribution or a bell
  curve.}
\label{fig:gaussian}
\end{figure}

Under these assumptions, statisticians have shown that the best estimate of the
true value is given by the mean $\mu$ which we all know is

\begin{equation}
\mu = { 1 \over N } \sum_i^N x_i
\end{equation}

\noindent where $x_i$ are the sum is over all $N$ measurements.  Great, this
gives us the best guess of the value, but what about it's uncertainty?  The {\em
  width} of the distribution of results is indicative of its uncertainty.  We
can measure the width by looking at the average deviation of the results from
the mean.  Notice that the signed deviation of a symmetric distribution from the
mean is always zero (not very useful) so we must somehow get rid of the sign.
We take the square of the difference $x_i - \mu$ and average
\begin{equation}
\sigma^2 = {1\over N} \sum_i^N \left( x_i - \mu \right)^2
\end{equation}
Sometimes you'll see the $N$ replaced with $N-1$ to give you something called
``an unbiased estimator.''  Either is fine for this class, and the difference
for large $N$ is negligible anyway. 

This technique of repeated measurements is useful if the other techniques above
don't apply.  You could also use it to {\em measure} the resolution of a device
or measurement technique for use during future measurements. 

\subsection{Systematic Uncertainties}

Systematic uncertainties are much harder to quantify in a general fashion.
Since they can systematically move the measurement away from the correct value,
averaging will not neccessarily remove these no matter how many measurements you
make.  This property can often serve as a useful test to classify an uncertainty
as statistical or systematic: would repeated measurements tend to decrease the
size of the uncertainty?  Systematic uncertainties can come from the measurement
devices used or from assumptions and approximation used in the analysis of your
data.  Typical sources in this lab might be things like an uncalibrated
instrument, or the assumption that a wire is zero resistance. 

\subsection{Precision versus Accuracy}

Finally, it is important to understand the difference between \textit{precision}
and \textit{accuracy}. Precision characterizes the level of uncertainty in your
measurements, whereas accuracy characterizes how close your measurements are to
the ``true'' value.\sidenote{Of course, you may not know the ``true'' value,
  depending on the nature of your experiment.} As illustrated in
Figure~\ref{fig:precacc}, your measurements can be precise and accurate (good!),
precise and inaccurate (a possible sign of one or more systematic
uncertainties), imprecise and accurate (OK but maybe build a better instrument),
or even imprecise and inaccurate (start over!).

\begin{figure}
\includegraphics[width=3in]{../images/precision-accuracy.png}
\caption{Graphic illustrating the difference between precision and accuracy.}
\label{fig:precacc}
\end{figure}

\section{Propagation of Uncertainties}

When measured values are used to calculate quantities it is important to
propagate the uncertainty on the measured quantity through to the calculated
value.  For a simple function of one measured variable it is easy to see
(Figure~\ref{fig:error-prop})
that the trick is to multiply the measured uncertainty by the derivative of the
function with respect to the measured variable: 
\begin{equation}
\sigma_f = {df \over dx} \sigma_x
\end{equation}
\noindent When the calculated value is a function of more than one measured
quantities, the total uncertainty is found by adding the individual
uncertainties in quadrature (or adding the squares of the uncertainties)
\begin{equation}
\sigma_f^2 = \sum_i ({df \over dx} \sigma_i)^2
\label{eq:error-prop}
\end{equation}

For example, suppose you have measured the mass $m$ of an object and its
acceleration $a$ and you wish to calculate the force $F$ from Newton's second
law,
\begin{equation}
F = ma.
\label{eq:newton}
\end{equation}

\begin{figure}
\includegraphics[width=3in]{../images/error-prop.png}
\caption{Propagating uncertainties.}
\label{fig:error-prop}
\end{figure}
You measured the mass to be $m = 550.10~ {\rm g}$ and the scale has a resolution of $\sigma_m = 0.02 ~{\rm g}$.  The acceleration you measured was $a = 43 ~{\rm cm}/{\rm s}^2$ and the accelerometer has a resolution of $\sigma_a = 5 ~{\rm cm}/{\rm s}^2$.  You would record these values as $m = 550.10 \pm 0.02 ~{\rm g}$ and $a = 43 \pm 5 ~{\rm cm}/{\rm s}^2$.

To calculate the central value of $F$, simply use the central value for the
measured quantities in Eq.~\ref{eq:newton}: 
\begin{equation}
F = 550.10 ~{\rm g} \times 43~ {\rm cm}/{\rm s}^2 = 23654 ~{\rm dynes}
\end{equation}

To calculate the uncertainty, use Eq.~\ref{eq:error-prop}.
\begin{eqnarray*}
\sigma_F^2 &=& \left( {dF \over dm} \sigma_m \right)^2 + \left( {dF \over da} \sigma_a \right)^2 \\
&=& \left( a \sigma_m \right)^2 + \left( m \sigma_a \right)^2 \\
&=& \left( 43 ~{\rm cm}/{\rm s}^2\times  0.02 ~{\rm g} \right)^2 + \left( 550.1 ~{\rm g}\times  5~ {\rm cm}/{\rm s}^2\right)^2 \\
&=& 7.56\times 10^6 ~{\rm dynes^2}
\end{eqnarray*}
so $\sigma_F = 2750 ~{\rm dynes}$, and your final answer should be written
\begin{equation}
F = 24000 \pm 3000 ~{\rm dynes}.
\end{equation}
where we have rounded the uncertainty to one significant figure.\sidenote{If we
  are already uncertain in the thousands place, it is silly to be reporting
  hundreds and smaller places!} The central value should then be written no more
accurately that this number.  The general rule is to round the uncertainty to one
(or sometimes two, for an intermediate result) significant digits and then use
that as the lowest decimal place of the quoted value.

\subsection{Common cases}
There are several common cases you may wish to remember to save yourself some time:
\begin{enumerate}
\item Scalar multiplication, $f(x) = cx$
\[
\sigma_f = c\sigma_x;
\]
\item addition, $f(x,y) = x + y$
\[
\sigma_f^2 = \sigma_x^2 + \sigma_y^2;
\]
\item a product, $f(x,y) = xy$, or division, $f(x,y) = x/y$
\[
\left( \frac{\sigma_f}{\bar{f}} \right)^2 = \left( \frac{\sigma_x}{\bar{x}} \right)^2 + \left( \frac{\sigma_y}{\bar{y}} \right)^2
\]
(combine the {\em fractional uncertainties} as you would for addition). 
\end{enumerate}


\section{Significant Digits}

If you have no knowledge of the resolution of your instrument\sidenote{For
  example, in a practice problem from a book.} pay attention to significant
digits!  Please don't write down every digit your calculator gives you (unless
it is justified).  Ignoring this rule can lead to false conclusions.

Consider the case of Olympic gymnastics: judges give scores to the tenth of a
point and the scores are averaged.  Yet medals are given out based on results to
the thousandths place.  Can you come up with a scenario where the wrong medals
are given out?

The number of significant digits is usually the digits in a number after
dropping any leading zeros.  This rule is not always unambiguous, however; for
instance, trailing zeros when there is no decimal point as in 310 can be
troublesome.  Again, use common sense and be careful when you record values.  A
nice convention is to write the values of physical measurements so that the last
measured digit falls to the right of the decimal point.  You can do this by
either using scientific notation or simply choosing a larger unit of
measurement.

Finally, with most arithmetic operations you should keep in your answer the {\em
  fewest} number of significant digits you have in your data.  The only
exception is when adding (or subtracting) numbers when you use the fewest number
of {\em decimal places} in your data.\sidenote{For example, consider $103.25 -
  0.0001$.}

\section{Exercises}

\begin{enumerate}
\item Measure the width and length of the whiteboard, calculate the area and the
  uncertainty.
  
\item Measure and calculate the density of some metal shapes.  Compare your
  answer with others in the class and try to identify which are made of the same
  material. 

\end{enumerate}

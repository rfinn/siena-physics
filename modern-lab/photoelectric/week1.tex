\documentclass{tufte-handout}
\pagenumbering{arabic}
\usepackage{amsmath}
\usepackage[pdftex]{graphicx}
\usepackage{color}
\usepackage{enumitem}
\usepackage{natbib}
\usepackage{comment}
\usepackage{bm}
\usepackage{mdwlist}

\usepackage{hyperref}
\hypersetup{
    colorlinks=true,       % false: boxed links; true: colored links
    linkcolor=red,       % color of internal links
    citecolor=red,        % color of links to bibliography
    filecolor=magenta,      % color of file links
    urlcolor=blue           % color of external links
}

\newcommand{\cs}{${}^{137}_{55}{\rm Cs}$ }
\newcommand{\ba}{${}^{137}_{56}{\rm Ba }$}
\newcommand{\bam}{${}^{137}_{56}{\rm Ba^* }$}

%\pagestyle{myheadings}
%\markright{Fall~2016\hspace{1.9in}Mechanics Lab}


%\documentclass[12pt]{aastex}
%\usepackage{graphicx,mdwlist,longtable,url}
%\usepackage[final]{pdfpages}
%
%\usepackage[title,titletoc,toc]{appendix}
%\usepackage{longtable}
%
%% left-justify the section headings and make them bigger
%\usepackage{titlesec}
%\newcommand*{\justifyheading}{\raggedright}
%\titleformat*{\section}{\Large\bfseries}
%\titleformat*{\subsection}{\large\bfseries}
%
%\renewcommand{\bottomfraction}{0.9}
%
%\setlength{\parindent}{0.cm}
%\setlength{\parskip}{0.3cm}
%\setlength{\topmargin}{-1.3cm}
%\setlength{\textheight}{9in}
%\setlength{\evensidemargin}{0cm}
%\setlength{\textwidth}{6.46in}
%\setlength{\oddsidemargin}{0in}
%
%\pagestyle{myheadings}
%\markright{Fall~2016\hspace{1.9in}Mechanics Lab}
%
%% these make the longtables span the width of \textwidth
%\setlength\LTleft{0pt}
%\setlength\LTright{0pt}



\begin{document}
{\LARGE {\em 
\noindent Modern Physics---PHYS~220
\vspace{0.5mm}

\noindent Fall 2021
\vspace{3mm}
}}


{\LARGE {\em \noindent Lab: Photoelectric Effect}}

\large{\noindent Josh Diamond, John Cummings, George Hassel, Mark Rosenberry, \& Matt Bellis}


\vspace{0.5cm}
\noindent{\bf \LARGE Week 1 - Prelab}\\
\vspace{0.5cm}

\section{Objective}

Make measurements to determine Planck’s constant, $h$, and how the stopping voltage
depends on light intensity. See if your results agree with Einstein’s work.

%Hand in your responses to these questions at the end of lab today.  Please email me either an electronic file or a legible photo/scan of a handwritten page(s).  Each person should hand in their own.

\section{Prelab questions}

\begin{enumerate}

\item What is the mass of an electron? What is the charge on an electron? (you can look these values up, but cite them)

\item What is the force on a charge due to an electric field?

\item What is the change in potential energy of a charge that moves from one electric potential to another?  What determines if this is an increase or decrease in potential energy?

\item \label{q-velocity} If an electron ``falls'' from rest through a potential to accelerate it, what would its velocity squared be at the end? Derive a relationship, then calculate the velocity an electron would have after
being accelerated through a 1 volt potential.


\item \label{q-photocell} Phillipp Lenard did a series of experiments around 1900 that showed shining a light on a material (particularly metals) can result in electrons being ejected from the surface, a phenomenon known as the photoelectric effect.  These ejected electrons can be collected for measurements by placing a collector electrode at an appropriate potential near the emitter.  What potential relative to the emitter electrode will attract the electrons?  What potential will repel them?  If the potential repels the electrons, is it possible for {\em any} electrons to reach the collector?

\item Use the simulation at http://phet.colorado.edu/en/simulations/photoelectric to experiment with a simple model of a photocell, described in question~\ref{q-photocell}.  Look particularly at:
\begin{itemize}
	\item How does the current (\# of electrons) depend on the frequency (or wavelength) of the light?  Sketch a plot to show the behavior.  Notice that there is a photon frequency below which no electrons are ejected.  This is called the cutoff frequency.
	\item How does the current depend on the potential of the collector?  Sketch a plot to show the behavior.  What does the voltage that {\em just} stops the electrons from reaching the collector tell you?  This is called the stopping potential.
	\item How does the current depend on the brightness or intensity of the light? Again sketch a plot.  {\em Note:} stay well above the cutoff frequency when investigating the intensity dependence.  There is a strange effect in the simulation which I think is a glitch, but I'm not sure.
	\item What properties of the photoelectric effect depend on the material?
\end{itemize}

\item What is Einstein's equation for the energy of a photon?  From whose work did he get this idea from?

\item \label{q-energycons} Assuming electrons in a metal are bound by {\em at least} a binding energy $W$ (which is a property of the particular material), use conservation of energy to write down the maximum kinetic energy of the ejected electrons.  Explain each term.

%\item  \label{q-pe-equation} How can the collector potential be used to measure the kinetic energy of the electron?  Rewrite the equation you derived in question~\ref{q-energycons} to be in terms of the potential of the plate, $V$, rather than the kinetic energy of the electron.

%\item Using the equation you derived in question~\label{q-pe-equation}, devise a plan that uses multiple measurements of frequency and the corresponding stopping potential, $V_0$, to find Planck's constant, $h$.

\end{enumerate}


\section{Experimental design}
Refer to the {\bf Equipment Instructions} for more information.

Then work to answer the following questions experimentally:

\begin{enumerate}
    \item How will you extend the questions thus far to make a measurement of Planck’s constant?

    \item Using data and your experience with the Simulation, can you determine how Photocurrent depends
on the light intensity? How about how Stopping potential depends on the light intensity?
\end{enumerate}

Submit your responses to Canvas today.  Each person should hand in their own separate work.

After submitting the prelab, read the full instructions. Compare and contrast with your procedure.
Submit your comments to Canvas

\end{document}

 

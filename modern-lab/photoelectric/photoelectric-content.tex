\section{Theory}


A full treatment is given in Tipler and Llewellyn, pp. 103-11.  The following explanation has been paraphrased from the equipment manuals referenced at the end of this document.

An important question addressed in this course involves the description of light.   At a macroscopic scale, light can be shown to behave as a wave.  That is, it is possible to quantitatively discuss the speed, $c$, wavelength, $\lambda$, and frequency, $\nu$, and to demonstrate the wave-like behaviors of reflection, refraction, interference and diffraction.  However, at the atomic level, it becomes necessary to consider light as a discrete particle, or packet of energy known as a photon.    

According to Planck's law of radiation, the emission and absorption of radiation is associated with transitions or jumps between two energy levels within an oscillating system.   The amount of radiant energy emitted or absorbed by such a system is quantized, with a specific value of 
\begin{equation} 
E = h \nu
\end{equation}
where $E$ equals the energy of the photon, $\nu$ is the frequency of the radiation, and $h$ is a fundamental constant that became known as Planck's constant.  This represents an important departure from the analogy with mechanical waves as the energy of light depends on the color, not the amplitude of the wave.   

The photoelectric effect provides a convincing demonstration of this principle.   When light strikes a metal target, electrons can be emitted from the surface and detected as a measurable photocurrent.    Einstein postulated that the energy of an incident photon is transferred to a surface electron.  Some of the energy is needed to liberate the electron from the metal.   This amount of energy is known as the work function, $W$, and depends on the type of metal.   The rest of the energy becomes the kinetic energy of the electron, up to a maximum value, $KE_{\rm max}$.    

This maximum kinetic energy is measured using a phototube, which is simply two electrodes enclosed in a vacuum tube.   Light strikes the larger target cathode, and the photocurrent is detected at the anode.  It is possible to apply a potential difference, $V$, and establish an electric field in the opposite direction of the current to slow the emitted electrons.   At a specific value known as the stopping potential, $V_0$, the photocurrent is reduced to zero, meaning that all electrons including those with the most  kinetic energy $KE_{\rm max}$, are completely deflected.   This also means that the energy of those electrons can be expressed as $qV_0$.   Einstein's equation describing the photoelectric effect is thus:     
\begin{equation} 
E = h \nu = KE_{\rm max} +W=qV_0+W.
\label{eq-energy}
\end{equation}
One important implication is that the intensity of light will increase the total number of photons, but it will not increase the energy of the individual photons, or the stopping potential.  This also suggests that for a target of given $W$, there will be a threshold frequency, $\nu_0$, below which no photocurrent can be produced.  

%In photoelectric emission, light strikes a material, causing electrons to be emitted.  The classical wave model predicted that as the intensity of incident light was increased, the amplitude and thus energy of the wave would increase.  This would then cause more energetic photoelectrons to be emitted.   The new quantum model, however, predicted that higher frequency light would produce higher energy photoelectrons, independent of intensity, while the magnitude of the photoelectric current, or number of electrons was dependent on the intensity as predicted by the quantum model.   Einstein applied Planck's theory and explained the photoelectric effect in terms of the quantum model using his famous equation for which he received the Nobel prize in 1921:
%\begin{equation} 
%E = h \nu = KE_{\rm max} +W
%\end{equation}
%where KE is the maximum kinetic energy of the emitted photoelectrons and W is the energy needed to remove them from the surface of the material (the work function).  E is the energy supplied by the quantum of light known as a photon.   
 

%In 1901 Planck published his law of radiation.  In it he stated that an oscillator, or any similar physical system, has a discrete set of possible energy values or levels; energies between those values never occur.  

%Planck went on to state that the emission and absorption of radiation is associated with transitions or jumps between two energy levels.  The energy lost or gained by the oscillator is emitted or absorbed as a quantum of radiant energy, the magnitude of which is expressed by the equation: 
%\begin{equation} 
%E = h \nu
%\end{equation}
%where E equals the radiant energy, $\nu$ is the frequency of the radiation, and h is a fundamental physical constant of nature.  The constant, h, became known as Planck's constant. 



%For a brief summary of the theory, see EI1, pages 1-2 and EI2, page 2.

%From Sargent-Welch:
%A phototube consists of two electrodes enclosed in an evacuated glass tube.  One electrode has a large photosensitive surface and is called the cathode or the emitter.  The other electrode is in the form of a wire and is called the anode or the collector.   In normal operation the anode is held at a positive potential with respect to the cathode.  When the cathode is exposed to light, electrons are ejected from its photosensitive surface.  These electrons are attracted to the positive anode and form a current which can be measured with a galvanometer.

%The kinetic energy of the ejected electrons is determined by the frequency of the light striking the phototube.  The quantity of ejected electrons is dependent on the intensity of the light.  The maximum kinetic energy of the photoelectrons is given by: 
%\begin{equation}
%KE_{\rm max} = hf -W
%\end {equation}
%where h = Planck's constant, f = frequency, and W = Work function.

%Since the work function, which is the energy required to release an electron from the surface of the metal, is a constant, the maximum kinetic energy depends directly on the frequency of the light.   

%If the potential applied to the anode is gradually decreased and made negative, the electrons ejected from the cathode will not have enough energy to reach the anode and will be repelled back to the cathode.  At a certain voltage called the "stopping potential" the electron current from the cathode to the anode will become equal to zero.  At that point the maximum kinetic energy of the electrons is equal to 

%\begin{equation}
%eV= hf -W
%\end {equation}
%where e = electronic charge $1.602\times 10^{-19}$~Coulomb and V = stopping potential.

%By experimentally determining the stopping potential for several values of frequency and using the above equation, Planck's constant can be determined.  


%From PASCO:
%The emission and absorption of light was an early subject for investigation by German physicist Max Planck.  As Planck attempted to formulate a theory to explain the spectral distribution of emitted light, he ran into considerable difficulty.  Classical theory (Rayleigh-Jeans Law) predicted that the amount of light emitted from a black body would increase dramatically as the wavelength decreased, whereas experiment showed that it approached zero.  This discrepancy became known as the ultraviolet catastrophe.   

%Experimental data for the radiation of light by a hot, glowing body showed that the maximum intensity of emitted light also departed dramatically from the classically predicted values (Wien's Law).  In order to reconcile theory with laboratory results, Planck was forced to develop a new model for light called the quantum model.   In this model, light is emitted in small, discrete bundles or quanta.   

%By the late 1800's many physicists thought they had explained all the main principles of the universe and discovered all the natural laws.  But, as scientists continued working, inconsistencies that couldn't easily be explained began showing up in some areas of study.  

%In 1901 Planck published his law of radiation.  In it he stated that an oscillator, or any similar physical system, has a discrete set of possible energy values or levels; energies between those values never occur.  

%Planck went on to state that the emission and absorption of radiation is associated with transitions or jumps between two energy levels.  The energy lost or gained by the oscillator is emitted or absorbed as a quantum of radiant energy, the magnitude of which is expressed by the equation: 
%\begin{equation} 
%E = h \nu
%\end{equation}
%where E equals the radiant energy, $\nu$ is the frequency of the radiation, and h is a fundamental physical constant of nature.  The constant, h, became known as Planck's constant. 

%Planck's constant was found to have significance beyond relating the frequency and energy of light, and became a cornerstone of the quantum mechanical view of the subatomic world.  In 1918, Planck was awarded a Nobel Prize for introducing the quantum theory of light.

%In photoelectric emission, light strikes a material, causing electrons to be emitted.  The classical wave model predicted that as the intensity of incident light was increased, the amplitude and thus energy of the wave would increase.  This would then cause more energetic photoelectrons to be emitted.   The new quantum model, however, predicted that higher frequency light would produce higher energy photoelectrons, independent of intensity, while the magnitude of the photoelectric current, or number of electrons was dependent on the intensity as predicted by the quantum model.   Einstein applied Planck's theory and explained the photoelectric effect in terms of the quantum model using his famous equation for which he received the Nobel prize in 1921:
%\begin{equation} 
%E = h \nu = KE_{\rm max} +W
%\end{equation}
%where KE is the maximum kinetic energy of the emitted photoelectrons and W is the energy needed to remove them from the surface of the material (the work function).  E is the energy supplied by the quantum of light known as a photon.   

%A light photon with energy $h\nu$ is incident upon an electron in the cathode of a vacuum tube.  The electron uses a minimum W of its energy to escape the cathode leaving it with a maximum energy of $KE_{\rm max} $ in the form of kinetic energy.   Normally, the emitted electrons reach the anode of the tube and can be measured as a photoelectric current.  However, by applying a reverse potential V between the anode and the cathode, the photoelectric current can be stopped.  $KE_{\rm max}$ can be determined by measuring the minimum reverse potential needed to stop the photoelectrons and reduce the photoelectric current to zero.   Relating kinetic energy to stopping potential gives the equation:
%\begin{equation} 
%KE_{\rm max} = eV
%\end{equation}
%Therefore using Einstein's equation, 
%\begin{equation} 
% h \nu = eV + W
%\end{equation}
%When solved for V the equation becomes:
%\begin{equation} 
%V=(h/e)\nu - W/e
%\end{equation}
%If we plot V vs $\nu$ for different frequencies of light the graph will look like figure 2.  The V intercept is equal to W/e and the slope is h/e.  Coupling our experimental determination of the ratio h/e with the accepted value for e, $1.602\times 10^{-19}$~C, we can determine Planck's constant, h.
    
The predictions of this theory provide the foundation for an experiment to determine the value of Planck's constant and to verify the photon model of light.  The accepted value of Planck's constant $ h = 6.63 \times 10^{-34} {\rm J}\cdot{\rm s}$ in SI units.  However, at the atomic scale, it can be more convenient to work with electron-Volts (eV).   

\section{Preliminary Questions}

\begin{enumerate}
\item Sodium metal has a work function of 2.28 electron volts (eV).  Note:
$1 {\rm eV} = 1.6 \times 10^{-19} {\rm J}$.  Light of wavelength $\lambda$
is incident on the metal surface.
\begin{enumerate}
\item What is the maximum wavelength of incident light that can produce
a photoelectric effect ({\em i.e.}, that can lead to the emission of
electrons from the sodium metal surface)?
\item If light of wavelength $\lambda = 400 {\rm nm}$ is incident on
sodium, determine the maximum kinetic energy of the emitted electrons.
Also find the stopping potential.
\end{enumerate}
\end{enumerate}

\section{Notes on Equipment}

%The 3B Scientific apparatus for Planck's Constant is useful for examining three aspects of the photoelectric effect: 1.) demonstrating the concept of stopping potential, 2.) determining the value of Planck's constant, and 3.) investigating the relationship of light intensity and stopping voltage.  
%We will employ two different methods in our exploration of the
%photoelectric effect. One method uses the Sargent-Welch apparatus and
%involves measuring the current of photoelectrons in a closed circuit.
% The applied potential required to stop all photoelectrons is found
%from the point where the photocurrent $\to 0$.
% This method provides a good physical picture, although precise
%determination of the stopping potential is difficult.

%The other approach uses the PASCO apparatus, in which the key
%quantity--the stopping potential--is measured directly rather than being
%obtained from the measurements of the photocurrent.

The 3B Scientific Planck's Constant Apparatus uses specific LED's as a light source instead of spectral lines.  It is intended to illustrate the concept of stopping potential, to determine Planck's constant, and to investigate the relationship between $V_0$ and intensity.  The photocell is highly sensitive, so the following precautions are necessary:
\begin{enumerate}
\item The protective cover should never be removed
\item  When the experiment is completed, slide the empty sleeve over the collector tube of the photocell.   
\item Keep the apparatus secure so that it does not get shaken and do not expose it to extreme temperatures, high humidity, moisture or direct sunlight
\end{enumerate}

\section{Procedure}

\subsection{Simulation} The circuit in a photoelectric apparatus is typically a simple one, however, the photocell is usually shielded and not easily seen.   A versatile simulation of the photocell and the circuit used to apply the stopping potential is available at: {\underline{http://phet.colorado.edu/en/simulation/photoelectric}}.   Although it is possible to simulate the entire lab activity with this applet, we will use it to illustrate some basic functions of the photocell.
\begin{enumerate}
\item Open the applet and adjust the different variables.  Specifically, note what happens in the simulation when you: 
\begin{enumerate}
\item Increase the intensity slider to 100\%.
\item Change the wavelength of the light to different values.
\item On a  wavelength setting that emits electrons, change the potential applied by the battery.
\end{enumerate}
\item Investigate the concept of stopping potential: 
\begin{enumerate}
\item Set the intensity to 100\%, the wavelength to a color that emits electrons and the battery to 0.00 V.  Note what happens to the electrons in the simulation. 
\item What happens to the electrons and the value of current when you increase the battery potential in the positive direction?  In the negative direction?
\item Find and record the value of battery potential when the current first registers 0.
\item Does this value of potential change when you change the intensity of the light?  When you change the wavelength of the light?  Which has a bigger effect?
\end{enumerate}        

\item Investigate work function and threshold frequency: 
\begin{enumerate}
\item Reset the intensity to 100\%, the wavelength to a color that emits electrons and the battery to 0.00 V.  
\item Increase the wavelength until the current first registers zero, and then slightly beyond.   Note what happens to the electrons in the simulation.   Find the lowest wavelength that does not emit electrons, and record this value as the threshold wavelength
\item Change the metal target from the menu.   Repeat the previous steps to determine the threshold wavelength for all available targets.   Note that this is easy in the simulation, but it would be considerably more difficult and expensive in an experimental setup. 
\item Determine the threshold frequency and corresponding photon energy for each of the metal targets. 
\end{enumerate}        
\end{enumerate}
  

\subsection{Measurement of Planck's constant.}    
\label{sec-planckconst}
Following Equation~\ref{eq-energy}, we will measure the stopping potential for several frequencies of light.  
\begin{enumerate}
\item Plug in the transformer to supply power to the apparatus
\item Set the intensity of the light source to 75\%
\item Insert the plug for the first light source into the LED connector socket
\item Push together the jaws of the clip for the sleeve over the collector tube of the photocell and remove the tube
\item Push the LED unit fully onto the collector tube of the photocell until the jaws of the clip snap into place
\item Set the fine and coarse adjustment knobs for the voltage to a central position
\item Wait a few minutes, then slowly turn the coarse setting knob until the photoelectric current measured by the nanoammeter is approximately 0.
\item Use the fine setting knob to optimize the calibration.  Turn it around until the display oscillates between 0 and -0.
\item  Record the voltage as the stopping potential  $V_0$
\item Repeat this measurement for the other four LEDs.
\end{enumerate}

\subsection{Measurement of stopping voltage $V_0$.}   
\label{sec-stoppot}
Using a similar set-up as before, we experimentally determine the stopping voltage
\begin{enumerate}
\item Set the fine and coarse adjustment knobs for the voltage to 0.  The nanoammeter's reading of -1.  indicates saturation.   
\item Slowly turn the coarse setting knob until the photoelectric current measured by the nanoammeter provides actual data.   Record the voltage and the photocurrent.
\item Increase the voltage in increments of your choosing.  Record the voltage and photocurrent for a series of these increments.    
\item  Change the intensity to a different value, and repeat these steps.
\item Change to a different LED and obtain similar data sets for two different intensities.
\end{enumerate}

\subsection{Stopping voltage $V_0$ and light intensity}   
\label{sec-stopintensity}
\begin{enumerate}
\item Select an LED, set the light to maximum intensity, and determine the stopping voltage.   
\item Reduce the intensity to zero in a series of steps and determine the stopping voltage in each case.
\item After the experiment, close the plastic cover back over the tube.
\end{enumerate}

%This part of the lab uses the Sargent-Welch Planck's Constant apparatus to measure the photocurrent  as a function of anode voltage.  These data allow the determination of the stopping potential and observation of the effect of light intensity on the photocurrent and the stopping potential.

%\begin{enumerate}
%\item To measure the photocurrent $I$ we use a very sensitive digital
%current meter (Keithley picoammeter) in the position labeled
%``G'' in the circuit of Fig. 1 on page 3 of
%EI1.  As a check, we also include an analog microammeter in series
%with the picoammeter.  Connect the circuit in Fig. 1.
%Have the instructor check the circuit before turning on the power.
% Also, turn on the mercury vapor light source so that it can warm up.

%{\bf CAUTION: THE MERCURY VAPOR LAMP IS ENCLOSED IN A METAL HOUSING WITH A
%SMALL APERTURE.  DO NOT LOOK DIRECTLY INTO THE APERTURE.}


%\item  To prepare the picoammeter, carry out the following initialization
%steps.
%\begin{enumerate}
%\item Temporarily disconnect the picoammeter from the circuit.
% Turn on the picoammeter power.

%\item Press the ZCHK button to perform a zero check. The letters
%``ZC'' should appear in the display.

%\item To optimize accuracy for measuring very small currents,
%we also perform a zero correction. Use the range arrow buttons to
%select the lowest range -- the 2 nA range.  Then press the ZCOR key.
%``ZZ'' should appear in the display.
% Press the ZCOR button again and ZC appears.  Finally, press the ZCHK
%button and the display should be cleared.

%\item Reconnect the picoammeter in the circuit.
%\end{enumerate}

%\item The voltage applied to the phototube is read on the voltmeter.
% The voltage input is a DC power supply.  The power supply voltage is
%adjusted by two knobs on the apparatus, for coarse and fine
%adjustments.  Turn on the power supply and turn both knobs to
%achieve the maximum tube voltage.  This voltmeter reading should be
%between 3 and 4 volts.  If it is not in that range, adjust the power
%supply voltage to bring the reading into that range.

%\item Align the tube with the light from the mercury source.
% The system should be ready to record data.

%\item Insert the $435.8~{\rm m}\mu$ (that is 435.8 {\em milli-microns} or 435.8~nm) filter in the apparatus.  This selects the mercury emission
%spectral line of that wavelength.  Turn the voltage adjustment knobs
%to the smallest voltage setting.  Record the photocurrent as a
%function of voltage, starting with the smallest voltage and increasing
%the voltage in steps of 0.1 volt.  You should find that the current
%decreases, passes through zero, and approaches a small negative and
%approximately constant value.  At this point your data is complete.

%\item Now insert the beamsplitter between the light source and
%the tube.  The beamsplitter is essentially a thin block of glass
%mounted at an angle of about $45^\circ$
%relative to the direction of the light from the source.  The
%beamsplitter reduces the incident light intensity without affecting the
%wavelength.  Repeat the measurements of step 5.

%\end{enumerate}

%\section{Part B} Determining Planck's constant.  
%The second part of this lab uses the {\sc PASCO} $h/e$ apparatus for a precise determination of the stopping potential and it's dependence on light intensity and frequency.

%\begin{enumerate}

%\item Read and set up the {\sc PASCO} apparatus as indicated on pages
%3-6 of EI2 (through step 12).

%NOTE:  With this apparatus, the photocurrent is not measured.  Rather,
%we measure the stopping potential directly as a function of incident
%light frequency.  We also make detailed observations on the effect of
%intensity on the stopping potential.  The wavelengths and
%corresponding frequencies used correspond to the spectral lines of the
%mercury vapor light source, and are separated by passing the light
%through a diffraction grating (see Fig. 10, page 6 of EI2).  The
%wavelengths and frequencies used are tabulated in Fig. 10.  These
%values are taken as given quantities to be used in your calculations.

%Read steps 13-15, pages 5-6 of EI2 giving procedures for measuring the
%stopping potential.  Note that you must discharge the system by
%pressing the PUSH TO ZERO button before each new measurement.

%{\bf CAUTION: THE MERCURY VAPOR LAMP IS ENCLOSED IN A METAL HOUSING WITH A
%SMALL APERTURE.  DO NOT LOOK DIRECTLY INTO THE APERTURE.}


%\item Perform Experiment 1, Part A on pages 7-8 of EI2.
%
%NOTE: In this experiment, omit the observations of the time
%required to return to the recorded voltage. (The measurements you make
%with the Sargent-Welch apparatus provide the same information more
%directly.)

%\item Perform Experiment 2 on page 11-12 of EI2.

%\end{enumerate}

\section{Analysis}

\begin{enumerate}
\item Simulation: Make a table of the threshold frequency, $\nu_0$, and work function, $W$, for each of the metal targets.   Compare with known values, and identify the unknown metal target.  

\item Determining Planck's constant:
\begin{enumerate}
\item Determine the frequencies, $\nu$,  of the LEDs
\item Plot the stopping potential $V_0$ as a
function of frequency $\nu$ for the data from Section~\ref{sec-planckconst}.  
%You should have 5 data points plotted,corresponding to the 5 mercury spectral lines used.  
Assuming Einstein's equation is applicable, perform a
straight-line fit to the data %({\em e.g.} using Matlab or Excel) 
and determine the slope and $y$-intercept of the line.  From the slope, calculate your
experimental value of Planck's constant $h$.  
%Use SI units in the calculation.
%Compare your value of $h$ to the accepted value of Planck's constant.


%\item Estimate the uncertainty in the slope and $y$-intercept by
%performing the linear fit of the
%$V_0$ vs $\nu$ data using the LINEST function in Excel.
%Include the LINEST output in your report.

%NOTE: Instructions on using LINEST are given on page 4 of the handout on
%experimental error.
\item Estimate the uncertainty in the slope and $y$-intercept.
%\item From the slope and its uncertainty, calculate an experimental value of Planck's constant $h$ and  the uncertainty in $h$.  
Express your result as: experimental value $\pm$ uncertainty.  Compare your result for $h$ to the accepted value, taking the uncertainty and the units (SI or eV) into account.

\item From the $y$-intercept of the $V_0$ vs $\nu$  graph, determine the work function of the photosensitive material (including its uncertainty).  This comparison is best expressed in units of eV. Compare your value of $W$ to data on the work function for metals.% as given in your textbook.

\end{enumerate}


\item Measurement of stopping voltage: Plot the photocurrent $I$ as a function of applied voltage $V$ for the data obtained in Section~\ref{sec-stoppot}.   Plot the high and low light intensity data for both LEDs on the same set of axes.  For each graph, determine the stopping potential.

%\item On the same set of axes, plot a graph of the high light
%intensity data and of the low light intensity data from Procedure steps
%A5 \& A6, using the 435.8 nm filter.  For these graphs, plot the
%photocurrent $I$ as a function of applied voltage $V$.  For each graph,
%determine the stopping potential.

%NOTE: See EI1, pp. 3-4 (especially Fig. 2) for a
%discussion of the behavior of the data and the way it is reflected in
%the $I$ vs $V$ graph, as well as a method for obtaining the stopping
%potential.


\item  Stopping voltage and light intensity:
\begin{enumerate}
\item Based on your data and previous graph of the measurement of stopping voltage, is
there a relation between the light intensity and the photocurrent?  If
so, describe the relation qualitatively.  
\item Examine the data taken in Section~\ref{sec-stopintensity}.  Is there a relation between
the light intensity and the stopping potential?  If so, describe the
relation qualitatively.
\end{enumerate}


%\item  Based on your data and graphs in Analysis Step 1 above, is
%there a relation between the light intensity and the photocurrent?  If
%so, describe the relation qualitatively.  Is there a relation between
%the light intensity and the stopping potential?  If so, describe the
%relation qualitatively.

\item Describe qualitatively the predictions of Einstein's
photon theory of the photoelectric effect concerning:

\begin{enumerate}

\item the relative size of the photocurrent for the two light
intensities in the data of Section~\ref{sec-stoppot}.

\item the stopping potentials for the two light intensities in
the data of Sections~\ref{sec-stoppot}~and~\ref{sec-stopintensity}.

\end{enumerate}

\item Overall, do your experimental results support Einstein's
theory?  Explain.


%\item In Procedure step B2, the intensity of light of a given
%wavelength incident on the phototube was varied over a wide range, and
%the stopping potential was measured.  Did the wide variation of the
%light intensity have a {\em significant} effect on the
%stopping potential, as measured with this apparatus?  Discuss in terms
%of the predictions of Einstein's theory of the
%photoelectric effect.
%
%
%Was there a {\em small} effect on the stopping
%potential as measured with this apparatus?  Can you account for this
%small effect? HINT:  See the NOTE in the box on page 9 of EI2, just
%above the data table.

%\item Determination of Planck's constant.
%\begin{enumerate}

%\item Plot the stopping potential $V_0$ as a function of frequency $\nu$ for the data of Procedure step B2.  You should have 5 data points plotted, corresponding to the 5 mercury spectral lines used.  Assuming Einstein's equation is applicable, perform a straight-line fit to the data ({\em e.g.} using Matlab or Excel) and determine theslope and $y$-intercept of the line.  From the slope, calculate your experimental value of Planck's constant $h$.  Use SI units in the calculation. Compare your value of $h$ to the accepted value of Planck's constant.

%\item Estimate the uncertainty in the slope and $y$-intercept.

%\item Estimate the uncertainty in the slope and $y$-intercept by
%performing the linear fit of the
%$V_0$ vs $\nu$ data using the LINEST function in Excel.
%Include the LINEST output in your report.

%NOTE: Instructions on using LINEST are given on page 4 of the handout on
%experimental error.

%\item From the slope and its uncertainty, calculate an
%experimental value of Planck's constant
%$h$ and  the uncertainty in
%$h$.  Express your result as: experimental
%value $\pm$ uncertainty.  Compare your result for
%$h$ to the accepted value, taking the
%uncertainty into account.

%\item From the $y$-intercept of the
%$V_0$ vs $\nu$  graph, determine the work function of the
%photosensitive material (including its uncertainty).  Express your
%result in units of electron volts.

%Compare your value of $W$ to data on
%the work function for metals as given in your textbook.

%\end{enumerate}

\end{enumerate}

\section{References}

\begin{itemize}
\item Equipment Instructions (abbreviated EI1) for Sargent-Welch Planck's Constant Apparatus
\item Equipment Instructions (abbreviated EI2) for PASCO $h/e$ Apparatus and Accessory Kit
\item Equipment Instructions (abbreviated EI3) for 3B Scientific Planck's Constant Apparatus (U10700-115)
\item Thornton and Rex, Modern Physics $3^{\rm rd}$ edition, pp. 103-111
\item Tipler and Llewellyn, Modern Physics $5^{\rm th}$ edition, pp. 127-132
\end{itemize}

%\maketitle

\section{Theory}

See the equipment instructions (EI), pp. 1-2, 9.

\section{Preliminary Questions}

\begin{enumerate}
\item Consider an electron placed in between two oppositely charged
horizontal parallel plates.  Assume that the plates are spaced a
distance 1.0 cm apart and that the electric field between the plates is
uniform and perpendicular to the plates.  Also assume that the
direction of the electric force on the electron is opposite to the
force of gravity on the electron ({\em i.e.}, the electron's
weight).
	\begin{enumerate}
	\item Draw a vector diagram showing the forces acting on the electron.  Also
show on the diagram the electric field lines due to the parallel plates.
	\item Calculate the magnitude of the electric field such that the electron is
held at equilibrium.
	\item Determine the potential difference (voltage) between the plates needed
to produce the electric field found in the last part.
	\end{enumerate}

\item Repeat problem 1 for an electron placed on a spherical
water droplet of radius 0.001 mm and density of $1.0 {\rm g}/{\rm cm}^3 = 1000 {\rm kg}/{\rm m}^3$.

\item One reason for using charged liquid droplets (actually oil instead of
water) in this experiment is 
that they can be seen under suitable magnification and illumination,
while electrons are not directly visible.  There is another reason,
which emerges from the answers to part c in the two cases. Explain and
discuss briefly.

\item This problem provides practice in calculating the charge on
an oil drop, using sample data values similar to what might be obtained
experimentally for a particular drop.
\end{enumerate}


Sample data
\begin{itemize*}
\item Distance between reticule lines used in drop measurements: $0.5$~mm;
\item Average time for drop to fall the distance between reticule lines: $t_f =
  17.34$~s; 
\item Average time for drop to rise the distance between reticule lines: $t_r =
  3.79$~s; 
\item Voltage across plates: $V = 386$~volts;
\item Separation between plates: $d = 0.767$~cm;
\item Density of oil: $886~{\rm kg}/{\rm m}^3$;
\item Air pressure: $p=1.0$~atm;
\item Temperature inside space between plates: $T= 25^\circ$.
\end{itemize*}

Use the graph on page 19 of EI to determine the viscosity of air $\eta$ at
the given temperature.  Note that the numerical value of
$\eta$ given on the vertical axis of the graph is
to be multiplied by $10^{-5}$.  For
the value of the constant quantity $b$, refer to the list at the bottom
of the left column on page 9 of EI.

Refer to {\em Suggested Procedure for Computation of
the Charge of an Electron} on page 9 of EI.

Using the sample data, follow steps 1, 2 and 3 and compute the radius $a$
and the mass $m$ of the oil drop, and finally the charge $q$ on the drop

\marginnote{{\bf Important}: All quantities used in these
calculations must be in SI units. Many of the data values are given in
other units and must be converted into SI units before proceeding with
the calculations.}

\subsection{Answers sample calculation}

With this data, you should find that:
\begin{itemize}
\item $a = 4.9 \times 10^{-7} {\rm m}$

\item $m = 4.4 \times 10^{-16} {\rm kg}$

\item $q \approx 3e$, where $e$ is the magnitude of
the charge on the electron.
\end{itemize}


\section{Equipment}
See EI, pp. 3-4.  Be sure to read this section carefully to
familiarize yourself with the apparatus.


\section{Procedure}

\begin{enumerate}
\item The experimental set up and measurement procedures are described
in detail in EI, pp. 5-9. Be sure to read this section carefully to
familiarize yourself with the various steps involved.


\item Your actual data on the oil drops will consist of rise times and
fall times recorded for each drop you are able to see and move in the
field of view.  Record as many times as possible for each drop. The
specific steps involved in recording data on the drops are given in EI,
pp. 8-9.  Refer also to the Notes section below.

\item Before you attempt any measurements on oil drops, the droplet
viewing chamber must be carefully disassembled, cleaned, and
reassembled.

\marginnote{{\bf Important}: Turn off the high voltage supply before disassembling the
chamber.\ The digital voltmeter reading should be zero.}

See Fig. 5 showing the various parts of the droplet viewing chamber,
EI, p. 4.  See also the cleaning instructions, EI, p. 16, top. While
the chamber is disassembled, measure the thickness d of the plexiglass
spacer that fits between the upper and lower metal plates. Use a
Vernier caliper.  Note that d is the separation between the upper and
lower metal plates which (when charged by the high voltage supply)
produce the electric field that acts on the charged oil drops.


\item During the experiment, after oil is repeatedly sprayed into the
apparatus to obtain data, the viewing chamber will eventually become
clogged with oil. Then you must disassemble, clean, and reassemble the
apparatus as described above before attempting the further measurements
on oil drops.
\end{enumerate}

\subsection{Notes}
\label{secnotes}

\begin{itemize}
\item Try to get at least some measurements on oil drops which fall
relatively slowly and which also rise relatively slowly, compared to
other drops.  Such drops will tend to have relatively small mass and
charge, and they are more likely to provide evidence for a fundamental
unit of charge.


\item If you have a drop on which you have obtained several consistent
values for the rise time and fall time, try moving the ionization lever
to the ``on'' position to see if you can
change the charge on the drop.  If you succeed in this, then continue
to measure the rise time (which should now be noticeably different from
what was found before) and the fall time (which should be approximately
the same as before) until you accumulate several more values of each.\sidenote{These procedures are a restatement of the instructions given in steps 5-8 in EI at the top of p. 9).}

\item Record the potential difference V between the metal plates and the
distance between the reticule lines of the telescope viewing screen.
\end{itemize}

\section{Analysis}

\begin{enumerate}
\item Determine the rise and fall velocities
$v_r$ and $v_f$ from the average values of the
rise and fall times and the reticule lines spacing.\marginnote{Use SI units for all quantities.  Refer to Preliminary Question 4 for a
sample calculation.} Using the equations
in the {\em Suggested Procedure for Computation of
the Charge of an Electron} on page 9 of EI,
compute the radius $a$ and the mass $m$ of each drop, and the charge $q$ on
the drop. 

\marginnote{If there is an abrupt jump in
$v_r$ values for a given drop while
$v_f$ remains unchanged, it is likely
that the number of charges on the drop has changed.  The data
following this change must be analyzed separately from the previous
data.  Such a change in charge might have been produced deliberately
by moving the ionization lever to the on position as described in the
second of the three notes on measurements above.}

\item Plot your results for the drop charge as points along a
straight line axis calibrated in units of
$10^{-19}$ Coulomb, with the axis
calibrated from 0 to 20 (or use a larger maximum value than 20 if
needed to accommodate your data).  What sort of behavior do you
observe for your charge values?

\item Assuming that electric charge is observed in units of $e =
1.60 \times 10^{-19}$ C, compute the
number of charge units present on each of your drops.  Do your results
for the charge on a drop come ouput close to integral multiples of $e$?

\item Make another (hypothetical) choice for a value of $e$ which is
consistent with your data.  Recalculate the number of charge units
present on each of your drops.

\item Based only on your data, is there justification for preferring the
actual value of $e$ to your hypothetical value?  Can you suggest a quantitative method of deciding
between alternative values of $e$?  Comment on the amount and precision
of the data needed to establish Millikan's
conclusions.
\end{enumerate}

\section{Uncertainty Analysis}

Although the expression for $q$ is complex, it depends mostly on constants. We
can reduce the expression to one that depends only on rise and fall time:
\begin{equation}
q = \frac{4\pi}{3} \frac{\rho g d}{V} \left[ \left(\frac{9\eta}{2\rho g}\right)
  \left(\frac{1}{1 + \frac{b}{p a}}\right) \right]^{3/2} \frac{D^{3/2}}{\sqrt{t_f}}
\left(\frac{1}{t_f} + \frac{1}{t_r} \right)
\end{equation}

Redefining the constant term and separating the time-dependent term, it can be
shown that:
\begin{equation}
\delta q = Z \sqrt{ \frac{1}{4t_f^3} \left( \frac{3}{t_f} +
  \frac{1}{t_r}\right)^2 (\delta t_f)^2 + \frac{1}{t_f t_r^4} (\delta t_r)^2 },
\end{equation}
where
\begin{itemize*}
\item $\rho$ = density = 886 kg/m$^3$
\item $g$ = acceleration due to gravity in m/s$^2$
\item $d$ = plate separation in m
\item $V$ = measured plate voltage in V
\item $\eta$ = viscosity of air in Ns/m$^2$ determined from graph and
  temperature 
\item $b$ = pressure constant = $8.20\times10^{-3}$~Pa~m
\item $p$ = atmospheric pressure in Pa
\item $a$ = droplet radius in m
\item $t_f$ = measured time to fall, $\delta t_f$ = uncertainty in fall time
\item $t_r$ = measured time to rise, $\delta t_r$ = uncertainty in rise time
\item $D$ = reticule line separation (0.5 mm for major lines, may be multiple)
\end{itemize*}

\section{References}

\begin{itemize}
\item Equipment Instructions (abbreviated EI) for the PASCO Millikan Oil Drop Apparatus
\item Thornton and Rex, Modern Physics $3^{\rm rd}$ edition, pp. 171-173
\item Tipler and Llewellyn, Modern Physics $5^{\rm th}$ edition, pp. 118-119
\end{itemize}


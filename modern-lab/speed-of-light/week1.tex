\documentclass{tufte-handout}

\include{comment}

\pagenumbering{arabic}
\usepackage{amsmath}
\usepackage[pdftex]{graphicx}
\usepackage{color}
\usepackage{enumitem}
\usepackage{natbib}
\usepackage{comment}
\usepackage{bm}
\usepackage{mdwlist}

% The following package makes prettier tables.  We're all about the bling!
\usepackage{booktabs}

% The units package provides nice, non-stacked fractions and better spacing
% for units.
\usepackage{units}

\newcommand{\angstrom}{\text{\normalfont\AA}}

% The fancyvrb package lets us customize the formatting of verbatim
% environments.  We use a slightly smaller font.
\usepackage{fancyvrb}
\fvset{fontsize=\normalsize}

% Small sections of multiple columns
\usepackage{multicol}

\usepackage{hyperref}
\hypersetup{
    colorlinks=true,       % false: boxed links; true: colored links
    linkcolor=red,       % color of internal links
    citecolor=red,        % color of links to bibliography
    filecolor=magenta,      % color of file links
    urlcolor=blue           % color of external links
}

\newcommand{\cs}{${}^{137}_{\ 55}{\rm Cs}$ }
\newcommand{\ba}{${}^{137}_{\ 56}{\rm Ba }$}
\newcommand{\bam}{${}^{137}_{\ 56}{\rm Ba^* }$}

%\pagestyle{myheadings}
%\markright{Fall~2016\hspace{1.9in}Mechanics Lab}


%\documentclass[12pt]{aastex}
%\usepackage{graphicx,mdwlist,longtable,url}
%\usepackage[final]{pdfpages}
%
%\usepackage[title,titletoc,toc]{appendix}
%\usepackage{longtable}
%
%% left-justify the section headings and make them bigger
%\usepackage{titlesec}
%\newcommand*{\justifyheading}{\raggedright}
%\titleformat*{\section}{\Large\bfseries}
%\titleformat*{\subsection}{\large\bfseries}
%
%\renewcommand{\bottomfraction}{0.9}
%
%\setlength{\parindent}{0.cm}
%\setlength{\parskip}{0.3cm}
%\setlength{\topmargin}{-1.3cm}
%\setlength{\textheight}{9in}
%\setlength{\evensidemargin}{0cm}
%\setlength{\textwidth}{6.46in}
%\setlength{\oddsidemargin}{0in}
%
%\pagestyle{myheadings}
%\markright{Fall~2016\hspace{1.9in}Mechanics Lab}
%
%% these make the longtables span the width of \textwidth
%\setlength\LTleft{0pt}
%\setlength\LTright{0pt}



\begin{document}
{\LARGE {\em 
\noindent Modern Physics---PHYS~220
\vspace{0.5mm}

\noindent Fall 2023
\vspace{3mm}
}}


{\LARGE {\em \noindent Lab: The Speed of Light}}

\large{\noindent Josh Diamond,  John Cummings, George Hassel, Mark Rosenberry, \& Matt Bellis}


\vspace{0.5cm}
\noindent{\bf \LARGE Week 1 - Prelab}\\
\vspace{0.5cm}

\section{Overview}
    In this lab you will measure the speed of
    Light by determining the time required for light to traverse a known
    distance and calculate
    the velocity using

    $$v = \frac{\Delta x}{\Delta t}$$

    Please compare the value you obtain (including uncertainty) to the standard value of c,
    299,792,458 m/s. Is your measurement consistent or not?

\section{Prelab questions}

\begin{comment}
    Hand in your responses to these questions at the end of lab today.  Please email me either an electronic file or a legible photo/scan of a handwritten page(s).  Each person should hand in their own.

    \begin{enumerate}

        \item A car drives from Albany, NY to New York City, traveling 154 miles in 2 hours and 34 minutes. What
            is the average speed of the car in miles per hour (mph)? In kilometers per hour (km/h)? In meters per second (m/s)? 

        \item How fast can you run? Give your answer in meters per second and in miles per hour? How fast was 
            Usain Bolt?

        \item What is the fastest object humanity has ever devised? Give your answer in both mph and m/s.

        \item How fast does a point on the Earth's surface move at the equator (m/s)? How fast is the Earth traveling
            around the sun (m/s)?

        \item If a plane was traveling at 10x the speed of sound, how long would it take to go around the Earth at 
            the equator?

        \item An athlete is training by running as fast as they can between two markers placed 20 meters
            apart. They are moving at 3.5 m/s, on average and they go back and and forth between the two 
            markers 12 times. How long does it take them? 

        \item Suppose the markers are 21 meters apart. How long does it take the athlete to do the 12 repetitions,
            assuming they still move at 3.5 m/s?

        \item A bumblebee flaps its wings about 200 times per second. What is this rate in Hertz (Hz)? How many times
            does a bumblebee flap its wings in 1 minute? What is the length of time between each flap?


    \end{enumerate}
\end{comment}


    \begin{enumerate}
        \item Using the standard value of the speed of light in vacuum and looking up distances, how long does it take to get transmit a radio signal
            \begin{enumerate}
                \item From New York City to Los Angeles?
                \item From the Earth to the Moon?
                \item From the Sun to the Earth?
            \end{enumerate}

        \item What is the speed of light in feet/nanoseconds?
        \item If your detector is getting a ringing (An electronic echo) of its original signal recurring 12 ns after the initial signal, how far away would the reflection be taking place?
        \item What is the wavelength of the laser we will use for this lab? Will that value affect our measurement?

        \item Look over the Universal Laser Receiver Operator’s Manual
            What is the smallest horizontal scale for the oscilloscopes available in our laboratory today?

    \end{enumerate}

\section{Experimental design}
Consider the laser, receiver and other apparatus. How do you combine them to measure c?
Discuss, and design a procedure you submit to Canvas with the above questions.

Once this had been submitted, please go ahead and read the full laboratory instructions. Did
you take the same approach? Are there things you left out? Perhaps things you think we left
out? Submit your Comments to Canvas as well.

\end{document}

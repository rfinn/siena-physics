\section{Theory}

This experiment explores the use of the Michelson interferometer to
determine the wavelength of visible light and perform other precision
measurements.

The basic principle of all interferometry measurements is that 
if two coherent light beams originate from a common source and 
travel different paths to a common end point, interference will 
occur at the end point. The nature of the interference is determined 
by the difference in the optical path lengths. A schematic diagram 
of a Michelson interferometer is shown in Figure~\ref{schematic} below.

\begin{figure}[htbp]
\begin{center}
\includegraphics[width=0.6\textwidth]{../images/inter.png}
\caption{Michelson Interferometer }
\label{schematic}
\end{center}
\end{figure}

A monochromatic light beam enters the interferometer along line $AB$. At
$B$ the beam encounters a half-silvered mirror which functions as a beam
splitter. Part of the beam continues along the same direction, reflects
from the movable mirror at $C$ and returns to $B$. The remainder of the beam
reflects from the beam splitter, strikes the fixed mirror at $D$ and
returns to the beam splitter at $B$. The two beams recombine at $B$ and
travel together to the screen. 

During the time when the beams are split, they travel different 
paths. One path is of length equal to twice the distance from 
$B$ to $C$, denoted $\langle BC\rangle$. The other path length is twice 
the distance from $B$ to $D$, denoted $\langle BD\rangle$. The difference 
between these path lengths determines the nature of the interference 
seen at a particular point on the screen. If we let the difference in path lengths be $PD$, we have

\begin{equation}
PD = 2\langle BC\rangle - 2\langle BD\rangle 
\label{eq:pathlength}
\end{equation}

then the condition for constructive interference is 

\begin{equation}
 PD = m\lambda
\label{eq:constructive}
\end{equation}
and the condition for destructive interference is

\begin{equation}
PD = (m+1/2)\lambda.
\label{eq:destructive}
\end{equation}

In general the interference pattern on the screen will consist of alternating
bright and dark regions (usually fine curved lines) called
``fringes''.\sidenote{In Eqs.~\ref{eq:constructive}~\&~\ref{eq:destructive},
  $\lambda$ is the light wavelength and $m$ is an integer (0, 1, 2, ...).}

Turning the micrometer dial on the interferometer displaces the moveable mirror,
changing the distance $\langle BC \rangle$, while $\langle BD\rangle $ remains
fixed. Thus, the path length difference $PD$ between the two beams can be varied
in a controlled manner. If the distance $\langle BC\rangle$ changes from an
initial value of $\langle BC\rangle_{1}$ to a final value $\langle
BC\rangle_{2}$ , then, using Eq.~\ref{eq:pathlength}, the change in the path
length difference PD is given by:
\begin{equation}
 PD_2 - PD_{1} = 2\langle BC\rangle_{2} - 2\langle BD\rangle - (2\langle
 BC\rangle_{1} - 2\langle BD\rangle).
\label{eq:deltapathlength}
\end{equation}

\noindent Hence,
\begin{equation}
PD_{2} - PD_{1} = 2(\langle BC\rangle_{2}  - \langle BC\rangle_{1}) = 2(d_{2} -
d_{1}),
\end{equation}
\noindent where $d_{1}$ is the initial micrometer reading and $d_{2}$ is the
final micrometer reading. 

Each time PD changes by one wavelength $\lambda$, the integer $m$ changes
by one unit (see Eqs.~\ref{eq:constructive}~\&~\ref{eq:destructive}), and the
observed pattern of interference fringes shifts by one fringe. As the micrometer
reading changes from $d_{1}$ to $d_{2}$, $m$ will change from $m_{1}$ to
$m_{2}$. Combining Eqs.~\ref{eq:constructive} and ~\ref{eq:deltapathlength}, we
can express 
\begin{align}
 \Delta m &= m_2 - m_1 \nonumber \\
 &= \frac{PD_{2}}{\lambda} - \frac{PD_{1}}{\lambda} \nonumber \\
 &= \frac{2(d_{2} - d_{1})} {\lambda} \nonumber
\end{align}
so
\begin{equation}
\boxed{\Delta m = \frac{2(d_{2} - d_{1})} {\lambda}}
\label{eq:delta-m}
\end{equation}
where $\Delta m$ is the number of fringes shifted past a fixed point as the
micrometer dial reading changes from $d_{1}$ to $d_{2}$. 

On this apparatus, the micrometer dial is designed so that one 
complete revolution of the micrometer dial corresponds to a change 
$d_{2} - d_{1} = 25~\mu$m.\sidenote{Note: 1 $\mu$m $= 10^{-6}$ m.}

%\subsection{Calibration of Michelson Interferometer}
%
%For precise work one can calibrate the micrometer dial by counting 
%the fringes shifted due to a given micrometer movement, if the 
%wavelength of the light source is known. Assume that \ensuremath{\Delta}m 
%fringes are shifted by a change d$_{2}^\prime$ - d$_{1}^\prime$ of the micrometer 
%dial. Using Eq.~\ref{eq:delta-m} we can calculate the change d$_{2}$ - d$_{1}$ expected 
%for perfect calibration. We then define the correction factor 
%$F$ as: 

%\begin{equation}
%\boxed{ F = {(d_2- d_1) \over (d_2^\prime - d_1^\prime)} }
%\label{eq:calibration}
%\end{equation}

%Once $F$ is determined, it may be used to correct other measurements. 
%One simply multiplies any micrometer dial reading change $d_2^\prime - 
%d_1^\prime $ by $F$ to obtain the corrected value $d_2 - d_1$.

\subsection{Index of refraction of a glass slide}

It is difficult to measure the index of refraction of a thin slab of transparent
material by observing refraction, because a beam refracted within the material
will return to its original direction after exiting the slab.\sidenote{Be sure
  you understand this statement by making a sketch.} Although the beam is
shifted slightly parallel to itself, this effect is not normally easily
measurable. 

An attractive solution to this problem is to measure the fringe shift occurring
when the slab is rotated in the path of one of the interferometer beams--while
the other beam is unchanged.  This rotation causes the light beam to traverse a
thicker portion of the material, as the slab is turned away from being
perpendicular to the light beam. Since the light wavelength is shorter in the
slab than in air\sidenote{Why?}, the beam passing through the slab will contain
increasingly more wavelengths as the angle from the perpendicular orientation
increases. For each additional wavelength contained in the slab (compared to an
equal thickness of air), the interference fringe pattern will shift by one
fringe.

Thus, there is a relationship between the following quantities:
\begin{description}
\item[$n$]  the index of refraction of the slab
\item[$t$]  the thickness of the slab
\item[$\theta$]  the angle through which the slab is rotated, relative to an  initial position in which the slab is perpendicular to the  incident light beam
\item[$\Delta m$]  the number of fringes passing a fixed point when the slab 
is  rotated by angle \ensuremath{\theta} from its initial position.
\end{description}

To a good approximation, it can be shown that this relationship is: 

\begin{equation}
\boxed{ n = { t - \lambda ( \Delta m/2) \over t - \left[ \lambda ( \Delta
      m/2)/(1-\cos\theta) \right] } 
\label{eq:glassslide}}
\end{equation}

\subsection{Index of refraction of a gas}

If one of the interferometer beams travels through a sealed chamber, 
that beam will travel more rapidly inside the chamber than it 
would outside the chamber if the pressure of the air inside the 
chamber is less than normal atmospheric pressure (i.e. closer 
to being a vacuum).

This means that the index of refraction n of air actually depends 
on the air pressure $P$, and we shall denote this index of refraction 
as $n_P$ . Since the light frequency is unaffected by changes in 
the medium, the wavelength of the light being used depends on 
the index of refraction in the same way as the velocity of light, 
i.e.

\begin{equation}
\lambda_P = {\lambda \over n_P}
\label{eq:lambdaP}
\end{equation}

\noindent where $\lambda$ is the wavelength of the light in a vacuum and
$\lambda_P$ is the wavelength in air at pressure $P$.

In this experiment, we form the usual interference pattern with the two
interferometer beams, but now with one of the beams traveling through the air
chamber. At the initial atmospheric pressure, the number of wavelengths of light
contained in the air chamber is equal to $2t/\lambda_{\rm atm}$, where $t$ is
the thickness of the air layer in the chamber measured along the light
path.\sidenote{Where does the factor of two come from?} As the pressure inside
the chamber decreases, $n_P$ decreases, $\lambda_P$ increases, and fewer
wavelengths are contained inside the chamber. The change in the number of
wavelengths in the chamber as the pressure is varied will equal the number
$\Delta m$ of fringes that move past a fixed point in the interference pattern.

In the experiment, we begin with the air pressure in the chamber 
equal to atmospheric pressure, so that the index of refraction 
initially is $n_{\rm atm}$. The pressure is then decreased to a value 
$P$. The number of wavelengths inside the chamber decreases to 
$2t/\lambda_P$, and it follows that
\begin{align}
\Delta m &= \frac{2t}{\lambda_{\rm atm}} - \frac{2t}{\lambda_P} \nonumber \\ 
 &= \frac{2t}{\lambda} ( n_{\rm atm} - n_P ),
\label{eq:delta-m-P}
\end{align}

\noindent where $\Delta m$ is the number of fringes passing a fixed point while
the gas pressure in the chamber varies from atmospheric pressure to a pressure
$P$, and where the second equality was obtained using Eq.~\ref{eq:lambdaP}.

At a final gas pressure of zero in the chamber (a perfect vacuum), 
$n_P = 1$, and Eq.~\ref{eq:delta-m-P} becomes
\begin{equation}
\boxed{ (\Delta m)_{P=0} = \frac{2t}{\lambda}(n_{\rm atm} - 1)}
\label{eq:delta-m-P0}
\end{equation}

Although $(\Delta m)_{P=0}$ is not directly measurable (because we cannot obtain
a perfect vacuum in the chamber), it can be determined from the other $\Delta m$
measurements by graphical extrapolation. Knowing it, we can then use
Eq.~\ref{eq:delta-m-P0} to find $n_{atm}$, the index of refraction of air at
atmospheric pressure.

\section{Preliminary Question}

\begin{enumerate}
\item A monochromatic light source is used in a Michelson interferometer.  The
  micrometer dial controlling the moveable mirror is rotated until 100
  interference fringes pass a fixed point. The difference between the final and
  initial micrometer readings is 32.0 microns.\sidenote{The 32.0~microns is the
    displacement of the moveable mirror.} Use the given information to determine
  the wavelength of the light.
% HINT: Use Equation~\ref{eq:delta-m}.
\end{enumerate}

\section{Equipment}
PASCO Michelson interferometer, laser, lens, screen, 
mounted glass slide, micrometer, and gas cell.

\section{Procedure}

Caution: IN THOSE PORTIONS OF THE EXPERIMENT USING A LASER LIGHT 
SOURCE, BE CAREFUL NOT TO LOOK DIRECTLY INTO THE BEAM OR TO LET 
A DIRECT REFLECTION OF THE BEAM ENTER YOUR EYE.

\begin{enumerate}

\item Adjustment of Michelson interferometer

\begin{enumerate}
\item Use a He-Ne laser beam with the interferometer set up as 
in Fig.~\ref{schematic}. By adjusting the orientation of the fixed mirror, 
align the two recombined interferometer beams so that they are 
superimposed on the screen. 
\item Insert a lens between the laser and the beam splitter to 
broaden the laser beam. Observe the interference fringes projected 
on a screen.
\item Carefully adjust the lens and the orientation of the fixed 
mirror to obtain an interference pattern of concentric rings 
(resembling a target with a ``bulls-eye'').\sidenote{Explain 
why a ring pattern is formed. \textit{Hint: What is the shape of the 
laser light wave fronts when the lens is present?}} 
\end{enumerate}

%\item Calibration of Michelson interferometer
%
%\begin{enumerate}
%\item Set the micrometer dial to a low (but not zero) reading. 
%Record this reading. Then advance the micrometer dial until 30 
%fringes (i.e. bright rings) have passed a fixed point on the 
%screen, and record the micrometer dial reading. Repeat this process 
%three more times using different parts of the micrometer dial 
%scale, but not too close to the ends of the scale.

%Note: The micrometer dial is designed so that one complete 
%revolution of the micrometer dial corresponds to a change $d_2 - 
%d_1 = 25$ microns. Note: $1 {\rm micron } = 1 \mu {\rm m} = 10^{-6} {\rm m}$.


 
%\item Compute the average value of the shift in mirror position 
%$d_2^\prime - d_1^\prime$ for the four trials in step a. Referring to the 
%theory section, find the correction factor $F$ (Note: He-Ne Laser 
%light wavelength in a vacuum $= \lambda = 632.8$ nm). Check 
%your results with the instructor. Your value of $F$ will be used 
%in analyzing results of some of the other interferometer measurements. 
%\end{enumerate}

\item \label{proc:lambda} Measuring the wavelength of a He-Ne laser

The wavelength of the laser can be determined by using the micrometer
dial to displace the movable mirror by a known amount and counting
the corresponding number of fringes that pass by on the screen.
Count a fairly large number of fringes, say roughly 30, and note the
micrometer reading before and after.  Repeat this measurement several
times using different parts of the micrometer dial to reduce errors from
nonlinearities in the mirror mechanism.

\item Qualitative observations

\begin{enumerate}
\item What happens to the interference pattern when an opaque sheet 
is placed in the path of one of the two beams produced by the 
interferometer before they recombine. Why?

\item What happens to the interference pattern when the table holding 
the interferometer is tapped or lightly jarred? Explain the sensitivity 
of the pattern to these perturbations.
\end{enumerate}

\item Index of refraction of a glass slide\sidenote{Note: the micrometer dial
  controlling the moveable mirror is not used in this part of the experiment.} 

\begin{enumerate}
\item We use the laser light source and a glass slide as a sample.
Mount the glass slide on a magnetic holder placed at the center of a
table, which allows it to be rotated about a vertical axis.  The angle of
rotation can be read using the scale printed on the interferometer base.
The interferometer is designed so that slide is in the path of the laser
beam passing between the beam splitter and the moveable mirror of the
interferometer (path $BC$ in Fig.~\ref{schematic}).

\item Align the slide initially so that its plane is perpendicular to the laser
  beam passing through it. The fringe pattern itself can be used to determine
  this position---it is the point at which the fringes begin to reverse
  direction as the slide is slowly rotated.\sidenote{Why do the fringes reverse
    direction at this point? \textit{Suggestion: To accomplish the alignment,
      set the rotating table so that its arm is accurately at zero on the
      angular scale. Then, without allowing the table to move, carefully rotate
      the mount holding the glass slide until you observe the reversal of the
      fringe motion. Position the slide at the turning point just before the
      reversal begins.}}

\item Slowly rotate the table by moving its arm (without touching 
the slide), counting the number of fringes passing a fixed point. Continue until you 
reach 100 fringes, or until you reach the maximum calibrated 
angle (20 degrees)---whichever comes first. Record the number 
of fringes counted and the angle $\theta$ through which the 
slide has been rotated.\sidenote{$\theta$ should be determined to the 
nearest tenth of a degree.} Also measure the thickness of the slide with a
micrometer to three significant figures.
\end{enumerate}

\item Index of refraction of a gas\sidenote{Note: the micrometer dial
  controlling the moveable mirror is not used in this experiment.} 

Mount the gas cell so that one of the interferometer beams passes 
through it, perpendicular to its side windows. Count the fringes 
passing a fixed point as the pressure in the chamber is slowly 
reduced from atmospheric to the minimum value obtainable. Record 
the {\em cumulative number of fringes passing} (starting from 
atmospheric pressure) for each increment of 10 cmHg of pressure 
decrease.\sidenote{Note that the pressure gauge is actually a vacuum gauge, and
  gives a reading of the {\em pressure decrease} $\Delta P = P_{\rm atm} - P$ in
  cmHg, where $P_{\rm atm}$ is atmospheric pressure and $P$ is the pressure in
  the chamber. $P_{\rm atm}$ is approximately equal to 76 cmHg, and can be
  determined more precisely from a barometer.} Also measure the thickness $t$ of
the air space in the chamber along the light path. 

\end{enumerate}

\section{Analysis}

\begin{enumerate}
\item Compute the wavelength of the laser light using the data you collected in
  procedure step~\ref{proc:lambda} using Eq.~\ref{eq:delta-m}.\sidenote{Note:
    For a He-Ne Laser light wavelength in a vacuum $\lambda=632.8$~nm.}

\item Compute, using Eq.~\ref{eq:glassslide}, the index of refraction of the
  glass slide studied in procedure step~4. Compare to typical tabulated values
  of the index of refraction of glass.\sidenote{Note: Eq.~\ref{eq:glassslide} is
    complicated and care must be taken in the computation. Do not round off
    numerical data values or any intermediate numerical results. How many
    significant figures should your final result for $n$ have?}

\item Plot the cumulative number of fringes $\Delta m$ passing 
as a function of the pressure decrease $\Delta P = P_{\rm atm} - 
P$ for your data taken with the gas cell. Use Python (or your favorite
programming language) to fit a linear model through your data.
%Draw the best straight line through your points (or use a computer curve fit). 
%Extrapolate your graph to pressure $P = 0$ (i.e., to pressure decrease $\Delta
%P = P_{\rm atm}$), and determine $(\Delta m)_{P=0}$ from the extrapolated line.  
Use the intercept of your fit to infer $(\Delta m)_{P=0}$, the number of fringes
at $P=0$ (i.e., to pressure decrease $\Delta P = P_{\rm atm}$). 

Then use Eq.~\ref{eq:delta-m-P0} to calculate the index of refraction of air 
at atmospheric pressure, $n_{\rm atm}$. Compare this result to a tabulated 
value.\sidenote{Suggestion: First compute $n_{atm} - 1$ using
  Eq.~\ref{eq:delta-m-P0}. Then you can easily write the result for $n_{\rm
    atm}$ itself. Note: The calculated $n_{\rm atm}$ will be slightly greater
  than 1, and {\em the final result must not be rounded off to 1.0!}}

\end{enumerate}

\subsection{Uncertainty Analysis}

\begin{enumerate}

\item Measuring the wavelength: There are two measurements: the change in the
  number of fringes $\Delta m$, and the displacement of the movable mirror,
  $\Delta d = d_2-d_1$, as measured by the micrometer. Make at least 2-3
  different measurements, or two different ranges of displacement.  The
  expression for wavelength $\lambda$ is a simple quotient, so the uncertainty
  in the wavelength is a simple addition of the uncertainties in quadrature:
  \begin{align*}
    \sigma_\lambda = \lambda \sqrt{\left(\frac{\sigma_m}{m}\right)^2 +
      \left(\frac{\sigma_{\Delta d}}{\Delta d}\right)^2}.
  \end{align*}
  Note that if one of these fractions is significantly smaller than the other,
  you can omit it.\sidenote{Why?} Be sure to check that the experimental value
  for $\lambda$ you obtain agrees with your expectation for red light, and with
  your answer to the preliminary question.

\item Measuring the index of refraction of a glass slide: The expression for
  $n_{\mathrm{glass}}$ is an approximation and is somewhat complex. To simplify
  the uncertainty analysis, estimate uncertainty in the slide thickness, $t$,
  the change in the number of fringes, $\Delta m$, and the wavelength,
  $\lambda$, and find the relative uncertainty in each.\sidenote{In other words,
    divide the uncertainty by the measurement.} Now, the angle $\theta$ appears
  in the equation not directly as the angle, but as $\cos\theta$; however, note
  that
  \begin{align*}
    \sigma(\cos\theta) = \left| \frac{\partial
      \cos\theta}{\partial\theta}\right| d\theta \approx \sin\theta,
  \end{align*}
  and so the uncertainty in $n_{\mathrm{glass}}$ becomes
  \begin{align*}
    \sigma_n = n \sqrt{\left(\frac{\sigma_t}{t}\right)^2 +
      \left(\frac{\sigma_m}{m}\right)^2 +
      \left(\frac{\sigma_\lambda}{\lambda}\right)^2 + (\tan\theta)^2}.
  \end{align*}
  Of course, if one or more of these fractions is significantly smaller than the
  other, you can omit it.

\item Measuring index of air: Estimate the uncertainty in the measured
  uncertainties in the thickness $t$, the wavelength $\lambda$, and the change
  in the number of fringes, $\Delta m$.  Use a similar addition-in-quadrature
  developed above in order to estimate the uncertainty in $(n_{\mathrm
    atm}-1)$. 
\end{enumerate}

%\section{References}
%Serway and Jewett, {\underline {Physics for Scientists and 
%Engineers}}, 6$^{th}$ edition, pp. 1177-1181, 1194 

\documentclass{tufte-handout}
\pagenumbering{arabic}
\usepackage{amsmath}
\usepackage[pdftex]{graphicx}
\usepackage{color}
\usepackage{enumitem}
\usepackage{natbib}
\usepackage{comment}
\usepackage{bm}
\usepackage{mdwlist}

\usepackage{hyperref}
\hypersetup{
    colorlinks=true,       % false: boxed links; true: colored links
    linkcolor=red,       % color of internal links
    citecolor=red,        % color of links to bibliography
    filecolor=magenta,      % color of file links
    urlcolor=blue           % color of external links
}

\newcommand{\cs}{${}^{137}_{55}{\rm Cs}$ }
\newcommand{\ba}{${}^{137}_{56}{\rm Ba }$}
\newcommand{\bam}{${}^{137}_{56}{\rm Ba^* }$}

%\pagestyle{myheadings}
%\markright{Fall~2016\hspace{1.9in}Mechanics Lab}


%\documentclass[12pt]{aastex}
%\usepackage{graphicx,mdwlist,longtable,url}
%\usepackage[final]{pdfpages}
%
%\usepackage[title,titletoc,toc]{appendix}
%\usepackage{longtable}
%
%% left-justify the section headings and make them bigger
%\usepackage{titlesec}
%\newcommand*{\justifyheading}{\raggedright}
%\titleformat*{\section}{\Large\bfseries}
%\titleformat*{\subsection}{\large\bfseries}
%
%\renewcommand{\bottomfraction}{0.9}
%
%\setlength{\parindent}{0.cm}
%\setlength{\parskip}{0.3cm}
%\setlength{\topmargin}{-1.3cm}
%\setlength{\textheight}{9in}
%\setlength{\evensidemargin}{0cm}
%\setlength{\textwidth}{6.46in}
%\setlength{\oddsidemargin}{0in}
%
%\pagestyle{myheadings}
%\markright{Fall~2016\hspace{1.9in}Mechanics Lab}
%
%% these make the longtables span the width of \textwidth
%\setlength\LTleft{0pt}
%\setlength\LTright{0pt}



\begin{document}
{\LARGE {\em 
\noindent Modern Physics---PHYS~220
\vspace{0.5mm}

\noindent Fall 2021
\vspace{3mm}
}}


{\LARGE {\em \noindent Lab: Michelson Interferometry}} 

\large{\noindent Josh Diamond, John Cummings, Mark Rosenberry, \&  Matt Bellis}




\vspace{0.5cm}
\noindent{\bf \LARGE Week 1 - Prelab}\\
\vspace{0.5cm}

\section{Prelab questions}
Submit on Canvas your answers to these questions at the end of lab today. Each person should hand in their own.

\begin{enumerate}

\item What is meant by the {\em phase} of a wave?  What is meant by the {\em phase difference} of two waves? What do {\em in-phase} and {\em out-of-phase} mean?  How do those ideas relate to {\em constructive} and {\em destructive} interference?

\item Consider two waves of equal wavelength $\lambda$ that originate from 
different sources but meet at the same point.
If the waves are perfectly {\it in phase}, what behavior do you expect where the waves meet?
What does it mean if they are {\it in phase}? Draw any pictures you need to explain this process.

\item Consider two waves of equal wavelength $\lambda$ and assume that they start in phase at their respective
points of origin. If one wave travels a longer distance $\Delta L$ such that $\Delta L = \lambda$
What do you expect when the waves meet? What about for the following scenarios?
\begin{itemize}[itemsep=0pt,parsep=0pt,topsep=0pt,partopsep=0pt]
    \item $\Delta L = \frac{1}{2}\lambda$
    \item $\Delta L = \frac{3}{2}\lambda$
    \item $\Delta L = 3\lambda$
    \item $\Delta L = 978\lambda$
    \item $\Delta L = \frac{2}{5}\lambda$
    \item $\Delta L = 1.769\lambda$
\end{itemize}

\item Review the concepts of the double-slit interference pattern. Draw a sketch of this setup and
all relevant quantities that you would want to measure. How would you explain the pattern that appears on the 
viewing screen?

\item Suppose you have two narrow slits separated by distance $d$. A viewing screen is distance $L$ away. If you are
using light of wavelength $\lambda$, what is the separation between the bright spot in the middle of
the viewing screen and the first maximum on one side? If you use light of a smaller wavelength, does
the separation get larger or smaller?

\item What is a {\it micron}?

\item What is {\it refraction}?

\item Is the speed of light (electromagnetic waves) the same in all materials?

\item What is the {\it index of refraction} of a material?

\item Consider a wave with wavelength $\lambda_1$ and frequency $f_1$ that is moving in some medium
with velocity $v_1$. It then passes into a medium with a different velocity $v_2$. In this new
medium, are the wavelength or freqency different? The same?

\end{enumerate}

\section{Experimental design}

Look at the equipment available to measure with (HeNe laser, PASCO Michelson Interferometer, mounted glass slide, gas cell.)
Now, please design a procedure for the measurements described above.\\

One you have submitted that, please read the full instructions we have provided on how to make your measurements.  COMMENT in Canvas on the agreement or difference from your own plan.

\end{document}



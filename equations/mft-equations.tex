\documentclass[12pt]{article}
\usepackage[hmargin=1.in,vmargin=1in]{geometry}
\begin{document}
\begin{twocolumn}
% %\vspace*{-5cm}
\begin{flushleft}
{\bf Kinematics}\\
\medskip
$\displaystyle x_f = x_i + v_{ix} t + \frac{1}{2} a_x (\Delta t)^2$ \\
\medskip
$\displaystyle x_f = x_i + \frac{1}{2} ( v_{ix} + v_{fx})t$  \\
\medskip
%$\displaystyle y_f = y_i + v_{iy} t + \frac{1}{2} a_y (\Delta t)^2$ \\
%\medskip
%$\displaystyle x_f = x_i + (v_{fx} - v_{ix}) \Delta t$ \\
%\medskip
%$\displaystyle y_f = y_i + (v_{fy} - v_{iy}) \Delta t $\\
%\medskip
$\displaystyle v_{fx} = v_{ix} + a_x \Delta t $ \\
\medskip
%$\displaystyle v_{fy} = v_{iy} + a_y \Delta t $ \\
%\medskip
$\displaystyle v_{fx}^2 = v_{ix}^2 + 2 a_x (x_f - x_i)$\\
\medskip
%$\displaystyle v_{fy}^2 = v_{iy}^2 + 2 a_y (y_f - y_i)$\\
%\medskip
$\displaystyle v_{x} = \frac{\Delta x}{\Delta t} = \frac{dx}{dt} $\\
\medskip
$\displaystyle a_{x} = \frac{\Delta v_x}{\Delta t} = \frac{dv_x}{dt} = \frac{d^2x}{dt^2} $\\
\medskip

\bigskip
{\bf Forces}\\
\medskip
$\displaystyle f_{s} = \mu_{s} n$ \\
\medskip
$\displaystyle f_{k} = \mu_{k} n$ \\
\medskip
$\displaystyle F_{spring} = -k (x - x_0)$\\
\medskip
$\displaystyle \vec{F}_{\rm net} =  m \vec{a}$ \\
\medskip
%$\displaystyle F_{\rm net,y} = \Sigma F_y = m a_y$ \\
%\medskip
%$\displaystyle F_{\rm net,y} = \Sigma F_y = m a_y$ \\
%\medskip
%$\displaystyle F_{\rm net,z} = \Sigma F_z = m a_z$ \\
%\medskip
%$\displaystyle F_{\rm net,r} = \Sigma F_r = m a_r$ \\
%\medskip
%$\displaystyle F_c = \frac{m v^2}{r}$\\
%\medskip
%$\displaystyle F_g = \frac{G m_1 m_2}{r^2}$\\
%\medskip
%$\displaystyle E_g = -\frac{G m_1 m_2}{r}$\\ 
%\medskip
%$\displaystyle \Sigma \vec{F} = \frac{d \vec{p}}{d t}= \frac{\Delta
%  \vec{p}}{\Delta t} = m \vec{a}$\\ 

\bigskip
{\bf Momentum }\\
\medskip
$\displaystyle \vec{p} = m \vec{v}$ \\
%\medskip
%$\displaystyle p_y = m v_y$ \\
\medskip
$\displaystyle \Delta p_{x} = J_x = F_x \Delta t$ \\
\medskip
%$\displaystyle \Delta p_{x} = 0 \ (isolated~system)$ \\
$\displaystyle \Delta \vec{p} = 0 \ (isolated~system)$ \\
\medskip
$\displaystyle \vec{X}_{\rm CM} = \frac{1}{M}\sum m_i\vec{x}_i \ \left(M = \sum m_i\right)$ \\


\bigskip
{\bf Energy}\\
\medskip
$\displaystyle {K} = \frac{1}{2} m v^2$ \\
\medskip
$\displaystyle U_{g} = m g (y - y_0)$ \\
\medskip
$\displaystyle U_{spring} = \frac{1}{2} k (x-x_0)^2$\\
\medskip
%$\displaystyle \Delta E  = \Delta K + \Delta U + \Delta E_{th} = 0 \ (isolated~system)$ \\
%\medskip
$\displaystyle \Delta E_{system}  = W \ (Work)$ \\
%\medskip
%$\displaystyle W = F\Delta x \cos \theta$ \\

\medskip
$\displaystyle W = F \cdot d = F\Delta x \cos \theta = F_\parallel d$\\
%\medskip
%$\displaystyle J  = \Delta p =  F_{ave} \cdot \Delta t = \int F dt $\\

%\medskip
%$\displaystyle P = \frac{W}{t}$ \\
\medskip
\medskip



{\bf Circular Motion} \\
\medskip
%$\displaystyle a_{r} = \frac{v^2}{r} = r\omega^2$\\
%\medskip
$\displaystyle \sum F_{r} = \frac{m v^2}{r} $\\
\medskip
$\displaystyle \omega = 2 \pi f $ \\
\medskip
$\displaystyle T = \frac{1}{f} $ \\
\medskip


{\bf Rotational Motion} \\
\medskip
$\displaystyle s = r \theta$ \\
\medskip
$\displaystyle v = r \omega$ \\
\medskip
$\displaystyle a = r \alpha $\\
\medskip
$\displaystyle \theta_f = \theta_i + \omega_{i} t + \frac{1}{2} \alpha (\Delta t)^2$ \\
\medskip
%$\displaystyle \theta_f = \theta_i + \frac{1}{2} (\omega_{i} + \omega_{f}) \Delta t$ \\
%\medskip
$\displaystyle \omega_{f} = \omega_{i} + \alpha \Delta t $ \\
\medskip
$\displaystyle \omega_{f}^2 = \omega_{i}^2 + 2 \alpha (\theta_f - \theta_i)$\\

\medskip
$\displaystyle K_{rot} = \frac{1}{2} I \omega^2 $\\
\medskip
$\displaystyle \vec{\tau} = \vec{r} \times \vec{F}$\\
\medskip
$\displaystyle \tau = r F_{\perp} = rF \sin \theta$\\
\medskip
$\displaystyle \vec{\tau} = I \vec{\alpha} $\\
\medskip
%$\displaystyle \tau = \frac{\Delta L}{\Delta t} $\\
%\medskip
$\displaystyle I = \sum_i \ m_i \ r_i^2  \ (point \ masses)$\\
\medskip
$\displaystyle I = I_{CM} + m d^2  \ (parallel \ axis \ theorem)$\\
\medskip
$\displaystyle \vec{L} = I \vec{\omega} $\\
\medskip
$\displaystyle L = m v r \sin \theta$\\
\medskip
$\displaystyle \Delta L = 0 \ (isolated ~ system)$\\
\medskip
\medskip

 {\bf Fluids}\\
 \medskip
 $\displaystyle F_{buoyant} = \rho_F V g $\\
 \medskip
 $\displaystyle P = F/A $\\
 \medskip
 $\displaystyle P = \rho g \Delta h $ \\
 \medskip
$\displaystyle \rho_1 A_1 v_1 = \rho_2 A_2 v_2 $ \\
\medskip
$\displaystyle  P_1 + \frac{1}{2} \rho v_1^2 + \rho g h_1 = P_2 + \frac{1}{2} \rho v_2^2 + \rho g h_2$ \\
\medskip

% \medskip
 \medskip
 {\bf Gravity} \\
\medskip
$\displaystyle \vec{F}_G  = -\frac{G m_1 m_2}{r^2} \hat{r}$ \\

\medskip
\medskip
%{\bf Constants} \\
%\medskip
%$\displaystyle \rm g = 9.8~ m/s^2 $ \\
%  \medskip
%  $\displaystyle \rm G = 6.67 \times 10^{-11} ~ N ~m^2/kg^2$ \\
%  \medskip
% $\displaystyle \rm M_{sun} = 1.99 \times 10^{30} ~ kg$ \\
%  \medskip
% $\displaystyle \rm M_{earth} = 5.98 \times 10^{24} ~ kg$ \\
% \medskip
%  %$\displaystyle \rm M_{moon} = 7.35 \times 10^{22} ~ kg$ \\
%  %\medskip
% $\displaystyle \rm R_{sun} = 6.96 \times 10^{8} ~ m$ \\
% \medskip
% $\displaystyle \rm R_{earth} = 6.38 \times 10^{6} ~ m$ \\
% \medskip
% % %$\displaystyle \rm R_{moon} = 1.74 \times 10^{6} ~ m$ \\
% % %\medskip

% $\displaystyle \rm Earth-Sun \ distance = 1.5 \times 10^{11} ~ m$ \\
%\medskip
%$\displaystyle \rm I_{sphere} = \frac{2}{5} m r^2$\\
%\medskip
%$\displaystyle \rm I_{hoop} =  m r^2$\\
%\medskip
%$\displaystyle \rm I_{cylinder} =  \frac{1}{2} m r^2$\\
%\medskip
%$\displaystyle \rm I_{rod} =  \frac{1}{12} m L^2$\\
% %\medskip
% %$\displaystyle \rho_{water} = 1.0 \times 10^3~ \rm kg~m^3 $ \\
% %\medskip


%$\displaystyle k = 9.0 \times 10^9~ \rm N~m^2/C^2 $ \\
%\medskip
%$\displaystyle \epsilon_0 = 8.85 \times 10^{-12}~ \rm C^2/N\cdot m^2 $ \\
%\medskip
%$\displaystyle \rm 1~ eV = 1.6 \times 10^{-19}~ J$ \\
%\medskip
%$\displaystyle \mu_0 = 4 \pi \times 10^{-7}~ \rm T m / A$ \\
%\medskip
%$\displaystyle v_s = (331 + 0.60 \rm T)~\rm m/s$ \\
%\medskip
%$\displaystyle c = 3.00 \times 10^8~\rm m/s$ \\
%\medskip
%$\displaystyle I_0 = 10^{-12} ~\rm W/m^2$ \\
%\medskip
%$\displaystyle 1 amu = 1 u = 1.6605 \times 10^{-27}~\rm kg = 931.5~MeV/c^2$ \\
%\medskip
%$\displaystyle 1 eV = 1.60 \times 10^{-19} J $ \\
%\medskip
%$\displaystyle h = 6.626 \times 10^{-34} ~\rm J~s$ \\
%\medskip
%$\displaystyle m_e = 9.11 \times 10^{-31} ~\rm kg$ \\
%\medskip
%$\displaystyle m_p = 1.67 \times 10^{-27} ~\rm kg$ \\
% \medskip
% {\bf Geometry} \\
% \medskip
% $\displaystyle C = 2 \pi r$ \\
% $\displaystyle Area =  \pi r^2$ \\
% $\displaystyle Volume = \frac{4}{3} \pi r^3$ \\
% $\displaystyle Surface~Area = 4 \pi r^2$ \\


\bigskip
{\bf Electrostatics}\\
\bigskip
$\displaystyle F = K \frac{q_1 q_2}{r^2} = \frac{1}{4 \pi \epsilon_0} \frac{q_1 q_2}{r^2} $ \\
\medskip
$\displaystyle \vec{E} = \frac{\vec{F}}{q}$ \\
\medskip
$\displaystyle \vec{E} = \frac{k Q}{r^2}\hat{r} ~~~(point\  charge)$ \\
\medskip
%$\displaystyle \Delta PE = - W_{field}$ \\
%\medskip
%$\displaystyle V_{ba} = V_b - V_a = \frac{U_b - U_a}{q} = \frac{-W_{ba}}{q}$ \\
%\medskip
$\displaystyle \Delta U = q~\Delta V $ \\
\medskip
$\displaystyle U_e = K\frac{q_1 q_2}{r} \ ~~(point \ charge)$\\ 
\medskip
$E_{x} = -\frac{d V}{dx}$ \\
\medskip
$F_{x} = -\frac{d U}{dx}$ \\
\medskip
%$\displaystyle E = \frac{ -V_{ba}}{d}$ \\
%\medskip
$\displaystyle V = k \frac{Q}{r} = \frac{1}{4 \pi
  \epsilon_0}\frac{Q}{r}  \ (electric \ potential)$\\
\medskip
$\displaystyle V_{\rm tot} = \Sigma_{i} V_{i}$ \\
%\medskip
%$\displaystyle \Phi_e = \oint \vec{E} \cdot d\vec{A} =
%\frac{Q_{\rm in}}{\epsilon_0} ~~~(Gauss'\  Law)$\\ 
\bigskip

{\bf Current \& Circuits} \\
\bigskip
$\displaystyle I = \frac{\Delta Q}{\Delta t}$ \\
\medskip
$\displaystyle V = I~ R$ \\
\medskip
$\displaystyle R = \rho \frac{L}{A}$ \\
\medskip
$\displaystyle P = I~V = I^2~ R = \frac{V^2}{R}$ \\
\medskip
$\displaystyle R_{eq} = R_1 + R_2 + \cdots = \Sigma_i~ R_i ~ (series) $\\
\medskip
$\displaystyle \frac{1}{R_{eq}} = \frac{1}{R_1} + \frac{1}{R_2} +
\cdots = \Sigma_i ~ \frac{1}{R_i} ~ (parallel)$ \\ 
\medskip
$\displaystyle C_{eq} = C_1 + C_2 + \cdots = \Sigma_i~ C_i ~(parallel) $\\
\medskip
$\displaystyle \frac{1}{C_{eq}} = \frac{1}{C_1} + \frac{1}{C_2} +
\cdots = \Sigma_i ~ \frac{1}{C_i} ~ (series)$ \\ 
\medskip
$\displaystyle Q = C ~ V$ \\
\medskip
$\displaystyle C = \epsilon_0 \frac{A}{d}$ \\
\medskip
%$\displaystyle C = \epsilon \frac{A}{d}$ \\
%\medskip
%$\displaystyle \epsilon = K \epsilon_0$ \\
%\medskip
 $\displaystyle E_{\rm capacitor} =  \frac{\eta}{\epsilon_0} =
 \frac{Q}{\epsilon_0A} $ 

\medskip
$\displaystyle U_{capacitor} = \frac{1}{2} Q V = \frac{1}{2} C V^2 = \frac{1}{2} \frac{Q^2}{C}$ \\
\medskip
$\displaystyle V_C = V_0(1-e^{-t/RC}) $ \\
%\medskip
%$\displaystyle V=\frac{Kq}{r}$ 
\bigskip
%\clearpage

{\bf Magnetism} \\
\bigskip
${\displaystyle \vec{F}_{{\rm on}\ q}=q\left(\vec{E} + \vec{v}\times\vec{B}\right)}$ \\
\medskip
%$\displaystyle  f_{\rm  cyc}=\frac{qB}{2\pi m} $ \\
%\medskip
$\displaystyle \vec{F}_{\rm on \ wire}=l\vec{I}\times\vec{B}, ~|F| = I l B~sin\theta $ \\
\medskip
%$\displaystyle  {\vec{F}_{\rm parallel\  wires}=I_{1}lB_{2}=I_{1}l\frac{\mu_{0}I_{2}}{2\pi  d}=\frac{\mu_{0}lI_{1}I_{2}}{2\pi d}} $ \\
%\medskip

$\displaystyle B_{wire} = \frac{\mu_0 I}{2 \pi r} $ \\
\medskip
 %${\displaystyle B_{\rm solenoid}=\frac{\mu_{0}NI}{l}=\mu_{0}nI} $ \\
%\medskip
%$\displaystyle \Phi_m=\int\vec{B}\cdot d\vec{A} $ \\
%\medskip

$\displaystyle \Phi_B = \vec{B}\cdot\vec{A} = B A cos \theta  $ \\ 
\medskip
$\displaystyle Emf = - \frac{d\Phi_{B}}{dt} = -\frac{\Delta
  \Phi_B}{\Delta t} $ \\
\medskip
%$\displaystyle Emf  = -N \frac{\Delta \Phi_B}{\Delta t} $ \\
%\medskip
%$\displaystyle Emf = N B \omega A sin (\omega t) ~~~(generator) $ \\
%\medskip
 ${\displaystyle \oint\vec{B}\cdot d\vec{s}=\mu_{0}I}$ {\it
   (Amp\`ere's Law)} \\ 
%\medskip
%$\displaystyle \frac{V_S}{V_P} = \frac{N_S}{N_P} ~~~(transformer) $ \\
%\medskip
%$\displaystyle \frac{I_P}{I_S} = \frac{N_S}{N_P}  ~~~(transformer)$ \\
\bigskip

{\bf Maxwell's Equations} \\
\begin{tabular}{l|l}
$\displaystyle \oint \vec{E} \cdot d\vec{A} = \frac{Q_{\rm in}}{\epsilon_0}$  & $\nabla \cdot  \vec{E}=\frac{\rho_{in}}{\epsilon_0} $ \\
$\displaystyle \oint \vec{B} \cdot d\vec{A} = 0$  & $\nabla \cdot \vec{B}=0$\\ 
$\displaystyle \oint \vec{E} \cdot d\vec{s} = -\frac{d\Phi_B}{dt}$  & $\nabla \times \vec{E}=-\frac{d\vec{B}}{dt}$\\ 
$\displaystyle \oint \vec{B} \cdot d\vec{s} = \mu_0 I + \mu_0
\epsilon_0 \frac{d \Phi_E}{dt}$ & $\nabla \times  \vec{B}= \mu_0 J+ \mu_0 \epsilon_0  \frac{d \vec{E}}{dt} $ 
\end{tabular}

\bigskip
{\bf Simple Harmonic Motion \& Waves} \\
\bigskip
$\displaystyle x(t) = A~ \cos(\omega t + \phi_0)~~~(simple~harmonic~motion)$\\
\medskip
$\displaystyle D(x,t) = A~ \sin(k x - \omega t + \phi_0)~~~(traveling~wave)$\\
\medskip
%$\displaystyle v =\frac{dx}{dt} = -\omega A~ sin(\omega t + \phi)$\\
%\medskip
%$\displaystyle a = \frac{dv}{dt} =-\omega^2 A~ cos(\omega t + \phi)$\\
%\medskip
%$\displaystyle x = A~ sin(kx - \omega t)$\\
%\medskip
$\displaystyle k = 2 \pi / \lambda $ \\
\medskip
$\displaystyle \omega = 2 \pi f $ \\
\medskip
$\displaystyle f = \frac{1}{T} $ \\
\medskip
$\displaystyle v = f \lambda $ \\
\medskip
$\displaystyle T = 2 \pi \sqrt{\frac{L}{g}}  ~~~(pendulum)$\\
\medskip
$\displaystyle T = 2 \pi \sqrt{\frac{m}{k}} ~~~(mass \ on\  spring)$\\
\medskip
$\displaystyle \omega = \sqrt{\frac{k}{m}} $\\
\medskip
$\displaystyle U_{spring} = \frac{1}{2} k (\Delta x)^2$\\
\medskip
%$\displaystyle F_{spring} = -k \Delta x$\\ 
%\medskip
$\displaystyle E_{spring} = \frac{1}{2} k A^2$\\
\medskip
%{\bf Waves} \\
$\displaystyle v_{string} = \sqrt{\frac{F_T}{\mu}} $ \\
\medskip
%$\displaystyle \mu = \frac{m}{L}~~(linear~density)$ \\
%\medskip
%$\displaystyle I \propto A^2 $ \\
%\medskip
$\displaystyle I = \frac{P}{4 \pi r^2} ~~~(intensity)$ \\
\medskip
$\displaystyle L = \frac{m \lambda_m}{2} ~~(standing ~ waves ~ on ~ 
string)$ \\ 
\medskip
%$\displaystyle \frac{sin \theta_2}{sin \theta_1} = \frac{v_2}{v_1} $ \\
%\medskip
%$\displaystyle \theta \approx \frac{\lambda}{L} $ \\
%\bigskip

{\bf Sound} \\
\bigskip
$\displaystyle v_s = (331 + 0.60 \rm T)~\rm m/s$ \\
\medskip
$\displaystyle \beta (dB) = 10~\log_{10}\left(\frac{I}{I_0}\right) $ \\ 
\medskip
$\displaystyle f' = f_0~ \left(\frac{v \pm v_o}{v \pm
  v_s}\right)~~(Doppler~effect)$ 
\\ 
\bigskip

{\bf Geometric Optics} \\
\bigskip
%$ \displaystyle \theta_{r}=\theta_{i}$ \\
%\medskip
$\displaystyle n_1~ sin \theta_1 = n_2~ sin \theta_2$ \\
\medskip
%$\displaystyle f = \frac{1}{2} r $ \\
%\medskip
$\displaystyle \frac{1}{s}+\frac{1}{s^{'}}=\frac{1}{f} \ or \ \frac{1}{d_o}+\frac{1}{d_i}=\frac{1}{f} $ \\
\medskip
$\displaystyle m = \frac{h^{'}}{h} = - \frac{s^{'}}{s}$ \\
\medskip
$\displaystyle n = \frac{c}{v_{medium}}$ \\
\medskip
$ \displaystyle \lambda_{medium}=\lambda_{vacuum}/n$ \\
\bigskip

{\bf Light as a wave} \\
\bigskip
$\displaystyle d ~sin \theta_m = m~ \lambda
(double\ slit,~bright~fringes)$ \\  
\medskip
%$\displaystyle d ~sin \theta = (m + \frac{1}{2})~ \lambda  \hfill (double\ slit, \ dark
%\ fringes)$ \\
%\medskip
$\displaystyle a ~sin \theta_p = p ~ \lambda~(single \ slit,
\ dark~fringes)$ \\ 
\medskip
$\displaystyle y_m = L \tan\theta$ \\
\medskip
$\displaystyle w = \frac{2\lambda L}{a}$ \\
\medskip
$\displaystyle \theta = 1.22 \frac{\lambda}{D}~~(Rayleigh~criterion)$ \\
\medskip
%$\displaystyle I = I_0 ~cos^2 \theta $ \\
%\medskip
%$\theta_{c}=\sin^{-1}\left(\frac{n_{2}}{n_{1}}\right)$\\
%$m=-s^{\prime}/s = h^{\prime}/h$\ \ \ (lateral magnification)\\

%$$\frac{n_{1}}{s}+\frac{n_{2}}{s^{'}}=\frac{n_{2}-n_{1}}{R}$$


%$\frac{1}{f}=\left(n-1\right)\left(\frac{1}{R_{1}}-\frac{1}{R_{2}}\right)$\\


{\bf Special Relativity} \\
\medskip
$\displaystyle \gamma = \frac{1}{\sqrt{1 - v^2/c^2}} $ \\
\medskip
% 1D Lorentz Transformations 
$
\left\{
\begin{array}{l}
x'=\gamma (x- v t) \\
t'=\gamma \left( t- vx/c^2 \right) \\
y'=y\\
z'=z\\
\end{array}
\right.
$ \\
\medskip
$\displaystyle \Delta t =  \gamma \Delta t_0 $ \\
\medskip
$\displaystyle L  = \frac{L_0}{\gamma}$ \\
\medskip
$\displaystyle  p = \gamma m_0 v $ \\
\medskip
$\displaystyle  m_{rel} = \gamma m_0 $ \\
\medskip
$\displaystyle  E_0 = m_0 c^2$ \\
\medskip
$\displaystyle  KE = (\gamma - 1) m_0 c^2$ \\
\medskip
$\displaystyle  E = KE + m_0 c^2 = \gamma m_0 c^2$ \\
\medskip
$\displaystyle  E^2 = p^2c^2 + m_0^2 c^4$ \\
\medskip
%$\displaystyle  u = \frac{v + u'}{1 + v u'/c^2}$ \\
%Relativistic velocity transformations:
\[
u_x'=\frac{u_x-v}{1-\frac{v u_x}{c^2}}
\quad  u_y'=\frac{u_y}{\gamma\left(1-\frac{v u_x}{c^2}\right)}
\quad  u_z'=\frac{u_z}{\gamma\left(1-\frac{v u_x}{c^2}\right)} 
\]
%Doppler effect:
$\displaystyle f=\sqrt{\frac{1+\beta}{1-\beta}} f_0 \, , \quad \textrm{(approaching)} $\\
\medskip
$\displaystyle f=\sqrt{\frac{1-\beta}{1+\beta}} f_0 \, , \quad \textrm{(receding)} $\\
\medskip
{\bf Quantum \& Atom} \\
\medskip
$\displaystyle  \lambda_P~T = 2.90\times 10^{-3}~ \rm m~K$ \\
\medskip
$\displaystyle  E = n h f$ \\
\medskip
$\displaystyle  E = h f$ \\
\medskip
$\displaystyle  p = \frac{E}{c} = \frac{h f}{c} = \frac{h}{\lambda}$ \\
\medskip
$\displaystyle  \lambda = \frac{h}{p}$ \\
\medskip
$\displaystyle  r_n = {n^2} a_o $ \\
\medskip
$\displaystyle  E_n = -13.6~eV~\frac{Z^2}{n^2}$ \\
\medskip
%Rydberg-Ritz formula:
$\displaystyle \frac{1}{\lambda_m}=R\left( \frac{1}{m^2} - \frac{1}{n^2}\right) \, , \quad n>m $\\
$R=1.096776\times 10^7 m^{-1}$ for hydrogen. (Lyman,Balmer, Paschen for  $m=1,2,3$)
\medskip
%Heisenberg uncertainty principle
$\displaystyle \Delta x \cdot \Delta  p \geq \frac{\hbar}{2} $ \\
\medskip
$\displaystyle H \psi=i \hbar \frac{d\psi}{dt} $ \\
\medskip
$\displaystyle H=-\frac{\hbar^2}{2m}\frac{d^2 \psi}{dx^2}+V\psi $\\\
\medskip
$\displaystyle E_n=n^2 \frac{\hbar^2 \pi^2}{2mL^2} $  (infinite well)\\ 
\medskip
$\displaystyle E_n=\left(n+\frac{1}{2} \right) \hbar \omega $ (harmonic oscillator)\\ 
\medskip
$\displaystyle E_n^1= \psi^0 H^1 \psi^0 $ (Perturbation Energy)\\
\medskip
multiplicity of states \\
\medskip
compton effect

\newpage
{\bf Thermodynamics and Statistical Mechanics} \\
\bigskip
$\displaystyle \Delta U = Q + W$\\ 
\medskip
$\displaystyle Q=mc\Delta T$\\
\medskip
%$\displaystyle \Delta H = \Delta U + P\Delta V$\\
%$W=-P_{ext}V$\\
$W=-\int P dV$\\
\medskip
$U=NkT$\\
\medskip
$PV=NkT$\\
\medskip
$\displaystyle \beta = \frac{\Delta V/V}{\Delta T}$\\
\medskip
$\displaystyle S = \frac{Q}{\Delta T} = k \ln \Omega$\\
\medskip
$\displaystyle L = \frac{Q}{m}$\\
\medskip
$\displaystyle v_{rms} = \sqrt{\frac{3 k T}{m}} $ \\
\medskip
\noindent{\bf Adiabatic}: no heat exchanged\\
\noindent{\bf Isothermal}: constant temperature\\
\noindent{\bf Isobaric}: constant pressure\\
\bigskip

{\bf Classical Mechanics} \\
\bigskip
%$\displaystyle \mathcal{L} = T - U$ \\
$\displaystyle \mathcal{L}(q,\dot{q},t)  = T - U$ \\
\medskip
$\displaystyle \frac{d}{dt}\left(\frac{\partial \mathcal{L}}{\partial \dot{q}_{i}}\right) - \frac{\partial \mathcal{L}}{\partial q_{i}}$ = 0 \\
\medskip
$\displaystyle \mathcal{H}(q,p) = T + U$ (for a conservative potential) \\
\medskip
$\displaystyle \dot{p} = -\frac{\partial \mathcal{H}}{\partial q} \qquad
\dot{q} = +\frac{\partial \mathcal{H}}{\partial p}$
\bigskip

{\bf Blackbody radiation} 
\bigskip
%Stefan's law
$\displaystyle R=\sigma T^4 \, , \quad \sigma=5.67 \times 10^{-8} W/m^2 K^4 $ \\
\medskip
%Wien's displacement law
$\displaystyle \lambda_m=a/T \, , \quad a=2.898 \times 10^{-3} m\cdot K $\\
\medskip
%Planck's law
$\displaystyle u(\lambda)=\frac{8 \pi h c \lambda^{-5}} {e^\frac{hc}{\lambda k T}-1} $\\
\medskip
%Photoelectric Effect
$\displaystyle e V_0=\left( \frac{1}{2} m v^2\right)=hf-\phi $ \\
\medskip
%Compton Scattering
$\displaystyle \lambda_2-\lambda_1=\frac{h}{mc}(1-\cos \theta)  \, , \quad \textrm{where } \frac{h}{mc}=0.00243\,  \textrm{nm} $\\
\bigskip

{\bf Nuclear  physics} \\
\bigskip
${}^A_Z X_N \to   A=Z+N $ \\
\medskip
$\displaystyle B=Z M_H c^2 +N m_n c^2 - M_A c^2 $ \\
where $M_H=1.007825 u$, $m_n=1.008665u$.\\
\medskip

$\displaystyle N(t)=N_0 e^{-\lambda t} $\\
\medskip
$\displaystyle  \tau=\frac{1}{\lambda} $ \\
\medskip
$\displaystyle  t_{1/2}=ln(2) \tau=0.693 \tau $ \\
\medskip
$\displaystyle \alpha$ decay: ${}^A_{Z>83} X_N \to {}^{A-4}_{Z-2} X_{N-2}+\alpha$. \\
\medskip
$\displaystyle \beta^{-}$ decay: ${}^A_{Z} X_N \to {}^{A}_{Z+1} X_{N-1}+\beta^{-}$.\\
\medskip
$\displaystyle \beta^{+}$ decay: ${}^A_{Z} X_N \to {}^{A}_{Z-1} X_{N+1} +\beta^{+}$.\\
\medskip
{\bf Statistics and Uncertainty} \\
$\sigma = \sqrt{N_{counts}}$ \\

\begin{picture}(140,95)
\put(30,10){\vector(0,1){85}}
\put(0,15){\vector(1,0){145}}
\qbezier(0,17)(20,20)(40,35)
\qbezier(40,35)(55,45)(65,45)
\qbezier(65,45)(75,45)(90,35)
\qbezier(90,35)(110,20)(130,17)
\put(210,40){\makebox(0,0)[r]{$P(x)=\frac{1}{\sigma \sqrt{2 \pi}} e^{-\frac{1}{2} \left( \frac{x-\mu}{\sigma} \right)^2}$}}
\multiput(45,14)(40,0){2}{\multiput(0,0)(0,4){6}{\line(0,1){2}}}
\multiput(65,14)(0,4){8}{\line(0,1){2}}
\put(45,12){\makebox(0,0)[tc]{$\scriptstyle\mu-\sigma$}}
\put(65,12){\makebox(0,0)[tc]{$\scriptstyle\mu\vphantom{-sigma}$}}
\put(85,12){\makebox(0,0)[tc]{$\scriptstyle\mu+\sigma$}}
\put(138,12){\makebox(0,0)[tc]{$x$}}
\end{picture}


\medskip
{\bf Astronomy} \\
Kepler's First Law: planets orbit in elliptical orbits \\
\medskip
Second Law:  planets sweep out equal are in equal time \\
\medskip
Third Law: $p^2 \propto  a^3 $ \\
\medskip
Hubble's Law: $v_{recession} = H_0 d$
\medskip
{ \bf Miscellaneous} \\
orthogonal vectors: $\vec{A} \cdot \vec{B} = 0$
\end{flushleft}
\end{twocolumn}






\end{document}

\section{Electrostatics}% (Ch 25 - 31)}
\begin{itemize}
\item Electric Force % (Ch 25)
\begin{itemize}
\item carriers of charge; triboelectric series
\item know how to use Coulomb's law to calculate the force between
  charged particles
\item conductors and insulators
\item know how to calculate the net force from multiple charges (superposition of charges);
  remember that force is a vector!
\end{itemize}

\item Electric Field (Ch 25-27)
\begin{itemize}
\item electric field is the force per charge; it is also a vector!
\item be able to draw electric field lines around various charge
  distributions 
\item know how to calculate the net electric field from multiple charges (superposition of
  charges); remember that electric field is a vector!
\item electric field of a capacitor
\item know how to calculate the electric field inside and outside a
  long wire using Gauss' Law 
\end{itemize}

\item Electric Potential Energy ($W$ and $U_{elec}$) (Ch 28)
\begin{itemize}
\item electric potential energy is the energy stored in a charge configuration
\item if you have to do work to bring the charges together from
  infinity, then the configuration has positive potential energy
\item electric potential energy is a scalar, so you can add the potential from various charges by just adding the numbers
\item comparison of electric potential energy and gravitational energy
\item you should be able to calculate the net electric potential energy from multiple charges
\end{itemize}

\item Electric Potential ($V$) (Ch 28-29)
\begin{itemize}
\item electric potential is the electric potential energy per charge
\item electric potential is a scalar, so you can add the potential from various charges by just adding the numbers
\item net electric potential from multiple charges
\item calculating potential from electric field and vice versa
\item electric potential of a capacitor
\end{itemize}

\item Circuits (Ch 31)
\begin{itemize}
\item Kirchkoff's loop law for potential - the sum of the potential
  difference must be zero around any closed loop
\item Kirchkoff's junction law for current - the current flowing into
  a junction (where wires split) must be the same as the current
  flowing out of a junction
\item Ohm's Law: $\Delta V = I R$
\item be able to draw a circuit based on a description or picture
\item rules for parallel and series circuits including current,
  voltage and resistance
\begin{itemize}
\item current is the same through all parts of a series circuit
\item the voltage drop across branches of a parallel circuit is the
same
\end{itemize}
\item power dissipation by a resistor; $P = IV = I^2 R$
\item household circuits - review {\it Spring Break Circuits} lab
\end{itemize}
\end{itemize}

\section{Magnetism} %(Ch 32 - 33)}

\begin{itemize}
\item Magnetic Field (Ch 32)
\begin{itemize}
\item magnetic field lines point away from North and toward South
\item Be sure to specify directions and/or rotations clearly.  Use the
  symbols for ``into the page'' ($\otimes$) and ``out of the page''
  ($\odot$).
\item Review the Magnetism worksheets 1-5.
\item Cross product: 
$mag(\vec{A} \times \vec{B}) = A B sin\theta$, where $\theta$ is the
  angle between the two vectors and the direction the direction is
  given by the right-hand-rule:  fingers in direction of $u$, curl
  into $w$, and the cross-product is in the direction indicated by
  thumb 
\item The force on a charged object moving in a magnetic field is
  given by the Lorentz force law: $\vec{F}_m = q \vec{v} \times \vec{B}
  $ The direction is given by the right-hand-rule for the cross
  product, where your fingers go in the direction of $q\vec{v}$, curl
  into direction of \bt, and thumb shows direction of force.
\item Note the units of the $B$-field \bt{} is Tesla, where $\rm 1 T =
  1 (N/C)/(m/s)$. 
\item When a charged particle enters a \emph{uniform} magnetic field,
  it will undergo circular motion, with the magnetic force,
  $\vec{F}_m$, providing the centripetal force:

This is always the case for circular motion: $\displaystyle \sum
\vec{F} = m a = m \frac{v^2}{r}$

The net force is just the magnetic force: $\displaystyle \vec{F}_m = m
\frac{v^2}{r}$

$\displaystyle  q \vec{v} \times \vec{B} = m \frac{v^2}{r}$ 

If you want to figure out how long a charged particle takes to
complete one orbit, use the relation: 
$\displaystyle v = \frac{distance}{time} = \frac{2 \pi r}{T}$
where T (period) is the time to complete one
orbit/rotation/revolution. 
\end{itemize}

\item Magnets and Currents (Ch 32)
\begin{itemize}
\item A magnetic field can exert a force on a current-carrying wire.
  The force on a wire segment that is located within a magnetic field
  is: $\displaystyle \vec{F}_{m, seg} = I\vec{\ell} \times \vec{B} $
  where $\ell$ is the length of wire in the magnetic field and I is the
  current.
\item You can represent a current-carrying loop as a bar magnet (a
  magnetic dipole).  If you curl your fingers in the direction of the
  current, your thumb shows the direction of North.
\item A current-carrying loop will rotate in a magnetic field so that
  the North pole of the loop aligns with the external magnetic field
  lines.
\item Motors take advantage of this fact; they run current through a
  wire loop that is sitting in a magnetic field, and the wire loop
  will rotate.  We did a lab on this - go over it!
\end{itemize}

\item Currents Create Magnetic Fields  (Ch 32)
\begin{itemize}
\item In 1820, Hans Oersted discovered that moving charges create magnetic fields.
\item Similarly, current-carrying wires also create a magnetic field.
\item The magnetic field of a long current-carrying wire is given by:
$\displaystyle B = \frac{\mu_0}{2 \pi} \frac{I}{r}$
where I is the current, and r is the distance from the wire where you
want to measure the magnetic field. 
\item The magnetic field at the center of a current-carrying loop is
  given by: $\displaystyle B = \frac{\mu_0 I}{2 R}$ The magnetic field
  curls around the wire, and the direction is given by the
  right-hand-rule.  Put your right-hand thumb in the direction of the
  current, and your fingers curl in the direction of the magnetic
  field.
\item As far as we know, all magnetism is created from moving charges.  
\end{itemize}

\item Electromagnetic Induction (Ch 33)
\begin{itemize}
\item Faraday's law says that a changing magnetic flux through a
  closed wire loop induces a current in that loop: $\displaystyle
  \mathcal{E} = - \frac{d \Phi_{m}}{dt}$
\item The magnetic flux is the dot product of the magnetic field and
  the area of the loop, so the flux can change by: changing the
  strength of the magnetic field, changing the size of the loop, and/or
  changing the orientation of the loop.  A current is only induced
  when the magnetic flux is changing! 
\item Lenz's Law states that the current induced by Faraday's law
  seeks to {\bf oppose} the change in the magnetic flux.   
\item It's useful to draw the magnetic flux at the beginning and end
  of the problem to determine the direction of the induced current.
\end{itemize}

\section{Maxwell's Equations }% (Ch 34)}
%\item Maxwell's Equations %(Ch 34)
\begin{itemize}
\item $\oint \vec{E}\cdot d\vec{A} = \frac{Q_{enc}}{\epsilon_0} $\\
Gauss's law - charges create electric fields and create a net electric flux through a closed surface.
 \item $\oint \vec{B}\cdot d\vec{A} = 0 \text{Gauss's law for magnetism}$\\
 Gauss's law for magnetism - There are no magnetic monopoles.  Flux out of the surface (from the north pole) equals flux back into the surface(into the south pole).
\item $\displaystyle \oint \vec{E}\cdot d\vec{l} = -\frac{\partial \Phi_B}{\partial t}$\\
Faraday's Law - changing magnetic flux creates an electric field.

\item $\displaystyle \oint \vec{B}\cdot d\vec{l} = \mu_0 I_{\text{enc}} + \mu_0 \epsilon_0 \frac{\partial \Phi_E}{\partial t}$\\
Ampere-Maxwell's Law - current or a changing electric flux creates a magnetic field.
\end{itemize}

\section{Simple Harmonic Motion and Waves (Ch 14 and Ch 20)}

\item Simple Harmonic Motion and Waves (Ch 14)
\begin{itemize}
\item mass on a spring
\item calculation of potential energy, kinetic energy, angular frequency, frequency, period, amplitude
\item calculation of position, velocity, and acceleration as a
  function of time
\end{itemize}

\item Waves (Ch 20)
\begin{itemize}
\item be able to identify longitudinal vs transverse waves
\item know the characteristics of waves: frequency, period,
  wavelength, wave speed
\item know how the characteristics of waves are related to each
  other.  For example, for all waves, $speed = frequency \times wavelength$.
\end{itemize}

\item Standing Waves (Ch 21)
\begin{itemize}
\item know how to calculate  frequency and wavelength for strings and open/closed pipes
\item understand superposition principle for waves and how this leads
  to constructive and destructive interference 
\item understand how waves undergo reflection and refraction
\end{itemize}

\section{Optics (Ch 22 -23)}

\item Wave Optics (Ch 22)
\begin{itemize}
\item understand the principles of interference and diffraction
\item You should be very familiar with Young's two-slit interference experiment; the condition for
  bright (interference) fringes is: $\sin \theta_{m} = m \lambda/d$,
  where $d$ is the slit spacing and $m$ is the \emph{order} of the
  fringe 
\item Understand how light diffracts through a single-slit; the angular position of the dark
  fringes is given by $\sin \theta_{m} = m \lambda/ a$, where $a$ is
  the slit width
\item diffraction gratings produce a similar pattern as the two-slit
  experiment, but the resulting constructing interference bands are
  much narrower.  Diffraction gratings are used to analyze the spectra
  of all sorts of things.  In class, we looked at the spectrum of the
  Sun and the spectrum of emission-line tubes that contains gasses
  such as hydrogen, helium, argon and neon.
\item dispersion (e.g., through a prism) occurs because the index of
  refraction for many materials varies with the wavelength of light.
  This means that red light follows a different path than violet light.
\end{itemize}


\item Ray Optics (Ch 23)
\begin{itemize}
\item law of reflection: angle of incidence equals angle of
  reflection.  Remember that all angles are measured from the normal.
\item index of refraction: $n = c/v$
\item law of refraction (also known as Snell's law): $n_1 sin \theta_1 = n_2 sin \theta_2$
\item be able to identify convex vs concave lenses and mirrors
\item focal length - when parallel light rays enter a lens, they will
  converge at the focal point.  The focal length is the distance from
  the lens to this point.
\item radius of curvature
\item understand how magnification is measured from the heights of the
  image and object.
\item Know how to do graphical ray tracing for a simple convex lens
\item real vs virtual images - understand the difference between these
  two and how to know when a lens will produce a real or virtual image.
\item thin lens formula: $\frac{1}{d_o} + \frac{1}{d_i} =
  \frac{1}{f}$.  Know how to use this formula and how distances are
  measured.  
\end{itemize}

%\item Optical Instruments %(Ch 24)
%\begin{itemize}
%%\item lenses in combination
%\item human eye - review in-class eye activity
%%\item camera
%%\item microscope
%%\item telescope
%\item Rayleigh's criterion; resolution limit of a lens or mirror;
%  calculating the diffraction limit of a system, $ \theta = 1.22
%  \frac{\lambda}{a}$ 
%\end{itemize}
\end{itemize}


\section{Kinematics in One Dimension (Ch 1-2)}
\vspace*{-.15in}
% {\bf Constant velocity}

% You should be able to determine the average velocity of an object in two ways:
% \begin{enumerate}
% \item determine the slope of an position vs. time graph. 
% \item use the equation $v=\Delta x /\Delta t$ 
% \end{enumerate}

% You should be able to determine the displacement of an object in two ways: 
% \begin{enumerate}
% \item find the area under a velocity vs. time graph. 
% \item use the equation $\Delta x=v t  $
% \end{enumerate}

% Given a position vs. time graph, you should be able to: 
% \begin{enumerate}
% \item describe the motion of the object (starting position, direction of motion, velocity) .
% \item draw the corresponding velocity vs. time graph. 
% \item determine the average velocity of the object (slope). 
% \item write the mathematical model which describes the motion. 
% \end{enumerate}


% Given a velocity vs. time graph, you should be able to: 
% \begin{enumerate}
% \item describe the motion of the object (direction of motion, how fast). 
% \item draw the corresponding position vs. time graph.
% \item determine the displacement of the object (area under curve). 
% \item write a mathematical model to describe the motion. 
% \end{enumerate}

\begin{itemize}
\item motion diagrams - these represent an object's position as a
  function of time, kind of like the frames of a movie all
  superimposed on one image.  You should know how to translate verbal description into
  a motion diagram; know how to translate motion diagram into a
  description of the object's position, velocity, and acceleration
\item understand the difference between a vector and a scalar.  For
  example, distance is a scalar and displacement is a vector.  
\item velocity - change in position over change in time, $v_s =
  \frac{ds}{dt}$; average vs. instantaneous velocity
\item acceleration - change in velocity over change in time, $a_s =
  \frac{dv_s}{dt}$; average vs. instantaneous acceleration
\item kinematic chain:$ds/dt = v$, $dv/dt = a$
\item know how to estimate: velocity from position vs time graph,
  acceleration from velocity vs time graph (derivatives)
%\item displacement can be determined from the area under velocity vs time graph
%\item change in velocity is the area under an acceleration vs time graph
\item Know how to use the kinematic equations.
\item Free-fall is an example of a constant acceleration motion.  Remember: the {\em magnitude} of the acceleration of gravity is
  $\rm 9.8~m/s^s$; the {\em sign} of $\vec{g}$ depends on how you
  define your coordinate axes.
\end{itemize}

{\bf Uniform Acceleration}
\begin{itemize}
\item You should be able to determine the instantaneous velocity of an object in two ways: 
\begin{enumerate}
\item determine the slope of the tangent to a position vs. time graph at a given point. 
\item use the mathematical model $v_2 =a t+v_1$
%\item the area under an acceleration vs time graph. 
\end{enumerate}

\item You should be able to determine the displacement of an object two ways: 
\begin{enumerate}
\item find the area under a velocity vs. time curve 
\item use the mathematical model $x_f = x_i + v_i t + 1/2 a t^2$  
\end{enumerate}

\item You should be able to determine the acceleration of an object in five ways: 
\begin{enumerate}
\item find the slope of a velocity vs. time graph 
\item use the mathematical model $a=\Delta v /\Delta t $
\item rearrange the mathematical model $x_f = x_i  + v_i t + 1/2 a t^2$
\item rearrange the mathematical model $v_f = v_i + a t$
\item rearrange the mathematical model $v_f^2 =v_i^2 +2 a \Delta x$ 
\end{enumerate}

\item Given a position vs. time graph, you should be able to: 
\begin{enumerate}
\item describe the motion of the object (starting position, direction of motion, velocity) 
\item draw the corresponding velocity vs. time graph 
\item draw the corresponding acceleration vs. time graph 
\item determine the instantaneous velocity of the object at a given time 
\end{enumerate}

\item Given a velocity vs. time graph, you should be able to: 
\begin{enumerate}
\item describe the motion of the object (direction of motion, acceleration) 
\item draw the corresponding position vs. time graph 
\item draw the corresponding acceleration vs. time graph 
\item write a mathematical model to describe the motion 
\item determine the acceleration 
\item determine the displacement for a given time interval 
\end{enumerate}
\end{itemize}

\vspace*{-.15in}
\section{Motion in a Plane (Ch 3-4)}
\vspace*{-.15in}
\subsection{Vectors }

A vector is a quantity that must be specified by two numbers; a
scalar, in contrast, can be specified using just one number.  

A common example of a vector is position.  To specify where a point is
located on an $x-y$ plane, we must specify both its $x$ and $y$
coordinates.  We can also specify the location of this point using
polar coordinates; the point is some distance $r$ from the origin, and
at some angle $\theta$ as measured counter-clockwise from the $+x$
axis.  When using $x-y$ coordinates to specify a position, we say we
are specifying the {\it components} of the position vector.  We can
also specify the vector using its magnitude (or length) and direction, and this is
analogous to polar coordinates.

Examples of other vectors include: velocity, acceleration, force, and momentum.

Examples of scalars include: mass, speed, and energy.

You should be able to do the following:
\begin{enumerate}
\item Given the magnitude and direction of a vector, break the vector into its x and y components.
\item Given the x and y components of a vector, calculate the magnitude and direction.
\item Sum 2 or more vectors
\begin{enumerate}
\item using the tip-to-tail method
\item by breaking each vector into its components.
\end{enumerate}
\end{enumerate}

\subsection{Projectile Motion}
\begin{itemize}
\item Projectile motion describes all objects that are moving solely
  under the influence of gravity.  
\item The horizontal and vertical motions of a projectile are
  independent.  Remember the demonstration that we did in class where
  one ball was launched horizontally while another ball was dropped
  straight down.  The two hit the ground at the same time,
  demonstrating that the different horizontal motions do not affect
  the vertical motions.  Both balls have identical vertical motions.
\item The verticle motion of a projectile determines its time of flight
  (how {\em long} it is in the air).
\item The horizontal motion of a project determines its range (how {\em far}
  it travels horizontally).
\item When solving projectile motion problems, make a list of all the
  known quantities for both the horizontal and vertical (x and y)
  motions.

\item For an object undergoing projectile motion, you should be able
  to draw position, velocity and acceleration vs. time for both dimensions. 
\item Draw a force diagram for an object undergoing projectile motion. 
\item Given information about the initial velocity and height of a projectile determine 
\begin{enumerate}
\item the time of flight, 
\item the point where the projectile lands 
\item magnitude and direction of velocity at impact 
\item time to reach maximum height
\end{enumerate}
\item Explain what effect the mass of a projectile has on its time of
  flight.
\item If a projectile is launched at an angle $\theta$ above the
  horizontal, you need to break its initial velocity into horizontal
  (x) and vertical (y) components.  Typically, $v_{ix} = v_i cos\theta$
  and $v_{iy} = v_i sin\theta$, where $\theta$ is the angle the
  initial velocity makes with the horizontal.

\end{itemize}


\vspace*{-.15in}
\section{Forces (Ch 5-8)}
\vspace*{-.15in}
\begin{itemize}
\item What is a force?
\item Know the different types of forces: gravity, spring, tension,
  normal, kinetic and static friction.%, drag, thrust
\item Know how to identify the forces that are acting on an object -
  ask yourself:
\begin{enumerate}
\item what is the object touching?
\item what is the object interacting long-range with?
\end{enumerate}
\item Interaction diagram - shows the object and all the other objects
  it is interacting with.  Conceptually, this is the first step in
  isolating the forces acting on an object and drawing the free-body diagram.
\item Free-body diagram - this shows all the forces acting {\em on}
  the object
\item Newton's First Law:  be able to describe this and give examples.  {\it An object at rest or moving at constant velocity continues its current motion unless acted upon by an outside agent (force). }
%\item Newton's Second Law
\item Know how to apply Newton's Second Law to an object that is
  experiencing multiple forces.
\begin{enumerate}
\item draw a force diagram for an object given a written description of the forces acting on it. 
\item resolve forces into x and y components as necessary, then sum
  the forces in each direction.  
\item analyze of the kinematic behavior of the object.  If $\vec{a} =
  0$, the object will be at rest or moving with a constant velocity.
  If $\vec{a} \neq 0$, you can describe its motion as explained below.
\end{enumerate}

\item Newton's Third Law: {\it All forces come in pairs; paired forces
    are equal in magnitude, opposite in direction and act on separate
    bodies.  $F_{AB} = -F_{BA}$}.  
\item Newton's Third Law - know how to determine the acceleration of a {\em
    system} of objects, like two masses connected by a pulley;
  remember the acceleration of two joined objects must be the same.  I
  recommend defining the direction of the acceleration ($\vec{a}$) as
  positive for all objects; forces that are in the direction of acceleration
  are then positive, and forces that are opposite the direction of
  acceleration are negative.


\item linking forces and kinematics - by determing the net force on an
  object, you can calculate its acceleration.  Once you know
  acceleration, you can use the kinematic equations to describe how
  the object moves with time.
\item Distinguish mass versus weight.
\end{itemize}


\subsection{Inclined Plane}
Inclined planes are a common type of two-dimensional problem seen
in general physics.  They
provide a means to practice breaking forces into components in a
coordinate system that is not aligned with the horizontal and vertical
directions.

\begin{itemize}
\item When you analyze the forces for an object on an inclined plane, you
{\it almost always} want to rotate your coordinate system so that the
{\em x}-direction is along the incline and the {\em y}-direction is
perpendicular to the incline.
\item You should know how to break the force of gravity into its components along
  the plane and perpendicular to the plane.  Typically, $F_{g,x} = m g
  sin \theta$ and $F_{g,y} = m g cos \theta$, where $\theta$ is the
  angle between the bottom of the incline and the horizontal
  direction.  You should be able to
  prove this to yourself using geometry.
\item The sum of the forces in the y-direction is equal to zero (so
  long as the object is not accelerating up off the plane).  This
  allows you to solve for the normal force.  (The normal is {\bf not} equal
  to $mg$!!!)
\item If friction ($F_f = \mu F_N$) is involved, use the normal found
  from $\sum F_y = 0$. 
\end{itemize}




\vspace*{-.15in}
\section{Motion in a Circle (Ch 8)}
\vspace*{-.15in}
\begin{itemize}

\item An object that is moving in a circle is accelerating, even if
  its speed is constant, because the direction of its velocity vector
  is changing.
\item Remember the buggy activity.  When the buggy was tied to the
  ringstand, it moved in a circle.  When the string was cut, the buggy
  continued in a straight line that was tangent to the circle.  The
  string was pulling the buggy toward the center of the circle.
\item The direction of the acceleration (and thus the net force) is
  toward the center of the circle.
\item The magnitude of the acceleration for an object moving in a
  circle with constant speed is $a_c = v^2/r$.
\item You should use an {\em r-z} coordinate system for circular
  motion problems, where $r$ points to the center of the circle and
  $z$ is perpendicular to $r$.  (Technically, this coordinate system
  is attached to the object that is rotating.)
\item Vertical circular motion - we looked at a roller coaster and
  pilot in an airplane as examples; make sure you understand how the normal force
  changes in these situations.  The net force in the radial direction
  must be $m v^2/r$, and the normal force will adjust accordingly.  At
  the bottom of a vertical circle, the normal force will have to be
  greater than $F_g$ so that the net force is toward the center of the
  circle.  At the top of a vertical circle, the normal force will work
  with $F_g$ to provide the centripetal force; the normal force will
  go to zero as the speed of the object is descreased (see Exam 2 problem).

\end{itemize}

\vspace*{-.15in}
\section{Momentum (Ch 9)}
\vspace*{-.15in}
\begin{itemize}
\item momentum - $momentum = mass \times velocity$.  The units are $kg
  \ m/s$, and momentum is a vector.  
\item Impulse is defined as the change in momentum.  Impulse can be
  calculated from the area under Force vs. time graph.
\item Conservation of momentum - momentum is conserved for an isolated
  system.  Because momentum is a vector, you have a conservation of
  momentum equation for both the x and y directions, and the two
  directions must be treated separately.  In class, we focused mainly on collisions in one
  dimension, but in lab we did a two-dimensional problem (the accident reconstruction).
\item Understand how to express Newton's Second Law in terms of
  momentum: $\vec{F} = \frac{\Delta \vec{p}}{\Delta t}$.
\item Be able to distinguish between elastic vs. inelastic collision.
  Know what is conserved in each type of collision.
Conservation of momentum governs all collisions.  If a collision is elastic, 
meaning that the two objects bounce and don't stick together, then 
kinetic energy is also conserved.  {\bf Kinetic energy is not conserved in an inelastic 
collision.}
\item Draw before and after pictures when trying to conceptualize
  collisions or explosions.
\end{itemize}

\vspace*{-.15in}
\section{Energy (Ch 10)}
\vspace*{-.15in}
\begin{itemize}
\item Know the different forms of energy - kinetic, potential,
  thermal, source
\item Potential energy - potential energy must be able to be converted
  back to kinetic energy.  Examples include gravitational potential
  energy and the potential energy stored in a spring.
\item Source energy - this is usually some type of chemical energy
  (muscles, gasoline) or solar energy that can be transformed into
  mechanical energy.
\item Conservation of energy - applies to an isolated system
\item Remember that energy is a scalar.  This means that {\em
    direction} does not enter into the energy equations.
\item Know how to use energy bar charts to conceptualize conservation
  of energy problems.
\item The zeropoint for gravitational potential energy is arbitrary,
  so you can set it to be a convenient location.  We can
  only measure a change in potential energy.
\item Springs - Hooke's law, spring constant, potential energy stored in a
  spring ($U_{Spring} = 1/2 k \Delta x^2$).
\item ballistic pendulum - this is a great problem to review because it
  requires you to apply conservation of momentum (collision) and
  conservation of energy (pendulum swinging up to its max height).
\item Dissipative (irreversible) vs. non-dissipative (reversible) interactions.  Understand that
  once energy is converted to thermal energy, it can not go back to
  mechanical energy. (Mechanical energy is the sum of potential and
  kinetic energy.) 
\end{itemize}

 
\vspace*{-.15in}
\section{Work (Ch 11)}
\vspace*{-.15in}

\begin{itemize}
\item Work is energy transferred into or out of a system
\item if the system's energy increases, $W >0$; if the system's energy decreases, $W < 0$.
\item Work kinetic energy theorem: $\Delta K = W_{net}$
\item Work is the area under a Force vs position graph.  You should be
  able to calculate work given a graph of $F \ vs. \ x$.
\item The work done by a force is the component of the force that is in the direction of motion times the displacement.  This can be expressed as a dot product
$$ W = \vec{F}_{\parallel}~ \Delta \vec{r} = \vec{F} \cdot \Delta \vec{r} = F~\Delta r~ cos \alpha $$  where $\alpha$ is the angle between the force and the displacement.
\item A {\em conservative force} is a force for which the work done is independent of the path taken.  Example: gravity
\item A {\em non-conservative force} is a force for which the work done depends on the path taken; the process is irreversible.  Example: friction.  The work done by friction is represented as $\Delta E_{th}$ (change in thermal energy) because the energy goes into heating the objects.
\item Understand how the work done by non-conservative forces affects the conservation of energy equation
$$ \Delta E_{system} = \Delta K + \Delta U + \Delta E_{source} + \Delta E_{th}  = W_{ext}$$
\item You should know how to represent conservation of energy problems using energy bar charts.  Be sure to label the type of energy that each bar represents and whether it is initial or final energy.
\end{itemize}




\vspace*{-.15in}
\section{Rotational Motion (Ch 12)}
\vspace*{-.15in}
\begin{itemize}
\item Center of mass (vector) - for a uniform object (which will
  always be the case for us), the center of mass is at the geometric
  center of the object.  (This is where you would apply the force of
  gravity when making an extended free-body diagram.)
\item Moment of interia (scalar), or rotational inertia, is a measure
  of how difficult it is to rotate an object.  Think of this as the
  rotational equivalent of mass/inertia, which is a measure of how
  difficult it is to accelerate an object.
\item We will give you the moment of inertia for common extended
  objects like a sphere, disk, hoop, and rod.  You should know how to
  calculate the moment of intertia for discrete masses (like Jack and
  Jill on the seesaw) - you multiply each mass by the square of its distance from
  the point of rotation, and then add this for all masses:
\begin{equation}
I = \sum m_i r_i^2
\end{equation}
\item parallel axis theorem - allows you to convert a moment of
  inertia about an object's center of mass to the moment of inertia
  about a parallel axis.
\item You should know the symbol and definition for angular
  displacement ($\theta \equiv theta$), angular velocity ($\omega
  \equiv omega = \Delta \theta/\Delta t$), and angular acceleration
  ($\alpha \equiv alpha = \Delta \omega /\Delta t$).
\item You should understand the connection between angular and linear kinematic
  quantitites and be comfortable converting between the two.
\item You should understand how to convert among rpm, rad/s, rev/s,
  and period.  These are really just unit conversions.
\item Angular quantities (displacement, velocity, acceleration, torque) are
  positive for counterclockwise and negative for clockwise rotation.

\end{itemize}

\subsection{Rotational Energy}

In previous energy considerations, we only considered the kinetic 
energy associated with an object's motion through space, also referred to as
the motion of the center of mass.  We did problems where
a box slides down a frictionless incline.  In this case, the initial 
potential energy ($m g h$) is equal to the final kinetic energy ($\frac{1}{2} m v^2$).

If we change the box to a ball, the motion is different.  The ball is moving
down the incline, just like the box, but the ball is also rolling.  Thus,
 we have a new energy term to consider: the energy associated with rolling.  

We can break up the motion into two parts:  
\begin{enumerate}
\item the kinetic energy associated
with the ball moving down the incline ($\frac{1}{2} m v^2$), and 
\item the kinetic
energy associated with the ball rotating about its axis ($\frac{1}{2} I \omega^2$).  
\end{enumerate}
Thus, when an object rolls down an incline, its initial potential energy goes
into moving the object down the incline {\em and} rolling the object.  The
speed at the bottom of the incline is given by:

$$U^G_i = K_f  $$

$$U^G_i = K_{translation} + K_{rotation} $$

$$m g h = \frac{1}{2} m v_{cm}^2 + \frac{1}{2} I \omega^2 $$


The moment of inertia, I, always has an $r^2$ term. For an object that is rolling without slipping, $\omega = v/r$, so $\omega^2 = v^2/r^2$.  Thus the radius cancels out when we multiply I and $\omega^2$.  
This allows us to rewrite $K_{rotation}$ in terms of velocity, v, and we
can then combine the two kinetic energy terms.

\subsection{Angular Momentum, Conservation of Angular Momentum}
\label{angmomentum}
\begin{itemize}
\item Objects that have motion about a point have angular momentum, $\vec{L}$.
\item Analogous to linear momentum, angular momentum can be defined as the moment of inertia times the angular velocity:
$$ \vec{L} = I \vec{\omega} $$
\item For point particles, we can also define the angular momentum as the cross-product of the radius and momentum vectors:
$$ \vec{L} = \vec{r} \times \vec{p} $$
The magnitude of this angular momentum is given by:
$$ L = r m v sin\theta $$
where $\theta$ is the angle between $\vec{r}$ and $\vec{v}$.
\item Angular momentum can change if an object is experiencing a net torque
$$ \vec{\tau}_{net} = \frac{d \vec{L}}{dt} $$
\item The angular momentum of a system is conserved if there is no net torque.  We can solve this type of problem just like we would for a conservation of momentum problem:
$$ \vec{L}_i = \vec{L}_f $$
We demonstrated this in class when a student spinning on a stool
brings his/her arms in and starts to rotate faster.  In this case, the
student's moment of intertia decreased (her mass moved closer to the
axis of rotation), and her angular velocity increased so that the
product of $I \omega$ remains constant.
\item The direction of angular momentum is given by the
  right-hand-rule, where you curl your fingers in the direction of
  rotation and your thumb shows the direction of angular momentum.
  (This is the same as the direction of angular velocity.)
\end{itemize}

\vspace*{-.15in}
\section{Torque and Static Equilibrium (Ch 12)} 
\vspace*{-.15in}
\begin{itemize}
\item Torque ($\vec{\tau}$; the greek letter tau) is the effectiveness of a force to cause rotation.
\item Torque is a vector.  
\item Torque is the {\em cross-product} of the radius vector or level
  arm $\vec{r}$ and the force $\vec{F}$, where $\vec{r}$ is drawn from
  the point of rotation to where the force is applied..  
$$ \vec{\tau} = \vec{r} \times \vec{F}$$
%\item ${r}$ is the distance
 % from the point of the rotation to the point where the force is applied.
\item The magnitude of the torque is given by the lever arm (the distance from the point of rotation to where the force is applied) times the component of the force that is perpendicular to the lever arm.
$$ \tau = r F_{\perp} = r F sin \phi $$ where $\phi$ (phi) is the angle between the radius and force vectors.  The best approach for applying this is to choose $\phi$ to be the smallest angle between $\vec{r}$ and $\vec{F}$, and then determine the direction of the torque as described below.
\item The direction of the torque is given by the right-hand-rule (using right hand: fingers in the direction of $\vec{r}$, curl fingers into the direction of $\vec{F}$, thumb shows direction of torque).
\item You can also just say that a torque that wants to rotate an object counter-clockwise is positive, clockwise is negative.
\item The rotational version of Newton's Second Law states that the net torque on an object is equal to the moment of intertia times the angular acceleration:
$$ \vec{\tau} = I \vec{\alpha}$$
\item Static equilibrium requires that the net force is zero (so that
  there is no
  acceleration) and the net torque is zero (so that there is no rotation):
$$ \vec{F}_{net} = 0 ~~~ {\rm and} ~~~ \vec{\tau}_{net} = 0$$
\item Note that $\vec{F}_{net} = 0$ is really two equations: $\vec{F}_{net,x} = 0$ and $\vec{F}_{net,y} = 0$ 
\item In static equilibrium, the torque measured about {\em any point}
  must be zero.  This means you can choose your origin at any point
  when calculating the net torque.  Choose wisely!  You almost always
  want to place the origin at the point where one of the forces is
  applied, and the force at the origin will drop out of the torque
  equation.  If you have two unknown forces (like the activity we did
  in class where you had to predict the scale readings when the
  pendulum was placed on a board between the scales), you should put the origin at the position of one of the
  unknown forces.
\end{itemize}


% \vspace*{-.15in}
% \section{Gravity (Ch 13)}
% \vspace*{-.15in}
% \begin{itemize}
% \item You should know the historical development of gravity.
% \item You should know how to use the gravitational force law.  Remember:
% \begin{itemize}
% \item distance is measured from the centers of the two objects
% \item the gravitational force is always attractive
% \end{itemize}
% \item You should be familiar with Kepler's Laws of planetary motion.
%   The second law is a statement of conservation of angular momentum,
%   and the Third Law can be proven by setting the gravitational force
%   equal to $\frac{mv^2}{r}$.
% %\item the Cavendish experiment - how did Cavendish determine the gravitational constant?
% \end{itemize}

\vspace*{-.15in}
\section{How to Study}
\vspace*{-.15in}
\begin{itemize}
\item Read through all the slides and your notes.
\item Refer to book for sections that you don't remember well.
\item Work through whiteboard problems and think-pair-share questions.
\item Work through homework problems.  Be sure to compare your
  work to the solutions that I have posted online.  Make sure you
  understand how to complete each problem from start to finish,
  without refering to the solution!
\item Work through exam questions.  This means that you should be able
  to solve each problem {\em without} referring to the solution!  You
  may need to solve a problem several times using the solution before
  you are able to solve it on your own.  Practice, practice, practice!
\item Review your labs.
\end{itemize}



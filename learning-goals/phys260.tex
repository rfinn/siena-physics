\noindent{\bf Chapter 1: Energy in Thermal Physics}
\begin{itemize}[itemsep=0pt,parsep=0pt,topsep=0pt,partopsep=0pt]
\item Thermal equilibrium. What is temperature?
\item Ideal gas. Relationship between kinetic energy and temperature.
\item The equipartition theorem.
\item Internal energy, heat, and work. The Zeroth to Third laws of thermodynamics.
\item Isothermal and adiabatic compression and expansion.
\item Heat capacity and latent heat. How to determine at what temperature
two objects in thermal contact come to equilibrium.
\item Enthalpy. How is enthalpy different from internal energy?
\end{itemize}

\vspace{0.2cm}
\noindent{\bf Chapter 2: The Second Law}
\begin{itemize}[itemsep=0pt,parsep=0pt,topsep=0pt,partopsep=0pt]
\item Microstates, macrostates, multiplicity.
\item How to calculate the previous quantities for simple systems
like two-state systems and Einstein solids.
\item Interacting systems and how this affects the multiplicity
of states. How is this related to thermal equilibrium?
\item Large numbers and very large numbers. The use of Stirling's
approximation for certain calculations.
\item Multiplicity of states for an ideal gas. Concept of momentum
space and/or phase space.
\item Entropy for simple systems and for an ideal gas.
\end{itemize}

\vspace{0.2cm}
\noindent{\bf Chapter 3: Interactions and Implications}
\begin{itemize}[itemsep=0pt,parsep=0pt,topsep=0pt,partopsep=0pt]
\item Temperature and it's relationship to {\it entropy} and {\it energy}.
\item What is negative temperature and how it was defined.
\item How do we get heat capacity from entropy? What is the ``recipe"
that one could follow to get from entropy to heat capacity.
\item The third law of dynamics.
\item How is magnetization defined? What is magnetization? What is a paramagnet?
\item The thermodynamic identity that relates the change (differentials) of some
quantities to the change in internal energy.
\item What is happening at equilibrium? How do systems evolve toward equilibrium?
\item Diffusive equilibrium and chemical potential.
\item How is thermal physics and statistical mechanics related?
\end{itemize}

\vspace{0.2cm}
\noindent{\bf Chapter 4: Engines and Refrigerators}
\begin{itemize}[itemsep=0pt,parsep=0pt,topsep=0pt,partopsep=0pt]
\item Heat engine. How does it use hot and cold reservoirs to
generate work?
\item How do you calculate the maximum efficiency of an engine.
\item Carnot cycle. What are the different strokes of the cycle?
%\item Refrigerator. How does it use hot and cold reservoirs to
%cool a system?
%\item How do you calculate the efficiency of a refrigerator?
\item Internal combustion engine. What are the different
strokes of the engine?
\item Steam engine. What are the different strokes of the engine.
%\item Evaporative cooling. Throttling. Liquefaction of gases.
\end{itemize}



\vspace{0.2cm}
\noindent{\bf Chapter 5: Free Energy and Chemical Thermodynamics}
\begin{itemize}[itemsep=0pt,parsep=0pt,topsep=0pt,partopsep=0pt]
\item Free energy. Available work.
\item Thermodynamic potentials. Enthalpy. Helmholtz free
energy. Gibbs free energy.
\item The definitions of each and the relationships between
them.
\item Batteries and how much electrical energy could be extracted
from them.
\item Phase changes of pure systems. What is a phase change?
\item How to read a phase diagram. What does it mean?
\item How is Gibbs free energy related to a phase change?
\item How much does the Gibbs free energy vary across
a phase change?
\end{itemize}

\vspace{0.2cm}
\noindent{\bf Chapter 6: Boltzmann Statistics}
\begin{itemize}[itemsep=0pt,parsep=0pt,topsep=0pt,partopsep=0pt]
\item The Boltzmann factor. How is this related to the probability
of finding a system in a particular state?
\item The partition function. How is this calculated?
\item Calculating the average value of some quantity in the system.
\item The equipartition theorem and how it is related to the
Boltzmann factor.
\item The concept of a density function. How to turn it into
a useful value.
\item The Maxwell velocity distribution function.
\item How does the Maxwell distribution function change with varying
temperatures? What doesn't change with varying temperatures?
\end{itemize}


\vspace{0.2cm}
\noindent{\bf Chapter 7: Quantum Statistics}
\begin{itemize}[itemsep=0pt,parsep=0pt,topsep=0pt,partopsep=0pt]
\item The Gibbs factor and the grand partition function.
Why do we need these and not the regular Boltzmann factor
and partition function?
\item Under what conditions do quantum statistics become important?
\item What classifies something as a boson? As a fermion? Be able
to give examples of each.
\item How do fermions and bosons affect the occupancy of energy
levels differently?
\item What is the Fermi distribution? The Bose-Einstein
distribution?
\item What is degeneracy pressure? How does it arise?
\item What is blackbody radiation and what does it have to do
with quantum statistics?
\item What is the cosmic microwave background?
\end{itemize}



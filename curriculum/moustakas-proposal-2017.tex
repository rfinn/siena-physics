\documentclass[12pt,preprint]{aastex}
\usepackage{graphicx, amsmath, fancyhdr, mdwlist, color}

\pagestyle{plain}
%\lhead{ASTR392}
%\chead{Principles of Astrophysics II}
%\rhead{Spring 2017}

\newcommand{\red}[1]{\textcolor{red}{#1}}
\newcommand{\blue}[1]{\textcolor{blue}{#1}}

\setlength\parindent{0pt} % noindent

\begin{document}

\vspace*{-1.2in} 
\begin{center}
{\Large {\bf\sc Proposed Physics Curriculum Changes}} \\
\vspace*{2mm} 
{\bf\sc John Moustakas} \\
\vspace*{2mm} 
\sc {\bf 2017 May 4} \\
\vspace*{0.1in} 
\end{center}

{\large \sc \textbf{Motivation}}
\vspace{-1mm}

%{\bf Maybe a math methods course??}

It goes without saying that we have a strong interest in strenghtening our
program and in serving our students by providing the best possible undergraduate
physics education.  In this document I have reimagined the Physics BS curriculum
to try to address some of the shortcomings I have observed, including: (i) less
than optimal coordination between physics and math coursework and in particular
weak \emph{applied} mathematical skills among our students; (ii) an overly
burdensome major (i.e., too many courses/credits), which makes it challenging
for students to pursue other majors/minors, go abroad, start the physics major
late, etc.; and (iii) a curriculum that does not suitably prepare our students
for the GRE (in the fall of their senior year) and graduate school.  

%In this proposal I have focused just on the Physics BS but I think 

{\bf Please note that this is a strawman proposal meant to engender discussion!}  

%preparing students for graduate school.  This document proposes a large number
%of changes to the Physics BS curriculum, and a strawman Applied Physics BS
%program.

%{\large \sc \textbf{Proposed Physics BS}}
%\vspace{-1mm}

%\begin{deluxetable}{lclc}
%\tablecaption{Proposed Physics BS\label{table:bs}}
%\tablewidth{0pt}
%\tablehead{
%  \colhead{Fall Year 1} & \colhead{} &
%  \colhead{Spring Year 1} & \colhead{} }
%\startdata
\begin{table}[h!]
%\begin{center}
\caption{Proposed Physics BS}
\medskip
\begin{tabular}{lclc}
\hline
{\bf Fall Year 1} &  & {\bf Spring Year 1} &  \\
\hline
\hline
\blue{General Physics I}       & 4 & \blue{General Physics II} & 4 \\
\red{Math for Physicists \& Engineers I}   & 1 & \red{Math for Physicists \& Engineers II} & 1 \\
Calculus I                    & 4 & Calculus II & 4 \\
\blue{Software Tools for Physicists} & 3 &   &  \medskip \\

\hline
{\bf Fall Year 2} &  & {\bf Spring Year 2} &  \\
\hline
\hline
\blue{Electromagnetism I}  & 4 & \blue{Mechanics I} & 4 \\
Electronics         & 4 & Differential Equations & 3 \\
Calculus III        & 4 &  \medskip \\
\hline
{\bf Fall Year 3} &  & {\bf Spring Year 3} &  \\
\hline
\hline
\blue{Quantum Mechanics} & 3 & \blue{Statistical Mechanics} & 3 \\
\blue{Advanced Lab I} & 1 & \blue{Advanced Lab II} & 1 \\
Chemistry I          & 4 & Linear Algebra & 3 \medskip \\
\hline
{\bf Fall Year 4} &  & {\bf Spring Year 4} &  \\
\hline
\hline
\red{Advanced Physics Topics} or & 3 & Experimental Techniques & 2 \\ 
\hspace{3mm} Physics Elective &   &  &  \\ 
\red{Honors Thesis} &   & \red{Honors Thesis} &   \\
%\enddata
%\end{deluxetable}
\end{tabular}
%\end{center}
\end{table}

New or modified courses, in no particular order:
\begin{itemize*}
\item{\underline{\bf Math for Physicists \& Engineers I, II}
  \begin{itemize}
    \item[$\bullet$]{\underline{\em Summary}: New courses, 1 credit each,
      freshman year.}
    \item[$\bullet$]{\underline{\em Rationale}: A perennial problem in our
      calculus-based general physics sequence is the fact that students do not
      learn the basic mathematical tools they need until well into their
      sophomore year.  Meanwhile, teaching math in general physics leaves less
      time for physics.  The goal of the {\em Math for Physicists \& Engineers}
      two-course sequence---which would be taught in close coordination with the
      general physics sections---is to provide students with the mathematical
      tools from calculus, linear algebra, and differential equations needed to
      approach and solve basic physics problems.}
    \item[$\bullet$]{\underline{\em Example topics (unordered)}: vectors,
      matrices, coordinate systems (2D, 3D), differentiation, integration (1D),
      circular motion, cross/scalar product, linear equations, matrix
      diagonalization, div, grad, curl.}
  \end{itemize}
}
\item{\underline{{\bf General Physics I}}
  \begin{itemize}
    \item[$\bullet$]{\underline{\em Summary}: Same basic course but with more
      time spent on all topics.}
    \item[$\bullet$]{\underline{\em Rationale}: Not having to teach vectors,
      differentiation, and circular motion will allow for more time for all
      topics, but especially rigid-body statics and dynamics, as well as gravity
      (which are typically rushed at the end of the semester or not taught at
      all).}
    \item[$\bullet$]{\underline{\em Approximate topics}: kinematics, forces,
      energy, work, momentum, rigid-body statics and dynamics, gravity.}
  \end{itemize}
}
  
\item{\underline{{\bf General Physics II}}
  \begin{itemize}
    \item[$\bullet$]{\underline{\em Summary}: Remove most of electromagnetism
      and add thermodynamics, fluids, and some basic topics from modern
      physics.}
    \item[$\bullet$]{\underline{\em Rationale}: Students frequently struggle
      with electromagnetism because they don't have the requisite mathematical
      background.  Moreover, the fact that we don't teach (general physics
      level) thermodynamics means that time has to be spent in {\em Thermal
        Physics} on these topics.  So the idea here is to push electromagnetism
      (with the exception of the charge model and basic circuits) to students'
      sophomore year, freeing up considerable time for thermodynamics and
      fluids.  In addition, some basic topics from {\em Modern Physics} like the
      Bohr model of the atom and the Schrodinger equation could be introduced.}
    \item[$\bullet$]{\underline{\em Example topics (unordered)}: fluids,
      thermodynamics, waves, optics, charge model, basic circuits, basic
      atomic/modern physics.}
    \item[$\bullet$]{\underline{\em Potential Objections}: How will this impact
      non-physics majors?}
  \end{itemize}
}
  
\item{\underline{{\bf Electromagnetism I}}

  \begin{itemize*}
    \item[$\bullet$]{\underline{\em Rationale}: }
    \item[$\bullet$]{\underline{\em Example topics (unordered)}: 

}
  \end{itemize*}
}
\end{itemize*}


%\begin{deluxetable}{lclc}
%\tablecaption{Proposed Physics BS\label{table:bs}}
%\tablewidth{0pt}
%\tablehead{
%  \colhead{Fall Year 1} & \colhead{} &
%  \colhead{Spring Year 1} & \colhead{} }
%\startdata
\begin{table}[h!]
%\begin{center}
\caption{Proposed Applied Physics BS}
\medskip
\begin{tabular}{lclc}
\hline
{\bf Fall Year 1} &  & {\bf Spring Year 1} &  \\
\hline
\hline
\blue{General Physics I}       & 4 & \blue{General Physics II} & 4 \\
\red{Math for Physicists \& Engineers I}   & 1 & \red{Math for Physicists \& Engineers II} & 1 \\
Calculus I                    & 4 & Calculus II & 4 \\
Software Tools for Physicists & 3 &   &  \\
\hline
{\bf Fall Year 2} &  & {\bf Spring Year 2} &  \\
\hline
\hline
\blue{Electromagnetism I}  & 4 & \blue{Mechanics I} & 4 \\
Electronics         & 4 & Differential Equations & 3 \\
Calculus III        & 4 & \red{Strength of Materials} & 3 \\
\hline
{\bf Fall Year 3} &  & {\bf Spring Year 3} &  \\
\hline
\hline
\red{Mechanics II}    & 3 & \red{Fluid Mechanics} & 3 \\
\blue{Advanced Lab I} & 1 & \blue{Advanced Lab II} & 1 \\
Chemistry I           & 4 &  & 3 \\
\hline
{\bf Fall Year 4} &  & {\bf Spring Year 4} &  \\
\hline
\hline
\red{Advanced Physics Topics} or & 3 & Experimental Techniques & 2 \\ 
\hspace{3mm} Physics Elective &   &  &  \\ 
%\enddata
%\end{deluxetable}
\end{tabular}
%\end{center}
\end{table}



\begin{table}[h!]
%\begin{center}
\caption{Proposed Applied Physics BS}
\medskip
\begin{tabular}{lclc}
\hline
{\bf Fall Year 1} &  & {\bf Spring Year 1} &  \\
\hline
\hline
General Physics I             & 4 & General Physics II & 4 \\
Math for Physicists I         & 1 & Math for Physicists II & 1 \\
Calculus I                    & 4 & Calculus II & 4 \\
Software Tools for Physicists & 3 & Engineering Analysis & 2 \\
\hline
{\bf Fall Year 2} &  & {\bf Spring Year 2} &  \\
\hline
\hline
Electromagnetism I  & 4 & Mechanics I & 4 \\
Electronics         & 4 & Differential Equations & 3 \\
Calculus III        & 4 & Calculus II & 4 \\
\hline
{\bf Fall Year 3} &  & {\bf Spring Year 3} &  \\
\hline
\hline
Quantum Mechanics   & 3 & Statistical Mechanics & 3 \\
Advanced Lab I      & 1 & Advanced Lab II & 1 \\
Chemistry I         & 4 & Experimental Techniques & 2 \\
                    &   & Linear Algebra & 3 \\
\hline
{\bf Fall Year 4} &  & {\bf Spring Year 4} &  \\
\hline
\hline
Physics Elective    & 3 & &  \\
\end{tabular}
%\end{center}
\end{table}




\begin{table}[h!]
%\begin{center}
\begin{tabular}{lcccc}
\hline
 & {\bf Physics} & {\bf Auxiliary} & {\bf Number of} & {\bf Contact} \\ 
{\bf Program} & {\bf Credits/Courses} & {\bf Credits/Courses} & {\bf Courses} & {\bf Hours} \\  
\hline
\hline
Physics BS (Now)      & 36/12 & 32/9  & XX  \\
Physics BS (Proposed) & 31/12 & 26/8  & 52\\ 
Applied Physics BS    & XX/XX & XX/XX & XX  \\
\end{tabular}
%\end{center}
\end{table}


%{\large \sc \textbf{New BS with Various Minors}}
%\vspace{-1mm}
%
%With astro
%
%with data science
%
%with computer science

Perhaps we should define the program before trying to sell it?
The Applied Physics major would be the same as a regular Physics major, except
Add Strength of Materials (3 credits) as a 200 level course
Add Fluids and Heat Transport (1 credit, or 3) also 200 level
Add Electronics II (3 credits including a laboratory)
Change “2 of CM, EM, Q” to “3 of CM, EM, 
Optics, Nano”


\end{document}

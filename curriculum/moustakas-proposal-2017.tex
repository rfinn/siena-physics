\documentclass[12pt,preprint]{aastex}
\usepackage{graphicx, amsmath, fancyhdr, mdwlist, color, longtable}

\pagestyle{plain}
%\lhead{ASTR392}
%\chead{Principles of Astrophysics II}
%\rhead{Spring 2017}

\newcommand{\red}[1]{\textcolor{red}{#1}}
\newcommand{\blue}[1]{\textcolor{blue}{#1}}

\setlength\parindent{0pt} % noindent

\begin{document}

\vspace*{-1.2in} 
\begin{center}
{\Large {\bf\sc Proposed Physics Curriculum Changes and \\ Applied Physics
    Program}} \\ 
\vspace*{2mm} 
{\bf\sc John Moustakas} \\
{\em with tons of input from Graziano, Matt, Mark, John, and Rose} \\
\vspace*{2mm} 
\sc {\bf 2017 May 3} \\
\vspace*{0.1in} 
\end{center}

{\large \sc \textbf{Motivation}}
%\vspace{-1mm}

%{\bf Maybe a math methods course??}

It goes without saying that we have a strong interest in strenghtening our
program and in serving our students by providing the best possible undergraduate
physics education.  In this document I have reimagined the Physics BS curriculum
to try to address some of the areas I would personally like us to try to
improve, including:
\begin{itemize}
  \item[(i)]{There exists less-than-optimal coordination between physics and
    math coursework and, in particular, our students tend to exhibit weak
    \emph{applied} mathematical skills even after taking the full set of
    mathematics courses.}
  \item[(ii)]{I feel that the physics major is overly burdensome in terms of the
    number of courses and credits our students have to take, which makes it
    challenging for students to pursue other majors/minors (mathematics,
    astrophysics, data science, computational science), go abroad, start the
    physics major in their sophomore year, pursue the 3/2 engineering program,
    engage in independent research, and so forth.}
  \item[(iii)]{And finally I feel that the current curriculum does not suitably
    prepare our students for graduate school and particularly for the GRE which
    takes place in the fall of their senior year.}
\end{itemize}
Please note that this is a strawman proposal meant to engender discussion, not
to create any hard feelings whatsoever!  In particular, I have assumed that
existing courses are not ``owned'' by anyone and have only approached this
exercise from a high-level standpoint of what I think will best serve our
students. 

%preparing students for graduate school.  This document proposes a large number
%of changes to the Physics BS curriculum, and a strawman Applied Physics BS
%program.

%\clearpage

%\section*{Proposed Physics BS}\label{sec:bs}
\vspace{2mm}
{\large \sc \textbf{Current Physics BS}}
%\vspace{-2mm}

Before presenting my specific proposal I thought it would be helpful to review
the current curriculum.  Table~\ref{table:currentbs} summarizes the current
suite of courses a ``typical'' BS student would take.  For comparison across the
various proposed programs, the table also shows the number of student {\bf
  Credit Hours (CH)} and {\bf Faculty Contact Hours (FCH)}.\footnote{I have
  assumed three lecture sections and three lab sections of general physics (per
  semester) and one lecture and two lab sections (each) of Modern Physics and
  Electronics~I.}

\newpage
\begin{longtable}{lcclcc}
\caption{Current Physics BS Curriculum}\label{table:currentbs} \\   
\hline
{\bf Fall Year 1} & {\sc ch} & {\sc fch} & {\bf Spring Year 1} & {\sc ch} & {\sc fch} \\
\hline
\hline
General Physics I             & 4 & 18 & General Physics II        & 4 & 18 \\
General Physics Review I      & 0 &  1 & General Physics Review II & 0 & 1  \\
Calculus I                    & 4 & 0 & Calculus II                & 4 & 0  \\
Software Tools for Physicists & 3 & 3 &                            &   &    \\
\multicolumn{1}{r}{\emph{Total:}} & \emph{11} & \emph{22} &
\multicolumn{1}{r}{\emph{Total:}} & \emph{8}  & \emph{19} \\

\hline
{\bf Fall Year 2} & {\sc ch} & {\sc fch} & {\bf Spring Year 2} & {\sc ch} & {\sc fch} \\
\hline
\hline
Modern Physics & 4 & 9 & Computational Physics  & 3 & 3 \\
Electronics I  & 4 & 9 & Thermal Physics        & 3 & 3 \\
Calculus III   & 4 & 0 & Differential Equations & 3 & 0 \\
\multicolumn{1}{r}{\emph{Total:}} & \emph{12} & \emph{18} & 
\multicolumn{1}{r}{\emph{Total:}} & \emph{9}  & \emph{6}  \\ 

\hline
{\bf Fall Year 3} & {\sc ch} & {\sc fch} & {\bf Spring Year 3} & {\sc ch} & {\sc fch}  \\
\hline
\hline
Mechanics I  & 4 & 6 & Electromagnetism I      & 4 & 6 \\
Chemistry I  & 4 & 0 & Experimental Techniques & 2 & 5 \\
Applied Math & 3 & 0 & Linear Algebra          & 3 & 0 \\
\multicolumn{1}{r}{\emph{Total:}} & \emph{11} & \emph{6} &
\multicolumn{1}{r}{\emph{Total:}} & \emph{9} & \emph{9} \\  

\hline
{\bf Fall Year 4} & {\sc ch} & {\sc fch} &  {\bf Spring Year 4} & {\sc ch} & {\sc fch} \\
\hline
\hline
Advanced Lab I  & 1 & 3 & Advanced Lab II  & 1 & 3 \\
Quantum Physics & 3 & 3 & Physics Elective & 3 & 3 \\
\multicolumn{1}{r}{\emph{Total:}} & \emph{4} & \emph{6} &
\multicolumn{1}{r}{\emph{Total:}} & \emph{4} & \emph{6} \\
\hline \\
 &   &   &  & {\sc ch} & {\sc fch} \\  
 &   &   & \multicolumn{1}{r}{{\bf Grand Total:}} & {\bf 68} & {\bf 94} \\  
\end{longtable}

%\section*{Proposed Physics BS}\label{sec:bs}
\vspace{2mm}
{\large \sc \textbf{Proposed Physics BS}}
%\vspace{-2mm}

Table~\ref{table:bs} and the exposition below summarize the new proposed Physics
BS curriculum.  Courses in \red{red} are new while courses in \blue{blue} are
existing courses which would be modified as described below.

\newpage

\begin{longtable}{lcclcc}
\caption{Proposed Physics BS Curriculum}\label{table:bs} \\
\hline
{\bf Fall Year 1} & {\sc ch} & {\sc fch} & {\bf Spring Year 1} & {\sc ch} & {\sc fch}  \\ 
\hline
\hline
\blue{General Physics I}                 & 4 & 18 & \blue{General Physics II} & 4 & 18 \\
\red{Math for Physicists/Engineers I} & 1 &  1 & \red{Math for Physicists/Engineers II} & 1 & 1 \\ 
Calculus I                               & 4 &  0 & Calculus II & 4 & 0 \\
Software Tools for Physicists            & 3 &  3 &             &   &   \\
\multicolumn{1}{r}{\emph{Total:}} & \emph{12} & \emph{22} &
\multicolumn{1}{r}{\emph{Total:}} & \emph{9}  & \emph{19} \\

\hline
{\bf Fall Year 2} & {\sc ch} & {\sc fch} & {\bf Spring Year 2} & {\sc ch} & {\sc fch} \\ 
\hline
\hline
\blue{Electromagnetism I}                  & 4 & 6 & \blue{Mechanics I}     & 4 & 6 \\
\red{Math for Physicists/Engineers III} & 1 & 1 & \red{Math for Physicists/Engineers IV} & 1 & 1 \\ 
Calculus III                               & 4 & 0 & Differential Equations & 3 & 0 \\
Chemistry I                                & 4 & 0 &                        &   &  \\
\multicolumn{1}{r}{\emph{Total:}} & \emph{13} & \emph{7} &
\multicolumn{1}{r}{\emph{Total:}} & \emph{8}  & \emph{7} \\

\hline
{\bf Fall Year 3} & {\sc ch} & {\sc fch} & {\bf Spring Year 3} & {\sc ch} & {\sc fch}  \\ 
\hline
\hline
\blue{Advanced Lab Techniques I} & 1 & 1 & \blue{Advanced Lab Techniques II} & 1 & 1 \\
Quantum Physics                  & 3 & 3 & \blue{Thermal Physics}            & 3 & 3 \\
Electronics I                    & 4 & 9 & Experimental Techniques           & 2 & 5 \\
%Chemistry I          & 4 & Linear Algebra & 3 \medskip \\
\multicolumn{1}{r}{\emph{Total:}} & \emph{8} & \emph{13} &
\multicolumn{1}{r}{\emph{Total:}} & \emph{6}  & \emph{9} \\

\hline
{\bf Fall Year 4} & {\sc ch} & {\sc fch} & {\bf Spring Year 4} & {\sc ch} & {\sc fch} \\ 
\hline
\hline
\blue{Physics Elective} & 3 & 3 & \blue{Physics Elective}  & 3 & 3 \\
\red{Honors Thesis} &   &   & \red{Honors Thesis}  &  &  \\
\multicolumn{1}{r}{\emph{Total:}} & \emph{3} & \emph{3} &
\multicolumn{1}{r}{\emph{Total:}} & \emph{3}  & \emph{3} \\
\hline \\
 &   &   &  & {\sc ch} & {\sc fch} \\  
 &   &   & \multicolumn{1}{r}{{\bf Grand Total:}} & {\bf 62} & {\bf 83} \\  
%\red{Advanced Physics Topics} or & 3 & Experimental Techniques & 2 \\ 
%\hspace{3mm} Physics Elective &   &  &  \\ 
%\red{Honors Thesis} &   & \red{Honors Thesis} &   \\
\end{longtable}

\noindent {\em Description of Proposed Courses and Course Changes}
\vspace{-4mm}
\begin{itemize*}
\item{\underline{\red{Math for Physicists \& Engineers I, II, III, IV}}
  \begin{itemize}
%    \item[$\bullet$]{\underline{\em Summary}: I propose we create two new
%      1-credit courses 
%      freshman year.}
    \item[$\bullet$]{\underline{\em Summary}: A perennial problem in our physics
      courses, especially our calculus-based general physics sequence, is the
      fact that students do not learn the basic mathematical tools they need
      until well into their sophomore and junior year.  Meanwhile, teaching math
      in our courses leaves less time for teaching physics!  The goal of the
      {\em Math for Physicists \& Engineers} four-course sequence---which would
      be taught in close coordination with the concomitant physics courses---is
      to provide students with the mathematical tools from calculus, linear
      algebra, and differential equations they need to solve a wide range of
      elementary physics and engineering problems.  These courses should also be
      structured to serve our applied physics majors (see below).}
    \item[$\bullet$]{\underline{\em Example Topics}: vectors; matrices;
      coordinate systems (1D, 2D, 3D); differentiation; integration (1D); line
      integrals; circular motion; cross/scalar product; linear equations; matrix
      diagonalization; Taylor series expansions; div; grad; curl; Fourier
      analysis; special functions.}
    \item[$\bullet$]{\underline{\em Issues to Consider}: We could consider
      enabling students to test out of one or more of these courses depending on
      their math background (or ability to learn the material on their own!).
      Also note that these courses would be taught by physics faculty.}
  \end{itemize}
}
\item{\underline{\blue{General Physics I}}
  \begin{itemize}
%    \item[$\bullet$]{\underline{\em Summary}: Same basic course but with more
%      time spent on all topics.}
    \item[$\bullet$]{\underline{\em Summary}: Not having to teach vectors,
      differentiation, and circular motion in general physics will allow for
      more time on all topics traditionally taught in this first semester of
      general physicsw, but especially rigid-body statics and dynamics, as well
      as gravity (which are typically rushed at the end of the semester or not
      taught at all).  It's also possible that a chapter on basic fluids could
      also be taught here.}
    \item[$\bullet$]{\underline{\em Example Topics}: kinematics, forces,
      energy, work, momentum, rigid-body statics and dynamics, gravity, and
      fluids (possibly).}
  \end{itemize}
}
  
\item{\underline{\blue{General Physics II}}
  \begin{itemize}
%    \item[$\bullet$]{\underline{\em Summary}: I propose we remove most of
%      electromagnetism from this course and add a significant amount of
%      thermodynamics and some basic topics from modern physics.}
    \item[$\bullet$]{\underline{\em Summary}: Students frequently struggle with
      electromagnetism in this course because they don't have the requisite
      mathematical background yet.  Moreover, the fact that we don't teach
      (general physics level) thermodynamics means that time has to be spent in
      {\em Thermal Physics} on these topics.  Therefore, I propose we push
      electromagnetism (with the exception of the charge model and basic
      circuits) to students' sophomore year, freeing up considerable time for
      thermodynamics and fluids (if not taught in {\em General Physics~I}).  In
      addition, some basic topics from {\em Modern Physics} like the Bohr model
      of the atom and the Schr\"{o}dinger equation could be introduced at the
      end of the semester.}
    \item[$\bullet$]{\underline{\em Example Topics}: fluids, thermodynamics,
      waves, optics, charge model, basic circuits, basic atomic/modern physics.}
    \item[$\bullet$]{\underline{\em Issues to Consider}: How will not teaching
      electromagnetism impact non-physics majors who only take General Physics I
      \& II?  Perhaps they can take General Physics IA \& IIA instead (which
      would not change)?}
  \end{itemize}
}
  
\item{\underline{\blue{Electromagnetism I}}
  \begin{itemize}
    \item[$\bullet$]{\underline{\em Summary}: I propose that this course becomes
      a sophomore-level course which introduces electromagnetism to our students
      for the first time, as well as special relativity.  With the mathematical
      foundations of general physics completed and the concurrent {\em Math for
        Physicists \& Engineers} courses, students should be able to tackle
      Maxwell's equations and electrodynamics at a reasonably rigorous level.}
%   \item[$\bullet$]{\underline{\em Rationale}: ...}
    \item[$\bullet$]{\underline{\em Example Topics}: electrostatics; electric
      potential; current; magnetic field; induction; Maxwell's equations; light;
      special relativity; electrodynamics.}
    \item[$\bullet$]{\underline{\em Issues to Consider}: This course is
      currently a 400-level course in the catalog.}
  \end{itemize}
}

\item{\underline{\blue{Mechanics I}}
  \begin{itemize}
    \item[$\bullet$]{\underline{\em Summary}: I propose that this course becomes
      a sophomore-level course centered around the Lagrangian mechanics
      formalism, supplemented with additional advanced topics not covered in
      general physics.  Hamiltonian mechanics would \emph{not} be introduced
      here.}
    \item[$\bullet$]{\underline{\em Example Topics}: particle
      dynamics; energy/work; conservative forces; rigid-body statics/dynamics;
      orbital mechanics and gravity; oscillators; rotating reference frames;
      Lagrangian mechanics.}
    \item[$\bullet$]{\underline{\em Issues to Consider}: This course perhaps
      more naturally belongs in the fall of the sophomore year but after a year
      of general physics students should really see electromagnetism then.  Note
      that this course is currently a 300-level course in the catalog.}
  \end{itemize}
}

\item{\underline{\blue{Thermal Physics}}
  \begin{itemize}
  \item[$\bullet$]{\underline{\em Summary}: I propose that this course become a
    junior level course which is taken \emph{after} electomagnetism and
    mechanics.  (In principle this could be taught fall of the junior year
    before quantum mechanics but I was trying to keep this as a spring course.)
    In particular, with the solid foundation of thermodynamics (from general
    physics, which of course would need to be reviewed), this course can become
    more advanced and focus more on statistical mechanics and heat transport.}
%   \item[$\bullet$]{\underline{\em Example topics (unordered)}: }
    \item[$\bullet$]{\underline{\em Issues to Consider}: This course is
      currently a 200-level course in the catalog.}
  \end{itemize}
}

\item{\underline{\blue{Physics Electives}}
  \begin{itemize}
    \item[$\bullet$]{\underline{\em Summary}: In addition to the required
      courses tabulated above students would also have to complete two
      additional upper-level physics electives.  Some possibilities include:
      astrophysics, computational physics, optics, nuclear and particle physics,
      simulation and modeling, nanoscience, solid state physics, strength of
      materials, an appropriate cognate courses in data science, etc.  Another
      possibility in the fall of the senior year would be to have an ``advanced
      topics'' course which covers additional advanced topics in
      electromagnetism, mechanics, and quantum physics.}
%    \item[$\bullet$]{\underline{\em Example topics (unordered)}: }
%    \item[$\bullet$]{\underline{\em Issues to Consider}: }
  \end{itemize}
}

\item{\underline{\blue{Advanced Lab Techniques I \& II}}
  \begin{itemize}
    \item[$\bullet$]{\underline{\em Summary}: I have not fully fleshed out this
      idea yet, but my thinking is that we reimagine the senior-level {\em
        Advanced Lab I \& II} sequence as a junior-level hands-on course which
      covers a wide range of laboratory techniques and methods not previously
      taught.  Some possible topics could include advanced statistical methods, 
      3D printing, machine shop methods, scientific writing, experimental
      design, etc.  Students could also start to lay out their {\em Honors
        Thesis} research proposal.}
%    \item[$\bullet$]{\underline{\em Example topics (unordered)}: }
%    \item[$\bullet$]{\underline{\em Issues to Consider}: }
  \end{itemize}
}

\item{\underline{\red{Honors Thesis}}
  \begin{itemize}
    \item[$\bullet$]{\underline{\em Summary}: My thinking here is that we allow
      students the option of carrying out an \emph{independent} research project
      in their senior year which would culminate in a written Honors Thesis.
      The department could then award the ``best'' thesis at the end of the year
      or at graduation.  Note that this sequence would be fully \emph{optional}
      and perhaps it could be combined with the existing \emph{Honors} program
      at Siena.}
%    \item[$\bullet$]{\underline{\em Example topics (unordered)}: }
%    \item[$\bullet$]{\underline{\em Issues to Consider}: }
  \end{itemize}
}
\end{itemize*}

\noindent {\em Additional Notes and Thoughts}
\vspace{-4mm}
\begin{itemize*}
\item{It goes without saying that modern physics and computational physics are
  missing from Table~\ref{table:bs} and that there has been a significant
  reshuffling of the order in which existing courses would be taught.  {\em
    Although my proposed curriculum isn't necessarily optimal, every change was
    made for very specific reasons that I look forward to debating!}}
\item{In this proposal the ``automatic'' math minor is no longer part of the
  major.  However, students interested in graduate school should be strongly
  encouraged to take linear algebra, applied math, complex analysis, and other
  upper-level math electives.  Alternatively, students can pursue a minor in
  data science, astrophysics, chemistry, etc.}
\item{In the interest of maximizing allocation of departmental and College
  resources, I propose we remove {\em General Physics Review} as a ``free''
  course.  One possibility would be to have the students attend one mandatory
  hour of office hours.  The ``freebie'' {\em Advanced Lab} contact hours have
  also been removed.}
\item{There should be a very strong computational thread connecting \emph{all}
  the courses---and especially the labs.  To be specific, we should write down
  the precise learning goals we want our courses to be able to accomplish (in
  terms of what we want our students to be able to \emph{do} computationally),
  and then implement those goals.} 
\item{One final thought: a single semester of electromagnetism may not be
  sufficient, although I'm not sure where a second semester would go.} 
%\item{Labs}
%\item{Advanced lab contact hours.}
%\item{Assessment methods (quizzes vs homework vs online homework)}
\end{itemize*}

%\newpage

\vspace{2mm}
{\large \sc \textbf{Proposed Applied Physics BS}}
%\vspace{-2mm}

The redesigned physics curriculum makes it straightforward to implement an {\em
  applied physics} BS, as summarized in Table~\ref{table:applied}.  Moreover, a
slight variation in this curriculum will much more easily allow our students to
transfer to RPI/Clarkson to pursue either aeronautical or mechanical engineering
as part of the 3/2 program, and should also allow our students to pursue the
masters in mechanical engineering at Clarkson as part of the 4/1 program.

To avoid double-counting, in this table I have only highlighted the
\emph{additional} faculty contact hours (above and beyond the nominal BS) that
the new major would require.

\begin{longtable}{lcclcc}
\caption{Proposed Applied Physics BS Curriculum}\label{table:applied} \\
\hline
{\bf Fall Year 1} & {\sc ch} & {\sc fch} & {\bf Spring Year 1} & {\sc ch} & {\sc fch}  \\ 
\hline
\hline
General Physics I                        & 4 & - & General Physics II & 4 & - \\
Math for Physicists/Engineers I          & 1 & - & Math for Physicists/Engineers II & 1 & -  \\ 
Calculus I                               & 4 & 0 & Calculus II & 4 & 0 \\
Software Tools for Physicists            & 3 & - & \blue{Solidworks / CAD} & 1 & 1 \\ 
\multicolumn{1}{r}{\emph{Total:}} & \emph{12} & \emph{-} &
\multicolumn{1}{r}{\emph{Total:}} & \emph{10}  & \emph{1} \\

\hline
{\bf Fall Year 2} & {\sc ch} & {\sc fch} & {\bf Spring Year 2} & {\sc ch} & {\sc fch} \\ 
\hline
\hline
Electromagnetism I                & 4 & - & Mechanics I                      & 4 & - \\
Math for Physicists/Engineers III & 1 & - & Math for Physicists/Engineers IV & 1 & - \\ 
Calculus III                      & 4 & 0 & Differential Equations           & 3 & 0 \\
Chemistry I                       & 4 & 0 &                                  &   &  \\
\multicolumn{1}{r}{\emph{Total:}} & \emph{13} & \emph{-} &
\multicolumn{1}{r}{\emph{Total:}} & \emph{8} & \emph{-} \\

\hline
{\bf Fall Year 3} & {\sc ch} & {\sc fch} & {\bf Spring Year 3} & {\sc ch} & {\sc fch}  \\ 
\hline
\hline
Advanced Lab Techniques I  & 1 & - & Advanced Lab Techniques II & 1 & - \\
\red{Mechanics II}         & 3 & 3 & Thermal Physics            & 3 & - \\
Electronics I              & 4 & - & \red{Solid State Physics} & 3 & 3 \\ 
%Chemistry I          & 4 & Linear Algebra & 3 \medskip \\
\multicolumn{1}{r}{\emph{Total:}} & \emph{8} & \emph{3} &
\multicolumn{1}{r}{\emph{Total:}} & \emph{7} & \emph{3} \\

\hline
{\bf Fall Year 4} & {\sc ch} & {\sc fch} & {\bf Spring Year 4} & {\sc ch} & {\sc fch} \\ 
\hline
\hline
\red{Strength of Materials} & 3 & 3 & Experimental Techniques & 2 & - \\
Honors Thesis               &   &   & Honors Thesis           &  &  \\
\multicolumn{1}{r}{\emph{Total:}} & \emph{3} & \emph{3} &
\multicolumn{1}{r}{\emph{Total:}} & \emph{2} & \emph{-} \\
\hline \\
 &   &   &  & {\sc ch} & {\sc fch} \\  
 &   &   & \multicolumn{1}{r}{{\bf Grand Total:}} & {\bf 63} & {\bf 10} \\  
\end{longtable}

\noindent {\em Description of Proposed Courses}
\vspace{-4mm}
\begin{itemize*}
\item{\underline{\blue{Solidworks / CAD}}
  \begin{itemize}
    \item[$\bullet$]{\underline{\em Summary}: Although we already offer a
      \emph{Solidworks} course, the scope should be expanded to include CAD so
      it can be used to satisfy the {\em Engineering Graphics and CAD} course at
      RPI, which is an important part of many/most of their engineering
      programs.}
  \end{itemize}
}

\item{\underline{\red{Mechanics II}}
  \begin{itemize}
    \item[$\bullet$]{\underline{\em Summary}: The goal of this course would be
      to supplement the material taught in {\em Mechanics I} to include
      additional content and problems from rigid-body dynamics and statics,
      continuum mechanics (waves, stress and strain), and fluid mechanics.
      Together, the {\em Mechanics I \& II} sequence should satisfy {\em
        Engineering Dynamics} (ENGR2090) course at RPI and the {\em Statics}
      (ES220) and {\em Rigid Body Dynamics} (ES223) courses at Clarkson.}
  \end{itemize}
}

\item{\underline{\red{Solid State Physics}}
  \begin{itemize}
    \item[$\bullet$]{\underline{\em Summary}: Nearly every undergraduate
      engineering program includes a course in material science (and we already
      have a solid state physics course in the catalog, PHYS430).  This course
      should satisfy the {\em Materials Science} (ENGR1600) course at RPI and
      the {\em Materials Science} (ES260) course at Clarkson.}
  \end{itemize}
}

\item{\underline{\red{Strength of Materials}}
  \begin{itemize}
    \item[$\bullet$]{\underline{\em Summary}: This course should satisfy the
      {\em Strength of Materials} (ENGR2530) course at RPI and the {\em Strength
        of Materials} (ES222) course at Clarkson.}
  \end{itemize}
}

\end{itemize*}
 
\noindent {\em Additional Notes and Thoughts}
\vspace{-4mm}
\begin{itemize*}
\item{One question is whether the combination of our general physics curriculum
  (including the laboratory component) together with the {\em Math for
    Physicists/Engineers} sequence is sufficient to satisfy the {\em
    Introduction to Engineering Analysis} (ENGR1100) at RPI.}
\item{Another question is whether our applied physics majors need an additional
  course in advanced thermodynamics, heat transfer, or fluid dynamics.  Clarkson
  appears to have several courses along these lines as part of their
  undergraduate mechanical engineering curriculum than RPI, so perhaps there's
  wiggle room.}
\item{Finally note that the applied BS program (as proposed) requires just one
  additional credit-hour above the nominal physics BS (and both require nearly
  two \emph{fewer} courses than the current curriculum!).} 
\end{itemize*}
 
%{\large \sc \textbf{New BS with Various Minors}}
%\vspace{-1mm}
%
%With astro
%
%with data science
%
%with computer science

%Perhaps we should define the program before trying to sell it?
%The Applied Physics major would be the same as a regular Physics major, except
%Add Strength of Materials (3 credits) as a 200 level course
%Add Fluids and Heat Transport (1 credit, or 3) also 200 level
%Add Electronics II (3 credits including a laboratory)
%Change “2 of CM, EM, Q” to “3 of CM, EM, 
%Optics, Nano”

\end{document}

\documentclass[12pt,preprint]{aastex}
\usepackage{graphicx, amsmath, fancyhdr, mdwlist}

\pagestyle{plain}
%\lhead{ASTR392}
%\chead{Principles of Astrophysics II}
%\rhead{Spring 2017}

\setlength\parindent{0pt} % noindent

\begin{document}

\vspace*{-1.2in} 
\begin{center}
{\Large {\bf\sc Proposed Physics Curriculum Changes}} \\
\vspace*{2mm} 
{\bf\sc J. Moustakas} \\
\vspace*{2mm} 
\sc {\bf April 1, 2017} \\
\vspace*{0.1in} 
\end{center}

{\large \sc \textbf{Motivation}}
\vspace{-1mm}

We have an interest in strenghtening our program and in preparing students for
graduate school.  This document proposes a large number of changes to the
Physics BS curriculum, and a strawman Applied Physics BS program.

New or modified courses, in no particular order:
\begin{itemize*}
\item{\underline{Math for Physicists I, II}--New courses, 1 credit each,
  freshman year. 
  \begin{itemize*}
    \item{{\em Rationale}: A perennial problem in our calculus-based general
      physics sequence is the fact that students do not learn the basic
      mathematical tools they need until well into their sophomore year.
      Meanwhile, teaching the math in general physics leaves less time for
      physics.  The goal of these two new courses---which should be taught in
      close coordination with the general physics sections---is to provide
      students with the mathematical tools needed to approach and solve basic
      physics problems.  In particular, key concepts from not just calculus but
      linear algebra and differential equations can be introduced early enough
      to be useful.}
    \item{{\em Approximate topics}: vectors, matrices, coordinate systems (2D,
      3D), differentiation, integration (1D), circular motion, cross/scalar
      product; linear equations, matrix diagonalization, div, grad, curl.}
  \end{itemize*}
}
\item{\underline{General Physics I} -- {\em Approximate topics}: kinematics,
  forces, energy, momentum, rigid-body statics/dynamics, gravity.  {\em
    Rationale}: Moving the needed mathematics to a separate course will allow
  more time for physics topics at a higher level, making this a truly
  calculus-based physics class.}
\item{\underline{General Physics II} -- {\em Approximate topics}: 

kinematics,
  forces, energy, momentum, rigid-body statics/dynamics, gravity.  {\em
    Rationale}: Moving the needed mathematics to a separate course will allow
  more time for physics topics at a higher level, making this a truly
  calculus-based physics class.}
\end{itemize*}




%{\large \sc \textbf{Proposed Physics BS}}
%\vspace{-1mm}

\begin{table}[h!]
%\begin{center}
\caption{Proposed Physics BS}
\medskip
\begin{tabular}{lclc}
\hline
{\bf Fall Year 1} &  & {\bf Spring Year 1} &  \\
\hline
\hline
{\bf General Physics I}       & 4 & {\bf General Physics II} & 4 \\
{\bf Math for Physicists I}   & 1 & {\bf Math for Physicists II} & 1 \\
Calculus I                    & 4 & Calculus II & 4 \\
Software Tools for Physicists & 3 &   &  \\
\hline
{\bf Fall Year 2} &  & {\bf Spring Year 2} &  \\
\hline
\hline
{\bf Electromagnetism I}  & 4 & {\bf Mechanics I} & 4 \\
Electronics         & 4 & Differential Equations & 3 \\
Calculus III        & 4 & Calculus II & 4 \\
\hline
{\bf Fall Year 3} &  & {\bf Spring Year 3} &  \\
\hline
\hline
Quantum Mechanics    & 3 & {\bf Statistical Mechanics} & 3 \\
{\bf Advanced Lab I} & 1 & {\bf Advanced Lab II} & 1 \\
Chemistry I          & 4 & Experimental Techniques & 2 \\
                     &   & Linear Algebra & 3 \\
\hline
{\bf Fall Year 4} &  & {\bf Spring Year 4} &  \\
\hline
\hline
Physics Elective & 3 & &  \\
\end{tabular}
%\end{center}
\end{table}



\begin{table}[h!]
%\begin{center}
\caption{Proposed Applied Physics BS}
\medskip
\begin{tabular}{lclc}
\hline
{\bf Fall Year 1} &  & {\bf Spring Year 1} &  \\
\hline
\hline
General Physics I             & 4 & General Physics II & 4 \\
Math for Physicists I         & 1 & Math for Physicists II & 1 \\
Calculus I                    & 4 & Calculus II & 4 \\
Software Tools for Physicists & 3 & Engineering Analysis & 2 \\
\hline
{\bf Fall Year 2} &  & {\bf Spring Year 2} &  \\
\hline
\hline
Electromagnetism I  & 4 & Mechanics I & 4 \\
Electronics         & 4 & Differential Equations & 3 \\
Calculus III        & 4 & Calculus II & 4 \\
\hline
{\bf Fall Year 3} &  & {\bf Spring Year 3} &  \\
\hline
\hline
Quantum Mechanics   & 3 & Statistical Mechanics & 3 \\
Advanced Lab I      & 1 & Advanced Lab II & 1 \\
Chemistry I         & 4 & Experimental Techniques & 2 \\
                    &   & Linear Algebra & 3 \\
\hline
{\bf Fall Year 4} &  & {\bf Spring Year 4} &  \\
\hline
\hline
Physics Elective    & 3 & &  \\
\end{tabular}
%\end{center}
\end{table}




\begin{table}[h!]
%\begin{center}
\begin{tabular}{lcccc}
\hline
 & {\bf Physics} & {\bf Auxiliary} & {\bf Number of} & {\bf Contact} \\ 
{\bf Program} & {\bf Credits/Courses} & {\bf Credits/Courses} & {\bf Courses} & {\bf Hours} \\  
\hline
\hline
Physics BS (Now)      & 36/12 & 32/9  & XX  \\
Physics BS (Proposed) & 31/12 & 26/8  & 52\\ 
Applied Physics BS    & XX/XX & XX/XX & XX  \\
\end{tabular}
%\end{center}
\end{table}


{\large \sc \textbf{New BS with Various Minors}}
\vspace{-1mm}

With astro

with data science

with computer science



Perhaps we should define the program before trying to sell it?
The Applied Physics major would be the same as a regular Physics major, except
Add Strength of Materials (3 credits) as a 200 level course
Add Fluids and Heat Transport (1 credit, or 3) also 200 level
Add Electronics II (3 credits including a laboratory)
Change “2 of CM, EM, Q” to “3 of CM, EM, 
Optics, Nano”


\end{document}
